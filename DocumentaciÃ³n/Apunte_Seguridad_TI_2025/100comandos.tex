\chapterimage{ampolletas.png}
\chapter{Comandos SO Linux}


\begin{enumerate}
\item \textbf{ls} - ver contenido del directorio (listar)
\item \textbf{pwd} - ruta del directorio actual
\item \textbf{cd} - cambiar de directorio
\item \textbf{mkdir} - crear nuevo directorio
\item \textbf{mv} - mover archivos / renombrar archivos
\item \textbf{cp} - copiar archivos
\item \textbf{rm} - eliminar archivos
\item \textbf{touch} - crear nuevo archivo en blanco
\item \textbf{rmdir} - eliminar directorio
\item \textbf{cat} - mostrar contenido de archivo en la terminal
\item \textbf{clear} - limpiar la ventana de la terminal
\item \textbf{echo} - mover datos a un archivo
\item \textbf{less} - leer archivo de texto una pantalla a la vez
\item \textbf{man} - mostrar manual de comandos de Linux
\item \textbf{sudo} - permite realizar tareas que requieren permisos de administrador o root
\item \textbf{top} - administrador de tareas en la terminal
\item \textbf{tar} - se utiliza para archivar múltiples archivos en un tarball
\item \textbf{grep} - se utiliza para buscar palabras en archivos específicos
\item \textbf{head} - ver las primeras líneas de cualquier archivo de texto
\item \textbf{tail} - ver las últimas líneas de cualquier archivo de texto
\item \textbf{diff} - compara el contenido de dos archivos línea por línea
\item \textbf{kill} - se utiliza para terminar programas no responsivos
\item \textbf{jobs} - muestra todos los trabajos actuales junto con sus estados
\item \textbf{sort} - es una utilidad de línea de comandos para ordenar líneas de archivos de texto
\item \textbf{df} - información sobre el disco del sistema
\item \textbf{du} - verificar cuánto espacio ocupa un archivo o directorio
\item \textbf{zip} - para comprimir tus archivos en un archivo zip
\item \textbf{unzip} - para extraer los archivos comprimidos de un archivo zip
\item \textbf{ssh} - una conexión segura y cifrada entre dos hosts sobre una red insegura
\item \textbf{cal} - muestra el calendario
\item \textbf{apt} - herramienta de línea de comandos para interactuar con el sistema de empaquetado
\item \textbf{alias} - atajos personalizados utilizados para representar un comando
\item \textbf{w} - información del usuario actual
\item \textbf{whereis} - se utiliza para localizar los archivos binarios, fuente y de página de manual
\item \textbf{whatis} - se utiliza para obtener una descripción de una línea de la página del manual
\item \textbf{useradd} - se utiliza para crear un nuevo usuario
\item \textbf{passwd} - se utiliza para cambiar la contraseña del usuario actual
\item \textbf{whoami} - imprimir el usuario actual
\item \textbf{uptime} - imprimir la hora actual cuando la máquina comienza
\item \textbf{free} - imprimir información del espacio libre en disco
\item \textbf{history} - imprimir el historial de comandos utilizados
\item \textbf{uname} - imprimir información detallada sobre tu sistema Linux
\item \textbf{ping} - para verificar el estado de conectividad a un servidor
\item \textbf{chmod} - para cambiar permisos de archivos y directorios
\item \textbf{chown} - para cambiar la propiedad de archivos y directorios
\item \textbf{find} - usar find para buscar archivos y directorios
\item \textbf{locate} - se utiliza para localizar un archivo, al igual que el comando de búsqueda en Windows
\item \textbf{ifconfig} - imprimir información de la dirección IP
\item \textbf{ip a} - similar a ifconfig pero con impresión más corta
\item \textbf{finger} - proporciona un resumen breve de información sobre un usuario
\end{enumerate}
