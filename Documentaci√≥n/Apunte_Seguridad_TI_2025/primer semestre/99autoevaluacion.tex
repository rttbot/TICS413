\chapterimage{ia.png}
\chapter*{Autoevaluación}


\section*{General}

\begin{flushright}
   \textit{x - 1pt}
\end{flushright}
%\vspace{20px}

\noindent Describir cómo el acceso no autorizado puede llevar a incidente de datos (data breach).\\


\noindent Relacionar cómo los requerimientos la disponibilidad, integridad y confidencialidad afectan los 7 dominios de una infraestructura de TI típica.\\


\noindent Describir riesgo, amenazas y vulberabilidades comunes a estos dominios.\\


\noindent Identificar un enfoque de seguridad por capas a través de los 7 dominios.\\


\noindent Desarrollar un marco de trabajo de una política de seguridad para TI para ayudar a reducir amenazas y vulnerabilidades comunes .\\


\noindent Relacionar cómo un estándar de clasificación de datos afecta los siete dominios.\\



\noindent Entender los principios de la gestión de riesgos.\\


\noindent Distinguir entre riesgo, amenaza y vulnerabilidad.\\


\noindent Identificar riesgos comunes de una infraestructura de TI.\\


\noindent Reconocer cómo las vulnerabilidades son usadas para explotar una infraestructura TI.\\








\section*{Seguridad Ofensiva}

\begin{flushright}
   \textit{Seguridad Ofensiva - 1pt}
\end{flushright}
%\vspace{20px}

\noindent Justifique:  la seguridad ofensiva es el proceso de irrumpir en los sistemas informáticos, explotar errores de software y encontrar lagunas en las aplicaciones para obtener acceso no autorizado a ellas.\\


\begin{flushright}
   \textit{Seguridad Ofensiva - 1pt}
\end{flushright}
%\vspace{20px}

\noindent Justifique: 
seguridad defensiva,  es el proceso de proteger la red y los sistemas informáticos de una organización mediante el análisis y la seguridad de cualquier amenaza digital potencial.



\begin{flushright}
   \textit{Seguridad Ofensiva - 1pt}
\end{flushright}
%\vspace{20px}

\noindent Comente - Ejemplifique: 
La seguridad ofensiva se enfoca en una cosa: irrumpir en los sistemas. La penetración en los sistemas se puede lograr mediante la \textbf{explotación de errores}, el \textbf{abuso de configuraciones inseguras} y el \textbf{aprovechamiento de políticas de control de acceso no aplicadas}, entre otras cosas. %Los equipos rojos y los probadores de penetración se especializan en seguridad ofensiva.
\\



\section*{herramientas}
\begin{flushright}
   \textit{ Web - 1pt}
\end{flushright}
%\vspace{20px}

\noindent ¿Cómo comando $gobuster$ permite encontrar páginas escondidas cuando el atacante ya está dentro de la red?. Cómo funciona el diccionario de palabras que típicamente se provee. Relación con los status code. \\


\section{Otros}
 \subsection{test}
 1. Tomar medidas para prevenir o reducir el impacto de un evento.


Mitigación
   Mitigación

2. Transmisión del riesgo a un tercero.


Transferencia
   Transferencia

3. Cese de la actividad riesgosa para eliminar la probabilidad de que ocurra un evento.


Evitación
   Evitación

4. Ignorar los riesgos y continuar con las actividades riesgosas.


Aceptación
   Aceptación

5. Una debilidad o defecto inherente.


Vulnerabilidad
   Vulnerabilidad

6. Una persona o entidad que deliberadamente toma acción para explotar un objetivo.


Amenaza
   Amenaza

7. Algo de valor que es propiedad de una organización, incluido el hardware físico y la propiedad intelectual.


Activo
   Activo



Haga coincidir los siguientes elementos con los controles administrativos, físicos o técnicos: (D1.3, L1.3.1)

1. Póliza de uso aceptable


Control administrativo
   Control administrativo

2. Lector de credenciales


Control Técnico
   Control físico

3. Señal de alto en el estacionamiento


Control físico
   Control físico

4. Procedimientos de Operaciones de Emergencia


Control administrativo
   Control administrativo

5. Lista de control de acceso


Control Técnico
   Control Técnico

6. Cerradura de puerta


Control físico
   Control físico

7. Capacitación para la concientización de los empleados


Control administrativo
    Control administrativo


1. Esto puede proteger la información en un archivador para que no sea vista por personas no autorizadas (confidencialidad) y evitar que se modifique cualquier documento (integridad).


Cerradura de la puerta
   Cerradura de la puerta

2. Este es abstracto, pero podría estar relacionado con la disponibilidad, porque cuanto antes funcione, más datos permanecerán disponibles. 



   Extintor de incendios

3. Esto puede proporcionar confidencialidad al proteger los datos del acceso no autorizado y la integridad de los cambios no autorizados. Incluso podría ampliarse para proporcionar disponibilidad si más de una persona necesita acceso de emergencia compartido a la información. 


Política de contraseñas
   Política de contraseñas

4. Esto puede proporcionar confidencialidad al proteger los datos del acceso no autorizado y la integridad de los cambios no autorizados. Incluso podría ampliarse para proporcionar disponibilidad si más de una persona necesita acceso de emergencia compartido a la información.



   Cifrado

5. Esto protege la disponibilidad al garantizar el acceso continuo a los sistemas durante un corte de energía.


Generador
   Generador

6. En general, esto se asociaría con la confidencialidad y la gestión de la identidad, pero se podría argumentar para los tres, lo mismo que una política de contraseñas. 



   Biometría

   \subsection{test}
1.Tu respuesta
Políticas
son los documentos de gobierno de más alto nivel en una organización, generalmente aprobados y emitidos por la gerencia, generalmente para respaldar una iniciativa de cumplimiento. (D1, L.1.5)

2. Un profesional de la seguridad que necesita instrucciones paso a paso para completar una tarea de aprovisionamiento podría usar un Tu respuesta 
Procedimiento
para asegurarse de que están realizando la tarea de manera consistente. (D1, L.1.5)

3. Marcos, o
Estándares
a menudo son ofrecidos por organizaciones de terceros y cubren objetivos específicos de asesoramiento o cumplimiento. (D1, L.1.5)

4. Usualmente por mandato de una agencia gubernamental, Tu respuesta 
Reglamentos y Leyes
son un conjunto de reglas que todos deben cumplir y generalmente conllevan sanciones monetarias por su incumplimiento. (D1, L.1.5)



\begin{comment}
    capítulo IV
Pregunta 1		1 / 1 punto
Está trabajando en la oficina de seguridad de su organización. Recibe una llamada de un usuario que ha intentado iniciar sesión en la red varias veces con las credenciales correctas, sin éxito. Este es un ejemplo de un(a)_______. (D2, L2.1.1)

Opciones de preguntas:

A) 

Emergencia


B) 

Evento


C) 

Política


D) 

Desastre


Ocultar 1 comentarios sobre preguntas
Correcto. El usuario informó que ocurrió algo medible; en este punto, no está claro si se trata de una ocurrencia normal o algo que presenta un impacto adverso, por lo que la mejor descripción es evento.

Pregunta 2		1 / 1 punto
Está trabajando en la oficina de seguridad de su organización. Recibe una llamada de un usuario que intentó iniciar sesión en la red varias veces con las credenciales correctas, sin éxito. Después de una breve investigación, determina que la cuenta del usuario se ha visto comprometida. Este es un ejemplo de un(a)_______. (D2, L2.1.1)

Opciones de preguntas:

A) 

Gestión de riesgos


B) 

Detección de incidentes


C) 

Malware 


D) 

Desastre


Ocultar 2 comentarios sobre preguntas
Correcto. El reporte del usuario y la posterior identificación del problema constituyen la detección de incidentes.

Pregunta 3		1 / 1 punto
Una entidad externa ha intentado obtener acceso al entorno de TI de una organización sin la debida autorización. Este es un ejemplo de un(a) _________. (D2, L2.1.1)

Opciones de preguntas:

A) 

Explotar


B) 

Intrusión


C) 

Evento


D) 

Malware


Ocultar 3 comentarios sobre preguntas
Correcto. Una intrusión es un intento, exitoso o no, de obtener acceso no autorizado.

Pregunta 4		1 / 1 punto
Al responder a un incidente de seguridad, su equipo determina que la vulnerabilidad que se explotó no era ampliamente conocida por la comunidad de seguridad y que actualmente no existen definiciones/listados conocidos en bases de datos o colecciones de vulnerabilidades comunes. Esta vulnerabilidad y exploit podría llamarse ______. (D2, B 2.1.1)

Opciones de preguntas:

A) 

Malware


B) 

Crítico


C) 

fractales


D) 

día cero


Ocultar 4 comentarios sobre preguntas
Correcto. Un exploit de día cero es un ataque que utiliza una vulnerabilidad que no es muy conocida en la industria en el momento del descubrimiento.

Pregunta 5		0 / 1 punto
¿Verdadero o falso? El departamento de TI es responsable de crear el plan de continuidad del negocio de la organización. (D2, L2.2.1)

Opciones de preguntas:
	a) Verdadero
	b) Falso

Ocultar 5 comentarios sobre preguntas
Incorrecto. Los miembros de toda la organización, no solo TI, deben participar en la creación del BCP para garantizar que todos los sistemas, procesos y operaciones se tengan en cuenta en el plan.

Pregunta 6		1 / 1 punto
El esfuerzo de Continuidad del negocio para una organización es una forma de garantizar que las funciones críticas de ______ se mantengan durante un desastre, una emergencia o una interrupción del entorno de producción. (D2, B 2.2.1)

Opciones de preguntas:

A) 

Negocio


B) 

Técnica


C) 

TI


D) 

Financiero


Ocultar 6 comentarios sobre preguntas
Correcto. El esfuerzo de continuidad comercial está diseñado para garantizar que las funciones comerciales críticas continúen durante los períodos de posible interrupción.

Pregunta 7		1 / 1 punto
¿Cuál de los siguientes es muy probable que se utilice en un esfuerzo de recuperación ante desastres (DR)? (D2, B 2.3.1)

Opciones de preguntas:

A) 

perros guardianes


B) 

Copias de seguridad de datos


C) 

Personal de contrato


D) 

Soluciones antimalware


Ocultar 7 comentarios sobre preguntas
Correcto. La restauración a partir de copias de seguridad suele ser especialmente útil durante un esfuerzo de recuperación ante desastres.

Pregunta 8		1 / 1 punto
¿Cuál de los siguientes se asocia a menudo con la planificación de DR? (D2, B 2.3.1)

Opciones de preguntas:

A) 

listas de control


B) 

Cortafuegos


C) 

Detectores de movimiento


D) 

No repudio


Ocultar 8 comentarios sobre preguntas
Correcto. Tanto las actividades de BC como las de DR generalmente incluyen listas de verificación para las personas que participan en el esfuerzo.

Pregunta 9		0 / 1 punto
¿Cuál de estas actividades se asocia a menudo con los esfuerzos de DR? (D2, L2.3.1)

Opciones de preguntas:

A) 

Empleados que regresan a la ubicación de producción principal


B) 

Ejecutar soluciones antimalware


C) 

Exploración del entorno de TI en busca de vulnerabilidades


D) 

Hazañas de día cero


Ocultar 9 comentarios sobre preguntas
Incorrecto. Los escaneos de vulnerabilidades son una parte regular de las prácticas de seguridad de TI, pero no suelen estar asociados con los esfuerzos de recuperación ante desastres.

Pregunta 10		1 / 1 punto
¿Cuál de estos componentes es probable que sea fundamental para cualquier esfuerzo de recuperación ante desastres (DR)? (D2, L2.3.1)

Opciones de preguntas:

A) 

Enrutadores


B) 

Computadoras portátiles


C) 

Cortafuegos


D) 

Copias de seguridad


\end{comment}