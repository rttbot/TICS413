
\chapterimage{map.png}
\chapter{Profesionales de la Ciberseguridad y certificaciones}
\vspace{95px}

%capitulo 14 

\begin{flushright}
    \textit{root-me}
\end{flushright}


Perfiles de seguridad ofensiva acorde a root-me: 
\begin{itemize}
\item  \textbf{analista de seguridad}: \textit{responsable de mantener la seguridad de los datos de una organización.}\\
Responsable de mantener la seguridad de los datos de una organización.
%
Los analistas de seguridad son parte integral de la construcción de medidas de seguridad en todas las organizaciones para proteger a la empresa de los ataques. Los analistas exploran y evalúan las redes de la empresa para descubrir datos procesables y recomendaciones para que los ingenieros desarrollen medidas preventivas. Este rol de trabajo requiere trabajar con varias partes interesadas para comprender los requisitos de seguridad y el panorama de seguridad.\\
%
\begin{itemize}
    \item \textbf{Responsabilidades}
Trabajar con varias partes interesadas para analizar la seguridad cibernética en toda la empresa.
Recopilar informes continuos sobre la seguridad de las redes, documentando los problemas de seguridad y las medidas tomadas en respuesta.
Desarrolle planes de seguridad, incorporando investigaciones sobre nuevas herramientas y tendencias de ataque, y las medidas necesarias en todos los equipos para mantener la seguridad de los datos.
\item \textbf{Rutas de aprendizaje}
Las rutas de aprendizaje de  le brindarán tanto el conocimiento técnico fundamental como la experiencia práctica, lo cual es crucial para convertirse en un analista de seguridad exitoso.
\textbf{Rutas de aprendizaje en TryHackMe}:
- Introducción a la Seguridad Cibernética
- Seguridad previa
- SOC Nivel 1
\end{itemize}


\begin{figure}
    \centering
    \includegraphics{}
    \caption{Caption}
    \label{fig:enter-label}
\end{figure}



\item \textbf{Ingeniero de Seguridad}: \textit{responsable de mantener la seguridad de los datos de una organización.}\\
%
Diseñar, monitorear y mantener controles, redes y sistemas de seguridad para ayudar a prevenir ataques cibernéticos
%
Los ingenieros de seguridad desarrollan e implementan soluciones de seguridad utilizando datos de amenazas y vulnerabilidades, a menudo obtenidos de miembros de la fuerza laboral de seguridad. Los ingenieros de seguridad trabajan para eludir una variedad de ataques, incluidos los ataques de aplicaciones web, las amenazas de red y las tendencias y tácticas en evolución. El objetivo final es retener y adoptar medidas de seguridad para mitigar el riesgo de ataque y pérdida de datos.\\
%
\begin{itemize}
\item \textbf{Responsabilidades}
Pruebas y detección de medidas de seguridad en todo el software.
Supervise las redes y los informes para actualizar los sistemas y mitigar las vulnerabilidades
Identificar e implementar los sistemas necesarios para una seguridad óptima.\\ 
\item \textbf{Rutas de aprendizaje}
Las rutas de aprendizaje de TryHackMe le brindarán tanto el conocimiento técnico fundamental como la experiencia práctica, lo cual es crucial para convertirse en un ingeniero de seguridad exitoso.
%
SOC Nivel 1 - 
Probador de penetración JR - 
Pentesting ofensivo. 

\end{itemize}
\item \textbf{Respondedor de Incidentes: }\textit{Identifica y mitiga los ataques mientras las operaciones de los atacantes aún se están desarrollando.}\\
Los respondedores de incidentes responden de manera productiva y eficiente a las infracciones de seguridad. Las responsabilidades incluyen la creación de planes, políticas y protocolos para que las organizaciones los promulguen durante y después de los incidentes. Esta es a menudo una posición altamente presionada con evaluaciones y respuestas requeridas en tiempo real, a medida que se desarrollan los ataques. Las métricas de respuesta a incidentes incluyen MTTD, MTTA y MTTR, mientras tanto para detectar, reconocer y recuperarse (de los ataques). El objetivo es lograr una respuesta rápida y efectiva, conservar la posición financiera y evitar implicaciones negativas de incumplimiento. En última instancia, los respondedores de incidentes protegen los datos, la reputación y la situación financiera de la empresa de los ataques cibernéticos.\\
%

\begin{itemize}\item \textbf{Responsabilidades}
Desarrollar y adoptar un plan de respuesta a incidentes completo y factible
Mantener sólidas mejores prácticas de seguridad y apoyar las medidas de respuesta a incidentes
Informes posteriores al incidente y preparación para futuros ataques, considerando los aprendizajes y las adaptaciones para tomar de los incidentes.\\
\item \textbf{Las rutas de aprendizaje} de TryHackMe le brindarán tanto el conocimiento técnico fundamental como la experiencia práctica, lo cual es crucial para convertirse en un Respondedor de incidentes exitoso. - 
SOC Nivel 1

\end{itemize}
\item \textbf{Examinador Forense:} \textit{Responsable de utilizar el análisis forense digital para investigar incidentes y delitos.}\\
Si te gusta jugar a los detectives, este podría ser el trabajo perfecto. Si trabaja como parte de un departamento de aplicación de la ley, se concentraría en recopilar y analizar evidencia para ayudar a resolver crímenes: acusar a los culpables y exonerar a los inocentes. Por otro lado, si su trabajo se enmarca en la defensa de la red de una empresa, utilizará sus habilidades forenses para analizar incidentes, como violaciones de políticas.
%

\begin{itemize}\item \textbf{Responsabilidades}
Recopile evidencia digital mientras observa los procedimientos legales
Analizar evidencia digital para encontrar respuestas relacionadas con el caso
Documente sus hallazgos e informe sobre el caso.

\end{itemize}
\item \textbf{Analista de Malware:} \textit{Analiza todos los tipos de malware para obtener más información sobre cómo funcionan y qué hacen.} El trabajo de un analista de malware consiste en analizar programas sospechosos, descubrir qué hacen y redactar informes sobre sus hallazgos. Un analista de malware a veces se denomina ingeniero inverso, ya que su tarea principal gira en torno a convertir programas compilados de lenguaje de máquina a código legible, generalmente en un lenguaje de bajo nivel. Este trabajo requiere que el analista de malware tenga una sólida experiencia en programación, especialmente en lenguajes de bajo nivel como el lenguaje ensamblador y el lenguaje C. El objetivo final es conocer todas las actividades que realiza un programa malicioso, saber cómo detectarlo y denunciarlo.\\
%

\begin{itemize}\item \textbf{Responsabilidades}
Realizar análisis estáticos de programas maliciosos, lo que implica ingeniería inversa
Realice análisis dinámicos de muestras de malware mediante la observación de sus actividades en un entorno controlado
Documentar e informar todos los hallazgos.
\end{itemize}
\item \textbf{Probador de Penetración:} \textit{Responsable de probar los productos tecnológicos en busca de lagunas de seguridad.} Es posible que vea pruebas de penetración denominadas pentesting y piratería ética. La función de trabajo de un probador de penetración es probar la seguridad de los sistemas y el software dentro de una empresa; esto se logra mediante los intentos de descubrir fallas y vulnerabilidades a través de la piratería sistematizada. Los probadores de penetración explotan estas vulnerabilidades para evaluar el riesgo en cada caso. Luego, la empresa puede tomar estos conocimientos para rectificar problemas y evitar un ciberataque en el mundo real.
\\

\begin{itemize}\item \textbf{Responsabilidades}
Realizar pruebas en sistemas informáticos, redes y aplicaciones basadas en web.
Realice evaluaciones de seguridad, auditorías y analice políticas
Evaluar e informar sobre conocimientos, recomendando acciones para la prevención de ataques.\\
\item \textbf{Rutas de aprendizaje}
Las rutas de aprendizaje de TryHackMe le brindarán tanto el conocimiento técnico fundamental como la experiencia práctica, lo cual es crucial para convertirse en un probador de penetración exitoso.
Probador de penetración JR
Pentesting ofensivo
\end{itemize}
\item \textbf{Equipo Rojo: }\textit{Juega el papel de un adversario, ataca a una organización y proporciona comentarios desde la perspectiva de los enemigos.} \\
Los miembros del equipo rojo comparten similitudes con los probadores de penetración, con un rol de trabajo más específico. Los evaluadores de penetración buscan descubrir muchas vulnerabilidades en los sistemas para mantener la defensa cibernética en buen estado, mientras que los equipos rojos se promulgan para probar las capacidades de detección y respuesta de la empresa. Este rol de trabajo requiere imitar las acciones de los ciberdelincuentes, emular ataques maliciosos, retener el acceso y evitar la detección. Las evaluaciones del equipo rojo pueden durar hasta un mes, normalmente por un equipo externo a la empresa. A menudo, se adaptan mejor a las organizaciones que cuentan con programas de seguridad maduros.\\
%

\begin{itemize} \item \textbf{Responsabilidades}
Emule el papel de un actor de amenazas para descubrir vulnerabilidades explotables, mantener el acceso y evitar la detección
Evaluar los controles de seguridad de las organizaciones, la inteligencia de amenazas y los procedimientos de respuesta a incidentes
Evalúe e informe sobre conocimientos, con datos procesables para que las empresas eviten instancias del mundo real. \\
\item \textbf{Rutas de aprendizaje}
Las rutas de aprendizaje de TryHackMe le brindarán tanto el conocimiento técnico fundamental como la experiencia práctica, lo cual es crucial para convertirse en un Red Teamer exitoso.
%
Probador de penetración JR
-Pentesting ofensivo
-Equipo rojo
\end{itemize}
\end{itemize}



\begin{table}[H]
\scalebox{0.8}{
\begin{tabular}{|l|l|l|}
\hline
# & Rol &  Rutas\\ \hline

1 & analista de seguridad & 
\begin{tabular}[c]{@{}l@{}} 
Introduction to Cyber Security\\
Pre Security\\
SOC Level 1\\
\end{tabular} \\ 
\hline

2 & \textbf{Ingeniero de seguridad} & 
\begin{tabular}[c]{@{}l@{}} 
Offensive Pentesting \\
JR Penetration Tester\\
SOC Level 1\\
\end{tabular} \\ 
\hline


3 & Respondedor de incidentes & 
\begin{tabular}[c]{@{}l@{}} 
SOC Level 1\\
\end{tabular} \\ 
\hline

7 & Probador de Penetración & 
\begin{tabular}[c]{@{}l@{}} 
Offensive Pentesting \\
JR Penetration Tester\\
\end{tabular} \\ 
\hline

8 & Red Teamer & 
\begin{tabular}[c]{@{}l@{}} 
Offensive Pentesting \\
JR Penetration Tester\\
Red Teamer\\
\end{tabular} \\ 
\hline

\end{tabular}
}
\end{table}

\section{NASA}
\section{ACM}

\begin{flushright}
    \textit{acm}
\end{flushright}
Principales áreas: 
\begin{itemize}
    \item KA-1 Seguridad de datos
        \begin{itemize}
            \item \textit{KU-1 Criptografía}
            \item KU-2 Forense digital
            \item \textit{KU-3 Integridad de datos y }atenticación
            \textit{\item KU-4 control de accceso}
            \item \textit{KU-5 protocolos de comunicación segura}
            \item KU-6 criptoanálisis
            \item \textit{KU-7 privacidad de datos}
            \item KU-8 seguridad en almacenamiento de información
        \end{itemize}
    \item KA-2 Seguridad de Software
    \begin{itemize}
            \item KU-9 Principios Fundamentales
            \item \underline{\textbf{KU-10  Diseño}}
            \item KU-11 implementación
            \item KU-12 análisis y pruebas
            \item \underline{\textbf{KU-13 despliegue y mantención}}
            \item KU-14 documentación
            \item KU-15 ética
            
            
        \end{itemize}
    \item KA-3 Seguridad de componentes
    \begin{itemize}
            \item \underline{\textbf{KU-16 diseño de componentes}}
            \item \underline{\textbf{KU-17 adquisición de componentes}}
            \item KU-18 pruebas de componentes
            \item KU-19 ingeniería reversa de componentes
        \end{itemize}
    \item \color{black} KA-4 Seguridad de conexiones
    \begin{itemize}
            \item KU-20 Medio físico
            \item KU-21 Conectores e interfaces con HW y componentes físicos
            \item \underline{\textbf{KU-22 arquitectura de HW}}
            \item \underline{\textbf{KU-23 Arquitectura de sistemas distribuidos}}
            \item \underline{\textit{\textbf{KU-24 Arquitectura de red}}}
            \item KU-25 Implementaciones de redes
            \item KU-26 Servicios de redes
            \item \textit{KU-27 Defensa de red}
        \end{itemize}\color{black} 

    \item \textbf{ KA-5 Seguridad del Sistema}
    \begin{itemize}
    
        \item \underline{\textbf{KU-28 Pensamiento Sistémico}}
        \item KU-29 Gestión del Sistema
        \item \textit{KU-30 Acceso a Sistema }
        \item \textit{KU-31 Control del Sistema}
        \item KU-32 Retiro de Sistema
        \item KU-33 Pruebas del Sistema
       \item \underline{\textbf{ KU-34 Arquitecturas Comunes de Sistema}}
\end{itemize}

\textbf{KA-6 Seguridad Humana}
\begin{itemize}
  \item KU-35 Gestión de Identidad
  \item KU-36 Ingeniería Social
  \item KU-37 Cumplimiento Personal de las Reglas/Políticas/Normas Éticas de Ciberseguridad
  \item KU-38 Conciencia y Comprensión
  \item KU-39 Privacidad social y de comportamiento 
  \item KU-40 Privacidad y seguridad  de Datos Personales
  \item \underline{\textbf{KU-41 Seguridad y Privacidad Utilizables}}
\end{itemize}

\textbf{KA-7 Seguridad Organizacional}
\begin{itemize}

  \item \underline{\textit{\textbf{ KU-42 Gestión de Riesgos}}}
  \item KU-43 Gobierno y Políticas de Seguridad
  \item KU-44 Herramientas Analíticas
  \item KU-45 Administración de Sistemas
  \item \underline{\textbf{KU-46 Planificación de Ciberseguridad}}
  \item KU-47 Continuidad del Negocio, Recuperación ante Desastres y Gestión de Incidentes
  \item \underline{\textbf{KU-48 Gestión de Programas de Seguridad}}
  \item KU-49 Seguridad del Personal
  \item KU-50 Seguridad de operaciones
\end{itemize}
\textbf{KA-8 Operación y Mantención}
\begin{itemize}
  \item KU-51 Soporte técnico y servicio al cliente
\end{itemize}

\textbf{KA-9 Seguridad Societal}
\begin{itemize}
 
  \item KU-52 Ciberdelincuencia
  \item KU-53 Legislación en Ciberseguridad
  \item KU-54 Ética en Ciberseguridad
  \item KU-55 Políticas de Ciberseguridad
  \item KU-56 Privacidad
\end{itemize}

    
\end{itemize}


\subsection{Consolidating cybersecurity in Europe: A case study on job profiles assessment}
Perfiles de seguridad acorde a paper ``Consolidating cybersecurity in Europe: A case study on job profiles assessment'':

\begin{itemize}

\item \textbf{General Cybersec Auditor}:	conduce auditoría interna/externa de controles de seguridad y sistemas de información con particular atención a los sistemas de entrada salida. Ejecuta auditorías de ciberseguridad y genera reportes técnicos de los hallazgos. Mantiene las políticas y estándares.	
\item \textbf{Technical Cybersec auditor}	Este perfil laboral es similar al JP1, pero con un enfoque especial en la implementación tecnológica. El trabajo incluye realizar un análisis exhaustivo de la adecuación y funcionamiento correcto de los sistemas de ciberseguridad, así como identificar áreas de mejora. Si se requieren mejoras, el auditor de seguridad también puede ser responsable de proporcionar un análisis de las medidas de seguridad recomendadas.	
\item \textbf{Threat Modeling Engineer}	Este perfil laboral se centra en establecer requisitos de seguridad, identificar riesgos de seguridad y posibles vulnerabilidades, calcular la criticidad de las amenazas y vulnerabilidades, y priorizar las opciones de solución. La creación de modelos de amenazas bien documentados proporciona garantías que se pueden utilizar para explicar y defender la postura de seguridad de un sistema de aplicación. A diferencia de los perfiles anteriores, es posible que un ingeniero de modelado opere directamente en la implementación del EES (Sistema de Seguridad Electrónica, por sus siglas en inglés)	
\item \textbf{Security Engineer}	El principal trabajo de un ingeniero de seguridad es crear y aplicar estrategias y estándares de seguridad. La mayoría de las tareas implican predecir las vulnerabilidades de la red o de los equipos informáticos y determinar cómo abordarlas. Se espera que este perfil laboral también opere directamente en la implementación.	
\item \textbf{Enterprise Cybersecurity Practicioner}	Una persona en esta posición debe ser capaz de dominar la gestión de riesgos, específicamente desde una perspectiva de ciberseguridad. El perfil laboral debe comprender al menos superficialmente la arquitectura de red y las vulnerabilidades de seguridad de la empresa, incluyendo las instalaciones de almacenamiento y computación. Pueden evaluar los riesgos y elegir medidas para mitigarlos, por ejemplo, asesorando sobre las mejores soluciones para la empresa en dispositivos móviles, almacenamiento y computación en la nube, técnicas criptográficas, tamaño y composición de equipos de respuesta, etc.	
\item \textbf{Cybersecurity Analyst}	Este perfil tiene experiencia en administración de redes (incluyendo seguridad), por ejemplo, en análisis de arquitectura y vulnerabilidades, así como en identificación y mitigación de amenazas. Un analista de ciberseguridad debe tener al menos un nivel moderado de habilidad en la respuesta a incidentes cibernéticos, como realizar un análisis de penetración utilizando herramientas profesionales.	

\end{itemize}

Las principalidades áreas en que deben manejarse los profesionales de mayor a menor prioridad son (academia vs industria) :	\begin{itemize}
    \item Academia 
        \begin{itemize}
            \item Data Security (DS)
            \item System Security (SS)
            \item Connection Security (CS)
            \item Organizational Security (OS)
            \item Software Security (SwS)
            \item Human Security
            \item Societal Security (SocS)
            \item Component Security (CompS)
            \item Operations and maintenance  (OM)
        \end{itemize}
    \item Industria
     \begin{itemize}
            \item Societal Security (SocS)
            \item Software Security (SwS)
            \item Data Security (DS)
            \item Organizational Security (OS)
             \item Human Security
            
            \item Connection Security (CS)
            \item System Security (SS)
            \item Component Security (CompS)
            \item Operations and maintenance (OM)
        \end{itemize}
\end{itemize}	
Las principalidades habilidades necesarias son:	\begin{itemize}
    \item Defensa de red (CS)
    \item Arquitectura de red (CS)
    \item Control y acceso a sistemas (SS)
    \item Principios Fundamentales(SWS)
    \item Protocolos de comunicación segura (DS) \item  Incidentes y Continuidad (OS)
    \item Manejo de Riesgos 
\end{itemize}					


\subsection{Detalles Areas ACM}
\begin{tcolorbox}
[colback=gray!5!white,colframe=blue!10!gray,title= Conceptos transversales] Los temas omnipresentes de \textbf{confidencialidad, integridad, disponibilidad, riesgo, pensamiento adversarial y pensamiento sistémico} que ayudan a los estudiantes a explorar las conexiones entre las \textbf{ocho áreas de conocimiento} y refuerzan la importancia de una mentalidad de seguridad en todas las áreas de conocimiento.

\end{tcolorbox}
Competencias esenciales:
\begin{itemize}
    \item CT-1: Describir mediante métodos apropiados, y utilizando terminología estándar de la industria, los \textbf{problemas relacionados con la ciberseguridad dentro de una organización en lo que respecta a la confidencialidad, integridad y disponibilidad}. \textit{Analizar}
\item CT-2: \textbf{Evaluar y responder adecuadamente a diversos riesgos} que pueden afectar el funcionamiento esperado de los sistemas de información. \textit{Evaluar}
\item CT-3: \textbf{Investigar las amenazas cibernéticas actuales y emergentes} e incorporar las mejores prácticas para mitigarlas. \textit{Aplicar}
\item CT-4: \textbf{Aplicar contramedidas adecuadas} para ayudar a proteger los recursos organizativos basándose en la comprensión de cómo piensan y operan los actores malintencionados. \textit{Aplicar}
\item CT-5: Discutir cómo los \textbf{cambios en una parte de un sistema pueden afectar a otras partes de un ecosistema de ciberseguridad}. \textit{Comprender}
\end{itemize}





\begin{tcolorbox}
[colback=gray!5!white,colframe=blue!10!gray,title= Conceptos transversales] Los temas omnipresentes de \textbf{confidencialidad, integridad, disponibilidad, riesgo, pensamiento adversarial y pensamiento sistémico} que ayudan a los estudiantes a explorar las conexiones entre las \textbf{ocho áreas de conocimiento} y refuerzan la importancia de una mentalidad de seguridad en todas las áreas de conocimiento.

\end{tcolorbox}
Competencias esenciales:
\begin{itemize}
    \item CT-1: Describir mediante métodos apropiados, y utilizando terminología estándar de la industria, los \textbf{problemas relacionados con la ciberseguridad dentro de una organización en lo que respecta a la confidencialidad, integridad y disponibilidad}. \textit{Analizar}
\item CT-2: \textbf{Evaluar y responder adecuadamente a diversos riesgos} que pueden afectar el funcionamiento esperado de los sistemas de información. \textit{Evaluar}
\item CT-3: \textbf{Investigar las amenazas cibernéticas actuales y emergentes} e incorporar las mejores prácticas para mitigarlas. \textit{Aplicar}
\item CT-4: \textbf{Aplicar contramedidas adecuadas} para ayudar a proteger los recursos organizativos basándose en la comprensión de cómo piensan y operan los actores malintencionados. \textit{Aplicar}
\item CT-5: Discutir cómo los \textbf{cambios en una parte de un sistema pueden afectar a otras partes de un ecosistema de ciberseguridad}. \textit{Comprender}
\end{itemize}



\newpage
\begin{tcolorbox}
[colback=gray!5!white,colframe=blue!10!gray,title=Seguridad de Datos] 
Se enfoca en la protección de los datos en reposo, durante el procesamiento y en tránsito. Esta área de conocimiento requiere la aplicación de algoritmos matemáticos y analíticos para su implementación completa.\end{tcolorbox}
Competencias esenciales
\begin{itemize}
    \item  DAT-E1: \textbf{Implementar la seguridad de datos mediante la selección de procedimientos criptográficos}, algoritmos y herramientas adecuadas en función de la política de seguridad y el nivel de riesgo en una organización. \textit{Aplicar}
    \item  DAT-E2: Discutir la \textbf{recolección y adquisición forensemente sólida de pruebas digitales}. \textit{Comprender}
    \item  DAT-E3: Aplicar principios, procesos, herramientas y técnicas utilizadas para \textbf{mitigar amenazas de seguridad y responder a incidentes de seguridad}. \textit{Aplicar}
    \item  DAT-E4: Utilizar \textbf{niveles apropiados de autenticación, autorización y control de acceso} para garantizar la integridad y seguridad de los datos en los sistemas de información y redes. \textit{Aplicar}
    \item  DAT-E5: \textbf{Inferir brechas en la seguridad de datos }considerando las tecnologías actuales y emergentes, así como el estado actual y las tendencias predominantes en la ciberdelincuencia. \textit{Comprender}
\end{itemize}
Competencias complementarias
\begin{itemize}
\item DAT-S1: \textbf{Realizar un análisis forense en una red local, en datos almacenados dentro de un sistema y en dispositivos móviles en un entorno empresarial}. \textit{Aplicar}
\item  DAT-S2: Describir conceptos técnicos complejos a audiencias técnicas y no técnicas en relación a la seguridad de datos. \textit{Analizar}
\end{itemize}
\textbf{Unidades de Conocimiento}
\begin{itemize}
\item Criptografía
\item Forense Digital
\item Integridad y Autenticación de Datos
\item Control de Acceso
\item Protocolos de Comunicación Segura
\item Criptoanálisis
\item Privacidad de Datos
\item Seguridad en el Almacenamiento de Información
\end{itemize}
\newpage 


\noindent \textbf{Resultados de Aprendizaje Escenciales en criptografía}
\begin{itemize}
\item[\textbf{DAT-LO-E01}] Analizar qué protocolos criptográficos, herramientas y técnicas son apropiados para proporcionar confidencialidad, protección de datos, integridad de datos, autenticación, no repudio y ofuscación. (Analizar)
\item[\textbf{DAT-LO-E02}] Aplicar algoritmos simétricos y asimétricos según corresponda para un escenario dado. (Aplicar)
\item[\textbf{DAT-LO-E03}] Investigar funciones hash para verificar la integridad y proteger los datos de autenticación. (Aplicar)
\item[\textbf{DAT-LO-E04}] Utilizar cifrados históricos, como el cifrado por desplazamiento, cifrado afín, cifrado por sustitución, cifrado de Vigenère, ROT-13, cifrado de Hill y simulador de la máquina Enigma, para encriptar y desencriptar datos. (Aplicar)
\end{itemize}


\noindent \textbf{Resultados de Aprendizaje Suplementarios}
\begin{itemize}
\item[\textbf{DAT-LO-S01}] Comparar los beneficios y desventajas de aplicar la criptografía en hardware versus software. (Analizar)
\item[\textbf{DAT-LO-S02}] Demostrar la importancia de la teoría matemática en la aplicación de la criptografía. (Comprender)
\item[\textbf{DAT-LO-S03}] Deducir la longitud mínima de clave necesaria para que los algoritmos simétricos sean efectivos. (Analizar)
\item[\textbf{DAT-LO-S04}] Contrastar los modelos de confianza en la infraestructura de clave pública (PKI), como jerárquico, distribuido, puente y red de confianza. (Analizar)
\item[\textbf{DAT-LO-S05}] Explicar cómo se utilizan el cifrado simétrico y asimétrico en conjunto para asegurar las comunicaciones electrónicas y transacciones, como las criptomonedas y otros activos criptográficos. (Comprender)
\item[\textbf{DAT-LO-S06}] Aplicar la criptografía simétrica y asimétrica, así como también
\end{itemize}

\newpage En particular en \textbf{Integridad y Autenticación de Datos}, los \textbf{Resultados de Aprendizaje Esenciales} son: 
\begin{itemize}
  \item[\textbf{DAT-LO-E10}] Contrastar los conceptos y técnicas para lograr la integridad de datos, autenticación, autorización y control de acceso. (Analizar)
  \item[\textbf{DAT-LO-E11}] Resumir los beneficios y desafíos de la \textbf{autenticación multifactor}. (Comprender)
  \item[\textbf{DAT-LO-E12}] \textbf{Ejecutar una o más técnicas de ataque de contraseñas, como ataques de diccionario}, ataques de fuerza bruta, ataques de tablas arcoíris, phishing y ingeniería social, ataques basados en malware, rastreo, análisis sin conexión y herramientas de descifrado de contraseñas. (Aplicar)
  \item[\textbf{DAT-LO-E13}] Aplicar funciones básicas asociadas con el almacenamiento de datos sensibles, como \textbf{funciones hash criptográficas, salting, }conteo de iteraciones, derivación de claves basada en contraseñas y gestores de contraseñas. (Aplicar)
\end{itemize}

\textbf{Resultados de Aprendizaje Suplementarios}
\begin{itemize}
  \item[\textbf{DAT-LO-S18}] Implementar la autenticación multifactor utilizando herramientas y técnicas como \textbf{tokens criptográficos, dispositivos criptográficos, autenticación biométrica, contraseñas de un solo uso y autenticación basada en conocimiento.} (Aplicar)
  \item[\textbf{DAT-LO-S19}] Ilustrar el uso de la criptografía para proporcionar integridad de datos, como códigos de autenticación de mensajes,\textbf{ firmas digitales, cifrado autenticado y árboles de hash.} (Aplicar)
\end{itemize}

\newpage

\textbf{Resultados de Aprendizaje Esenciales} para \textbf{Control de Acceso}
\begin{itemize}
  \item[\textbf{DAT-LO-E14}] Describir las \textbf{mejores prácticas de control de acceso,} como la separación de tareas, rotación de trabajos y política de escritorio limpio. (Comprender)
  \item[\textbf{DAT-LO-E15}] Discutir los \textbf{controles de seguridad física}, como el acceso con llave, trampas para personas, tarjetas clave y videovigilancia, seguridad a nivel de bastidor y destrucción de datos. (Comprender)
  \item[\textbf{DAT-LO-E16}] Implementar el control de acceso a los datos para gestionar identidades, credenciales, privilegios y acceso relacionado. (Aplicar)
  \item[\textbf{DAT-LO-E17}] Diferenciar entre los diferentes tipos de identidades, como identidades federadas. (Comprender)
  \item[\textbf{DAT-LO-E18}] Diferenciar los modelos de control de acceso, incluyendo basados en roles, basados en reglas y basados en atributos. (Comprender)
\end{itemize}

\textbf{Resultados de Aprendizaje Suplementarios}

\begin{itemize}
  \item[\textbf{DAT-LO-S20}] Investigar modelos de control de acceso, como basados en roles, basados en reglas y basados en atributos. (Aplicar)
  \item[\textbf{DAT-LO-S21}] Ilustrar el valor fundamental y los beneficios de las arquitecturas de seguridad utilizadas para proteger la información en sistemas informáticos. (Aplicar)
\end{itemize}




\newpage
\begin{tcolorbox}
[colback=gray!5!white,colframe=blue!10!gray,title=Seguridad de Software] 
Se centra en el desarrollo de software teniendo en cuenta la seguridad y las posibles vulnerabilidades, de manera que no pueda ser fácilmente explotado.
La seguridad de un sistema, así como de los datos que almacena y gestiona, depende en gran medida de la seguridad de su software. La seguridad del software depende de cómo se ajusten los requisitos a las necesidades que el software debe abordar, de cómo se diseña, implementa, prueba, despliega y mantiene el software. La documentación es fundamental para que todos comprendan estas consideraciones, y surgen consideraciones éticas a lo largo de la creación, el despliegue, el uso y el retiro del software.\end{tcolorbox}

\section{The Core Cyber-Defense Knowledge, Skills, and Abilities
That Cybersecurity Students Should Learn in School: Results
from Interviews with Cybersecurity Professionals}
Perfiles de seguridad acorde a paper ``Consolidating cybersecurity in Europe: A case study on job profiles assessment'':