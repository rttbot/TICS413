

\textbf{Pregunta}\\
Los ataques DDoS pueden ser muy perjudiciales para una organización. Como analista de seguridad, es importante reconocer la gravedad de un ataque de este tipo para conocer las oportunidades de proteger la red de ellos. ¿Qué acciones concretas  pueden tomar para ayudar para que si se envían nuevos ataques, la empresa esté preparada y mitigue el impacto?


\section{Legal y ética}

\subsection{Delitos Cibernéticos y Delitos Informáticos}

En esta sección, abordaremos los delitos cibernéticos e informáticos, sus impactos y las medidas para prevenirlos. Los delitos cibernéticos comprenden actividades ilegales realizadas a través de internet, como el fraude, el robo de identidad y los ataques de malware. Por ejemplo, el robo de identidad ocurre cuando un atacante accede sin autorización a la información personal de un usuario, lo que puede resultar en pérdidas financieras significativas. En Chile, la Ley N° 19.223 sanciona estos delitos, estableciendo penas para quienes accedan indebidamente a sistemas informáticos o alteren datos.

Por otro lado, los delitos informáticos se refieren a crímenes que involucran el uso de computadoras y redes, como el hacking y la distribución de virus. Un ejemplo de hacking es el acceso no autorizado a una base de datos gubernamental para extraer información confidencial. La legislación chilena también aborda estos temas, y se han propuesto reformas para actualizar las leyes en consonancia con los avances tecnológicos, como el proyecto de ley que busca modernizar la Ley N° 19.223 para incluir nuevas tipologías de delitos informáticos.

El cibercrimen es un término que describe actividades delictivas donde las computadoras o redes son una herramienta, un objetivo o un lugar de actividad delictiva. Estas categorías no son exclusivas, y muchas actividades pueden pertenecer a una o más categorías. El Departamento de Justicia de EE. UU. categoriza el delito informático según el papel de la computadora en la actividad delictiva:

- \textbf{Computadoras como objetivos}: Delitos que buscan adquirir información, controlar el sistema sin autorización o alterar la integridad de los datos. En Chile, la Ley N° 19.223 sanciona la alteración o destrucción de datos informáticos.
- \textbf{Computadoras como dispositivos de almacenamiento}: Uso de computadoras para facilitar actividades ilegales, como almacenar material ilegal.
- \textbf{Computadoras como herramientas de comunicación}: Delitos tradicionales cometidos en línea, como el fraude electrónico.

\textbf{Desafíos de la Aplicación de la Ley}

El efecto disuasorio de la aplicación de la ley en ataques informáticos se correlaciona con la tasa de éxito de arrestos y enjuiciamientos. La naturaleza del cibercrimen hace que el éxito constante sea difícil. En Chile, la colaboración internacional es clave para enfrentar estos desafíos, ya que muchos cibercrímenes trascienden fronteras. Recientemente, se ha discutido la adhesión de Chile a la Convención de Budapest, el principal tratado internacional que aborda los delitos informáticos mediante la cooperación internacional.

\textbf{Convención sobre Cibercrimen}

La Convención de Budapest incluye artículos sobre acceso ilegal, intercepción ilegal, interferencia de datos, uso indebido de dispositivos, fraude informático, delitos relacionados con la pornografía infantil, infracciones de derechos de autor, entre otros. Chile ha mostrado interés en adherirse a esta convención para fortalecer su marco legal en ciberseguridad.

\textbf{Resultados de la Encuesta CERT 2007 E-Crime Watch}

Esta encuesta incluye estadísticas sobre delitos como virus, acceso no autorizado, spam ilegal, spyware, ataques de denegación de servicio, fraude, phishing y robo de propiedad intelectual. Estos datos resaltan la importancia de la cooperación internacional y la actualización constante de las leyes para enfrentar las nuevas amenazas cibernéticas.


\subsection{Privacidad}
La privacidad es un derecho fundamental que enfrenta amenazas significativas en la era digital. La creciente recopilación y almacenamiento de datos personales por parte de empresas y gobiernos plantea riesgos para la privacidad individual. En esta sección, exploraremos los desafíos para mantener la privacidad en línea, las leyes y regulaciones que protegen la privacidad de los usuarios, y las mejores prácticas para asegurar que la información personal esté protegida. También analizaremos casos de violaciones de privacidad y sus consecuencias.

\subsubsection{Desafíos de la Privacidad en Línea}
- \textbf{Recopilación Masiva de Datos}: Las empresas y gobiernos recopilan grandes cantidades de datos personales, lo que aumenta el riesgo de violaciones de privacidad.
- \textbf{Tecnologías de Vigilancia}: El uso de tecnologías avanzadas para monitorear actividades en línea puede comprometer la privacidad de los usuarios.

\subsubsection{Leyes y Regulaciones}
- \textbf{Reglamento General de Protección de Datos (GDPR)}: Establece estándares estrictos para la protección de datos personales en la UE.
- \textbf{Ley Chilena de Protección de Datos}: La Ley N° 19.628 regula el tratamiento de datos personales en Chile, requiriendo consentimiento y medidas de seguridad adecuadas.

\subsubsection{Mejores Prácticas}
- \textbf{Cifrado de Datos}: Utilizar cifrado para proteger la información personal contra accesos no autorizados.
- \textbf{Anonimización}: Implementar técnicas de anonimización para minimizar el riesgo de identificación de individuos.
- \textbf{Educación y Concienciación}: Capacitar a los usuarios sobre la importancia de la privacidad y cómo proteger sus datos.

\subsubsection{Casos de Violaciones de Privacidad}
- \textbf{Impacto de las Violaciones}: Las violaciones de privacidad pueden resultar en daños a la reputación y sanciones financieras para las organizaciones.
- \textbf{Ejemplos Notables}: Analizar casos recientes de violaciones de datos para entender sus causas y consecuencias.

\subsection{Cuestiones Éticas}
Debido a la ubicuidad e importancia de los sistemas de información en organizaciones de todo tipo, hay muchos posibles usos indebidos y abusos de la información y la comunicación electrónica que crean problemas de privacidad y seguridad. Además de cuestiones de legalidad, el uso indebido y el abuso plantean preocupaciones éticas. La ética se refiere a un sistema de principios morales que se relaciona con los beneficios y daños de acciones particulares y con la corrección o incorrección de los motivos y fines de esas acciones. En esta sección, analizamos cuestiones éticas relacionadas con la seguridad de los sistemas informáticos y de información.

\subsubsection{Ética y las Profesiones de Tecnología de la Información}
Hasta cierto punto, una caracterización de lo que constituye un comportamiento ético para aquellos que trabajan con o tienen acceso a sistemas de información no es única para este contexto. Los principios éticos básicos desarrollados por las civilizaciones se aplican. Sin embargo, hay algunas consideraciones únicas en torno a las computadoras y los sistemas de información. Primero, la tecnología informática hace posible una escala de actividades que no era posible antes. Esto incluye un mayor registro de datos, particularmente sobre individuos, con la capacidad de desarrollar una colección de información personal más detallada y datos más precisos. La escala ampliada de comunicaciones y la interconexión traída por Internet magnifica el poder de un individuo para hacer daño. Segundo, la tecnología informática ha involucrado la creación de nuevos tipos de entidades para las cuales no se han formado previamente reglas éticas acordadas, como bases de datos, navegadores web, salas de chat, cookies, etc.

Además, siempre ha sido el caso que aquellos con conocimientos o habilidades especiales tienen obligaciones éticas adicionales más allá de las comunes a toda la humanidad. Podemos ilustrar esto en términos de una jerarquía ética basada en una que se discutió en [GOTT99]. En la parte superior de la jerarquía están los valores éticos que los profesionales comparten con todos los seres humanos, como la integridad, la equidad y la justicia. Ser un profesional con formación especial impone obligaciones éticas adicionales con respecto a aquellos afectados por su trabajo. Los principios generales aplicables a todos los profesionales surgen en este nivel. Finalmente, cada profesión tiene asociadas con ella reglas éticas específicas y obligaciones relacionadas con el conocimiento específico de aquellos en la profesión y los poderes que tienen para afectar a otros. La mayoría de las profesiones adoptan todos estos niveles en un código de conducta profesional, un tema que se discutirá posteriormente.

\subsubsection{Propuesta de Ideas con Ejemplos de Códigos de Conducta}
\begin{itemize}
    \item \textbf{Integridad y Honestidad}: Los profesionales deben actuar con integridad y honestidad en todas sus interacciones. Por ejemplo, el Código de Ética de la ACM (Association for Computing Machinery) enfatiza la importancia de ser honesto y confiable.
    \item \textbf{Confidencialidad}: Proteger la información confidencial es crucial. El Código de Conducta de la IEEE (Institute of Electrical and Electronics Engineers) establece que los ingenieros deben respetar la privacidad de los demás y proteger la información confidencial.
    \item \textbf{Responsabilidad Profesional}: Los profesionales deben asumir la responsabilidad de su trabajo y sus consecuencias. La AITP (Association of Information Technology Professionals) destaca la importancia de aceptar la responsabilidad total por el trabajo realizado.
    \item \textbf{Evitar Conflictos de Interés}: Los profesionales deben evitar situaciones donde sus intereses personales puedan influir en su juicio profesional. El Código de Conducta de la IEEE requiere que los ingenieros eviten conflictos de interés siempre que sea posible.
    \item \textbf{Uso Ético de la Tecnología}: Los profesionales deben usar la tecnología de manera que beneficie a la sociedad y no cause daño. El Código de Ética de la ACM incluye directrices sobre el uso responsable de la tecnología para mejorar el bienestar social.
\end{itemize}


\section{El modelo OSI: Fundamentos de comunicación en red}

El modelo OSI (Open Systems Interconnection) es un marco conceptual fundamental que divide la comunicación de red en siete capas, cada una con responsabilidades específicas. Este modelo proporciona una base para entender cómo funcionan las redes y cómo implementar seguridad en cada nivel.

\subsection{¿Por qué siete capas?}

La división en capas resuelve el problema de la complejidad en las comunicaciones de red. Cada capa se enfoca en una función específica, permitiendo que los desarrolladores trabajen en un nivel sin preocuparse por los detalles de las otras capas. Es como construir un edificio: los cimientos, las paredes, la electricidad y la decoración son responsabilidades separadas que trabajan juntas para crear el resultado final.

\subsection{Capa 1: Física - Los cimientos de la comunicación}

La capa física es la base de toda comunicación digital. Su función principal es mover bits (0 y 1) de un punto A a un punto B usando señales eléctricas, de luz o de radio. Esta capa no entiende "mensajes", "direcciones IP" ni "usuarios"; solo se preocupa de que los bits lleguen físicamente por el medio de transmisión.

\textbf{Analogía del mundo real}: Si internet fuera una ciudad y los datos fueran autos, la capa física sería la carretera por donde los autos se mueven. No decide a qué barrio van; solo asegura que haya pavimento y que los autos puedan avanzar.

\textbf{Protocolos principales}:
\begin{itemize}
    \item \textbf{Ethernet (IEEE 802.3)}: Para enviar bits por cable entre equipos de forma confiable
    \item \textbf{WiFi (IEEE 802.11)}: Define cómo viajan los bits de forma inalámbrica
    \item \textbf{RS-232/RS-485}: Para comunicaciones punto a punto simples entre equipos
    \item \textbf{Fibra óptica}: Transmisión por luz
    \item \textbf{DOCSIS}: Para cable coaxial
\end{itemize}

\subsection{Capa 2: Enlace de Datos - El portero del edificio}

La capa de enlace de datos se encarga de establecer una conexión directa entre dos dispositivos en la misma red. Su función principal es organizar los bits en tramas, detectar y corregir errores de transmisión, y controlar el acceso al medio físico.

\textbf{Analogía del mundo real}: La capa de enlace es como el portero de un edificio. Imagina que los datos son personas que quieren entrar al edificio (la red). El portero verifica quién entra, si tienen permiso, y se asegura de que no haya conflictos entre quienes entran al mismo tiempo. También revisa que nadie entre con errores y que todo esté en orden.

\textbf{Protocolos principales}:
\begin{itemize}
    \item \textbf{Ethernet}: Idioma utilizado para que los dispositivos conectados por cable a una red local puedan hablar entre ellos
    \item \textbf{MAC (Media Access Control)}: Cada dispositivo posee una dirección MAC única que los distingue entre sí
    \item \textbf{LLC (Logical Link Control)}: Gestiona los errores y el control de flujo de los datos
    \item \textbf{WiFi (802.11)}: Para redes inalámbricas
    \item \textbf{PPP}: Protocolo punto a punto
\end{itemize}

\subsection{Capa 3: Red - El sistema de direcciones y mapas}

La capa de red tiene como función principal permitir que los datos viajen desde un dispositivo de origen hasta uno de destino, incluso si están en redes distintas o ubicaciones geográficas lejanas. Esta capa hace posible la comunicación entre dispositivos conectados a diferentes redes.

\textbf{Analogía del mundo real}: Es como enviar una carta que debe atravesar múltiples oficinas postales antes de llegar a su destino. La capa de red utiliza direcciones IP que identifican de forma única cada dispositivo, y se encarga del encaminamiento para determinar la mejor ruta disponible.

\textbf{Protocolos principales}:
\begin{itemize}
    \item \textbf{IP (Internet Protocol)}: Asigna direcciones lógicas y transporta los paquetes
    \item \textbf{ICMP}: Gestiona mensajes de control y error (ping, traceroute)
    \item \textbf{IGMP}: Permite la gestión de grupos multicast
    \item \textbf{OSPF, BGP}: Protocolos de enrutamiento
    \item \textbf{IPsec}: Protege la integridad y confidencialidad mediante cifrado
\end{itemize}

\subsection{Capa 4: Transporte - El sistema de entrega de paquetes}

La capa de transporte permite la comunicación confiable y organizada entre aplicaciones en diferentes sistemas. Gestiona la segmentación de datos, dividiendo grandes flujos en unidades más pequeñas llamadas segmentos para su transmisión eficiente.

\textbf{Analogía del mundo real}: Es como pedir una mesa por internet. En vez de llegar armada en una caja grande difícil de transportar, las diferentes partes de la mesa llegan en cajas más pequeñas, para luego armarlas en casa. La capa de transporte se asegura de que todas las piezas lleguen completas y en el orden correcto.

\textbf{Protocolos principales}:
\begin{itemize}
    \item \textbf{TCP}: Protocolo orientado a conexión que garantiza entrega completa y ordenada
    \item \textbf{UDP}: Protocolo sin conexión, rápido pero sin garantías de entrega
    \item \textbf{TLS/SSL}: Añade seguridad al transporte de datos mediante cifrado
    \item \textbf{SCTP}: Protocolo de control de transmisión de stream
\end{itemize}

\subsection{Capa 5: Sesión - El coordinador de reuniones}

La capa de sesión permite a los usuarios en distintas máquinas establecer sesiones entre ellos. Ofrece servicios de control del diálogo, manejo de tokens y sincronización.

\textbf{Analogía del mundo real}: En una partida de ajedrez entre dos personas, el control del diálogo significa que juegan por turnos; el manejo de tokens asegura que solo una persona puede tener la pieza en la mano a la vez; y la sincronización permite que si se corta la luz, al volver continúen desde la última jugada, no desde el inicio.

\textbf{Protocolos principales}:
\begin{itemize}
    \item \textbf{RPC}: Permite que un programa ejecute funciones en otro equipo de forma remota
    \item \textbf{NetBIOS}: Protocolo usado en redes Windows para establecer sesiones
    \item \textbf{SQL*Net}: Comunicación con bases de datos Oracle
    \item \textbf{PPTP/L2TP}: Protocolos de túnel VPN
\end{itemize}

\subsection{Capa 6: Presentación - El traductor}

La función principal de la capa de presentación es traducir, transformar y preparar los datos para que puedan ser correctamente interpretados por el receptor. Esto incluye tareas como codificación, compresión y cifrado.

\textbf{Analogía del mundo real}: Es una combinación entre un traductor y un repartidor de cartas. La capa se encarga de que el mensaje llegue de forma entendible al destinatario (traductor) y lo encripta para que durante el viaje no pueda ser interpretado y leído por terceras partes.

\textbf{Protocolos principales}:
\begin{itemize}
    \item \textbf{SSL/TLS}: Proporciona cifrado y seguridad en la comunicación
    \item \textbf{MIME}: Permite que los correos incluyan archivos adjuntos, imágenes, audio
    \item \textbf{ASCII/UTF-8}: Traducen los datos a formatos que las aplicaciones puedan entender
    \item \textbf{JPEG, MPEG}: Formatos de imagen y video
    \item \textbf{XML}: Lenguaje de marcado extensible
\end{itemize}

\subsection{Capa 7: Aplicación - La recepción del edificio}

La capa de aplicación proporciona la interfaz entre los usuarios y los servicios de red. Su rol es facilitar la comunicación entre software y red, ofreciendo servicios como navegación web, correo electrónico o transferencia de archivos.

\textbf{Analogía del mundo real}: Si la red fuera un gran edificio, la capa de aplicación sería la recepción, el lugar donde los visitantes (usuarios) solicitan servicios (correo, web, archivos) sin preocuparse por cómo funcionan los pasillos, ascensores o electricidad del edificio.

\textbf{Protocolos principales}:
\begin{itemize}
    \item \textbf{HTTP/HTTPS}: Permiten la navegación web (HTTPS es la versión segura)
    \item \textbf{SMTP, IMAP, POP3}: Protocolos de correo electrónico
    \item \textbf{FTP/SFTP}: Permiten la transferencia de archivos (SFTP es la versión segura)
    \item \textbf{DNS}: Resolución de nombres de dominio
    \item \textbf{SSH}: Acceso remoto seguro
\end{itemize}

\subsection{Interacción entre capas}

Las capas trabajan en conjunto de manera jerárquica. Cada capa se comunica solo con las capas adyacentes: recibe servicios de la capa inferior y proporciona servicios a la capa superior. Esta separación de responsabilidades permite que cada capa se desarrolle independientemente, facilitando la evolución de las tecnologías de red.

Por ejemplo, cuando envías un correo electrónico, la aplicación (capa 7) prepara el mensaje, la capa de presentación (6) lo codifica, la capa de sesión (5) establece la conexión, la capa de transporte (4) lo divide en segmentos, la capa de red (3) lo enruta, la capa de enlace (2) lo empaqueta en tramas, y la capa física (1) lo transmite como señales eléctricas o de radio.

\section{Software Malicioso (Malware)}

\subsection{Definición y clasificación del malware}

