\chapterimage{Pictures/networks.jpg}
\chapter{Seguridad de Redes y Protocolos}
\vspace{65px}
\begin{flushright}
    \textit{La seguridad de redes requiere un enfoque de defensa en profundidad que proteja cada capa del modelo OSI.}
\end{flushright}

\section{Introducción a la seguridad de redes}

La seguridad de redes es un componente fundamental de la ciberseguridad que se enfoca en proteger la infraestructura de red, los datos que viajan a través de ella y los sistemas conectados. En un mundo cada vez más interconectado, donde las redes son la columna vertebral de la comunicación digital, la seguridad de redes se ha convertido en una prioridad crítica para organizaciones de todos los tamaños.

\subsection{¿Por qué es importante la seguridad de redes?}

La seguridad de redes es fundamental porque las redes modernas transportan información extremadamente sensible que debe ser protegida contra accesos no autorizados. Desde datos financieros hasta información médica confidencial, las redes corporativas contienen activos críticos que requieren protección robusta. Además, las redes representan el primer punto de entrada para la mayoría de los atacantes, quienes buscan explotar vulnerabilidades en la infraestructura de red para acceder a sistemas internos.

El cumplimiento normativo también juega un papel crucial, ya que muchas industrias como la financiera, médica y gubernamental requieren medidas específicas de seguridad de red para cumplir con regulaciones como PCI DSS, HIPAA o SOX. Una red comprometida puede paralizar completamente las operaciones de una organización, resultando en pérdidas financieras significativas y daños a la reputación que pueden tardar años en recuperarse. Los incidentes de seguridad de red, especialmente cuando resultan en filtración de datos de clientes, pueden erosionar la confianza de los clientes y afectar la viabilidad a largo plazo del negocio.

\subsection{Principios fundamentales de seguridad de redes}

La defensa en profundidad es un principio fundamental que establece que la seguridad debe implementarse en múltiples capas, creando redundancia que hace más difícil que un atacante comprometa completamente el sistema. Este enfoque reconoce que ningún control de seguridad es perfecto, por lo que múltiples capas de protección proporcionan una red de seguridad más robusta.

El principio de menor privilegio dicta que los usuarios y sistemas deben tener acceso mínimo necesario para realizar sus funciones específicas. Este principio reduce significativamente la superficie de ataque y limita el daño potencial si una cuenta es comprometida. La seguridad por diseño integra consideraciones de seguridad desde el inicio del desarrollo de sistemas y redes, en lugar de intentar agregarlas posteriormente como una reflexión tardía.

El monitoreo continuo es esencial para detectar amenazas en tiempo real y responder rápidamente a incidentes de seguridad. Esto incluye la vigilancia constante de tráfico de red, logs de sistema y comportamiento anómalo. Finalmente, la capacidad de respuesta rápida permite a las organizaciones contener y mitigar incidentes antes de que causen daños significativos, minimizando el impacto en las operaciones del negocio.

\section{Seguridad en redes inalámbricas (WiFi)}

Las redes inalámbricas presentan desafíos únicos de seguridad debido a su naturaleza de transmisión por radio. A diferencia de las redes cableadas, donde el acceso físico está limitado por la infraestructura de cables, las señales WiFi pueden ser interceptadas por cualquier dispositivo dentro del rango de transmisión. Esta característica inherente las hace más vulnerables a ataques y requiere medidas de seguridad específicas y robustas.

\subsection{Características únicas de las redes inalámbricas}

Las redes WiFi presentan características que las hacen fundamentalmente diferentes y más vulnerables que las redes cableadas. El canal de transmisión utiliza broadcast, lo que significa que las comunicaciones inalámbricas son más susceptibles a interceptación y jamming que las redes cableadas. Cualquier dispositivo dentro del rango de la señal puede potencialmente capturar el tráfico, a diferencia de las redes cableadas donde se requiere acceso físico directo.

La movilidad de los dispositivos inalámbricos también representa un desafío significativo. Los dispositivos WiFi son más portátiles, lo que aumenta los riesgos de pérdida física y acceso no autorizado. Un laptop perdido o robado puede contener credenciales de red almacenadas, representando una amenaza de seguridad considerable.

Además, muchos dispositivos inalámbricos tienen recursos limitados de memoria y procesamiento, lo que dificulta la implementación de medidas de seguridad robustas. Esta limitación es especialmente problemática en dispositivos IoT, que suelen tener capacidades computacionales mínimas pero están cada vez más presentes en entornos empresariales.

\subsection{Componentes del entorno inalámbrico}

El entorno inalámbrico consta de tres componentes principales que representan puntos de ataque críticos. El cliente inalámbrico incluye dispositivos como teléfonos móviles, laptops con WiFi, sensores inalámbricos y dispositivos Bluetooth. Estos dispositivos son el punto final de la comunicación y pueden ser comprometidos para obtener acceso a la red.

El punto de acceso inalámbrico proporciona la conexión a la red o servicio, incluyendo torres celulares, hotspots WiFi y puntos de acceso a redes cableadas. Estos dispositivos actúan como puertas de entrada a la infraestructura de red y son objetivos frecuentes de ataques.

El medio de transmisión, que transporta las ondas de radio para la transferencia de datos, también representa una fuente de vulnerabilidad. Las señales pueden ser interceptadas, interferidas o bloqueadas sin necesidad de acceso físico directo al equipo.

\subsection{Amenazas específicas a redes inalámbricas}

\subsubsection{Asociación accidental}

Las redes LAN corporativas o puntos de acceso inalámbricos a LANs cableadas en proximidad cercana pueden crear rangos de transmisión superpuestos. En este escenario, un usuario que intenta conectarse a una LAN puede bloquearse accidentalmente en un punto de acceso de una red vecina. Esto expone recursos de una LAN al usuario accidental, creando una brecha de seguridad significativa.

Por ejemplo, en un edificio de oficinas donde múltiples empresas comparten el mismo espacio, un empleado de la empresa A podría conectarse accidentalmente a la red de la empresa B. Esto podría permitir acceso no autorizado a recursos de red, archivos compartidos o incluso a la infraestructura de red de la empresa B.

\subsubsection{Asociación maliciosa}

En esta situación más peligrosa, un dispositivo inalámbrico se configura para aparecer como un punto de acceso legítimo. El atacante crea un punto de acceso falso que imita la configuración de seguridad y el nombre de red de una red legítima. Cuando los usuarios se conectan a este punto de acceso malicioso, el operador puede robar contraseñas de usuarios legítimos y luego penetrar una red cableada a través de un punto de acceso inalámbrico legítimo.

Un ejemplo común es el ataque "Evil Twin", donde el atacante configura un punto de acceso con el mismo SSID que una red WiFi pública legítima, como "Starbucks\_WiFi". Los usuarios desprevenidos se conectan al punto de acceso falso, permitiendo al atacante interceptar todo el tráfico, incluyendo credenciales bancarias, correos electrónicos y otros datos sensibles.

\subsubsection{Redes ad hoc}

Las redes ad hoc son redes punto a punto entre computadoras inalámbricas sin punto de acceso entre ellas. Estas redes pueden representar una amenaza de seguridad significativa debido a la falta de un punto de control central. En una red ad hoc, cada dispositivo puede comunicarse directamente con otros dispositivos, creando múltiples puntos de vulnerabilidad.

Por ejemplo, en una conferencia de seguridad, varios participantes podrían crear una red ad hoc para compartir archivos. Sin embargo, si uno de los dispositivos está comprometido, puede propagar malware a todos los demás dispositivos en la red ad hoc, o interceptar comunicaciones entre otros participantes.

\subsubsection{Robo de identidad (MAC spoofing)}

El robo de identidad ocurre cuando un atacante puede interceptar el tráfico de red e identificar la dirección MAC de una computadora con privilegios de red. La dirección MAC es un identificador único asignado a cada interfaz de red, y muchos sistemas de seguridad utilizan filtrado MAC para controlar el acceso a la red.

En un ataque de MAC spoofing, el atacante cambia la dirección MAC de su dispositivo para que coincida con la de un dispositivo autorizado. Por ejemplo, si un administrador de red utiliza un laptop con la dirección MAC 00:1B:44:11:3A:B7, un atacante podría cambiar su dirección MAC para que coincida y obtener acceso a recursos de red restringidos.

\subsubsection{Ataques Man-in-the-Middle}

Este tipo de ataque es particularmente peligroso en redes inalámbricas. El atacante se posiciona entre el usuario y el punto de acceso, interceptando y potencialmente modificando las comunicaciones. Las redes inalámbricas son especialmente vulnerables a estos ataques debido a la naturaleza broadcast de las transmisiones.

Un ejemplo práctico sería un atacante en un café que utiliza herramientas como Wireshark para capturar paquetes de red. El atacante puede interceptar correos electrónicos, contraseñas y otros datos sensibles que los usuarios envían a través de la red WiFi pública. En casos más sofisticados, el atacante puede modificar el contenido de las comunicaciones sin que el usuario lo detecte.

\subsubsection{Denial of Service (DoS)}

En el contexto de una red inalámbrica, un ataque DoS ocurre cuando un atacante bombardea continuamente un punto de acceso inalámbrico u otro puerto inalámbrico accesible con varios mensajes de protocolo diseñados para consumir recursos del sistema. Estos ataques pueden ser especialmente efectivos en redes inalámbricas debido a las limitaciones de ancho de banda y procesamiento.

Un ejemplo es el ataque de desautenticación, donde el atacante envía frames de desautenticación falsos a dispositivos conectados a la red WiFi. Esto fuerza a los dispositivos a desconectarse de la red, interrumpiendo sus comunicaciones. El atacante puede mantener este ataque de forma continua, efectivamente bloqueando el acceso a la red para todos los usuarios legítimos.

\subsubsection{Inyección de red}

Un ataque de inyección de red se dirige a puntos de acceso inalámbricos que están expuestos a tráfico de red no filtrado, como mensajes de protocolo de enrutamiento o mensajes de gestión de red. Estos ataques pueden ser muy sofisticados y difíciles de detectar.

Un ejemplo específico es el uso de comandos de reconfiguración falsos para afectar routers y switches, degradando el rendimiento de la red. El atacante podría enviar paquetes maliciosos que contienen comandos de configuración falsos, causando que los dispositivos de red cambien sus configuraciones de manera no autorizada. Esto podría resultar en la interrupción de servicios críticos o la apertura de brechas de seguridad adicionales.

\subsection{Medidas de seguridad para transmisiones inalámbricas}

\subsubsection{Técnicas de ocultación de señales}

Las organizaciones pueden implementar varias medidas para hacer más difícil que un atacante localice sus puntos de acceso inalámbricos. La desactivación del broadcast SSID es una práctica fundamental que apaga la difusión del identificador del conjunto de servicios por puntos de acceso inalámbricos. Esto hace que la red sea menos visible para dispositivos que buscan redes disponibles, aunque no la oculta completamente de herramientas especializadas.

La asignación de nombres crípticos a SSIDs es otra técnica importante. En lugar de usar nombres que revelen la identidad de la organización, como "Empresa\_ABC\_WiFi", se pueden usar nombres genéricos o engañosos que no proporcionen información útil a potenciales atacantes. Por ejemplo, usar "Network\_001" en lugar de "Departamento\_Finanzas" puede ayudar a ocultar la función específica de la red.

La reducción de la potencia de señal es una medida práctica que ajusta la potencia de transmisión al nivel más bajo que aún proporcione la cobertura requerida. Esto limita el alcance de la señal, reduciendo la superficie de ataque potencial. En un entorno de oficina, esto podría significar que la señal WiFi no llegue más allá de las paredes del edificio.

La ubicación estratégica de puntos de acceso en el interior, alejados de ventanas y paredes exteriores, es otra medida física importante. Esto minimiza la propagación de señales hacia áreas públicas donde podrían ser interceptadas. En edificios de oficinas, los puntos de acceso se pueden ubicar en el centro de cada piso en lugar de cerca de las ventanas exteriores.

El uso de antenas direccionales puede proporcionar mayor seguridad al enfocar la señal en direcciones específicas. En lugar de antenas omnidireccionales que transmiten en todas las direcciones, las antenas direccionales pueden dirigir la señal solo hacia áreas autorizadas, reduciendo la exposición a áreas no autorizadas.

Las técnicas de blindaje de señales, como el uso de materiales especiales en paredes y ventanas, pueden aislar el entorno inalámbrico. Estos materiales pueden absorber o reflejar las señales WiFi, previniendo que se propaguen más allá de los límites deseados de la organización.

\subsubsection{Encriptación}

La encriptación de todas las transmisiones inalámbricas es fundamental para proteger la confidencialidad de los datos. Esta medida es efectiva contra la interceptación en la medida en que las claves de encriptación estén aseguradas y se utilicen algoritmos criptográficos robustos.

La implementación de encriptación debe ser obligatoria para todas las comunicaciones WiFi, sin excepciones. Esto incluye tanto el tráfico de datos como las comunicaciones de gestión de red. Sin encriptación, cualquier persona dentro del rango de la señal puede capturar y analizar el tráfico de red, incluyendo contraseñas, correos electrónicos y otros datos sensibles.

Es importante destacar que la efectividad de la encriptación depende de la fortaleza de las claves utilizadas y de la implementación correcta de los protocolos criptográficos. El uso de claves débiles o la implementación incorrecta de algoritmos de encriptación puede hacer que la protección sea inefectiva, incluso cuando está presente.

\subsection{Estándares de seguridad WiFi}

\subsubsection{IEEE 802.11i - Red de Seguridad Robusta (RSN)}

El estándar IEEE 802.11i define un conjunto completo de servicios de seguridad para redes inalámbricas. Este estándar fue desarrollado para abordar las vulnerabilidades del protocolo WEP (Wired Equivalent Privacy) original y proporcionar un nivel de seguridad comparable al de las redes cableadas.

El servicio de autenticación utiliza un protocolo específico para definir un intercambio entre un usuario y un servidor de autenticación (AS). Este intercambio proporciona autenticación mutua, asegurando que tanto el cliente como el servidor verifiquen sus identidades. Además, el proceso genera claves temporales que se utilizan entre el cliente y el punto de acceso sobre el enlace inalámbrico, proporcionando una capa adicional de seguridad.

El control de acceso es una función crítica que hace cumplir el uso de la función de autenticación, enruta los mensajes apropiadamente y facilita el intercambio de claves. Este componente asegura que solo los dispositivos autenticados puedan acceder a la red y que las comunicaciones se manejen de manera segura.

La privacidad con integridad de mensajes es fundamental para proteger los datos transmitidos. Los datos a nivel MAC se encriptan junto con un código de integridad de mensajes que asegura que los datos no hayan sido alterados durante la transmisión. Esto previene tanto la interceptación como la modificación no autorizada de los datos.

\subsubsection{Fases de operación de IEEE 802.11i}

El funcionamiento de un IEEE 802.11i RSN se puede dividir en cinco fases distintas que trabajan en conjunto para establecer una conexión segura. La fase de descubrimiento es el primer paso, donde un punto de acceso usa mensajes llamados Beacons y Probe Responses para anunciar su política de seguridad IEEE 802.11i. Estos mensajes informan a los clientes potenciales sobre las capacidades de seguridad disponibles y los requisitos de autenticación.

Durante la fase de autenticación, la estación (STA) y el servidor de autenticación (AS) prueban sus identidades entre sí. Este proceso puede utilizar varios métodos de autenticación, incluyendo certificados digitales, tokens de seguridad o credenciales de usuario. La autenticación exitosa es un prerrequisito para el acceso a la red.

La gestión de claves es una fase crítica donde el punto de acceso y la estación realizan varias operaciones que resultan en la generación de claves criptográficas. Estas claves se utilizan para encriptar el tráfico de datos y asegurar la integridad de las comunicaciones. El proceso incluye la derivación de claves maestras y la generación de claves temporales específicas para cada sesión.

La transferencia de datos protegida es la fase operacional donde los frames se intercambian entre la estación y la estación final a través del punto de acceso. Durante esta fase, todos los datos transmitidos están encriptados y protegidos contra modificaciones no autorizadas. La comunicación continúa de manera segura hasta que la sesión termina.

La terminación de conexión es la fase final donde el punto de acceso y la estación intercambian frames para cerrar la conexión segura de manera ordenada. Este proceso asegura que las claves de sesión se invalidan apropiadamente y que no se pueden reutilizar para futuras comunicaciones.

\subsubsection{Protocolos de protección de datos}

\paragraph{TKIP (Temporal Key Integrity Protocol)}

TKIP fue diseñado como una solución temporal para dispositivos existentes que no podían soportar los nuevos estándares de seguridad. Este protocolo requiere solo cambios de software en dispositivos implementados con el enfoque de seguridad de LAN inalámbrica más antiguo llamado WEP (Wired Equivalent Privacy), proporcionando una mejora significativa en seguridad sin requerir actualizaciones de hardware.

El servicio de integridad de mensajes de TKIP es fundamental para detectar modificaciones no autorizadas. TKIP agrega un código de integridad de mensajes al frame MAC 802.11 después del campo de datos. Este código se calcula utilizando una función hash criptográfica y se verifica en el receptor para asegurar que el mensaje no ha sido alterado durante la transmisión.

La confidencialidad de datos se proporciona encriptando el MPDU (MAC Protocol Data Unit) más el valor MIC usando el algoritmo RC4. Aunque RC4 tiene vulnerabilidades conocidas, TKIP implementa mejoras significativas sobre WEP, incluyendo la rotación de claves y la prevención de ataques de reinyección de paquetes.

\paragraph{CCMP (Counter Mode-CBC MAC Protocol)}

CCMP está destinado para dispositivos IEEE 802.11 más nuevos que están equipados con hardware para soportar este esquema más robusto. Este protocolo representa una mejora significativa sobre TKIP y proporciona un nivel de seguridad comparable al de las redes cableadas.

El código de autenticación de mensajes de CCMP utiliza el algoritmo CBC-MAC (Cipher Block Chaining-Message Authentication Code), que es más robusto que el método utilizado en TKIP. Este algoritmo proporciona una protección más fuerte contra ataques de integridad y asegura que los mensajes no puedan ser modificados sin ser detectados.

La confidencialidad de datos en CCMP se logra usando el modo CTR (Counter) de operación de cifrado de bloques con AES (Advanced Encryption Standard). AES es un algoritmo de encriptación simétrica ampliamente reconocido y aprobado por el gobierno de los Estados Unidos para información clasificada. El uso de AES en modo CTR proporciona encriptación eficiente y segura de los datos transmitidos.

\subsection{WiFi Protected Access (WPA)}

\subsubsection{WPA}

WPA (WiFi Protected Access) representa un conjunto de mecanismos de seguridad diseñados para eliminar la mayoría de los problemas de seguridad identificados en el estándar 802.11 original. Este protocolo se basó en el estado actual del estándar 802.11i y proporcionó una mejora significativa sobre WEP sin requerir cambios de hardware en dispositivos existentes.

WPA implementa TKIP (Temporal Key Integrity Protocol) para proporcionar encriptación y integridad de datos, mejorando significativamente la seguridad sobre WEP. Para la autenticación, WPA utiliza el estándar 802.1X en entornos empresariales, proporcionando autenticación robusta basada en servidores de autenticación. En configuraciones domésticas, WPA utiliza PSK (Pre-Shared Key), permitiendo una configuración más simple mientras mantiene un nivel de seguridad aceptable.

\subsubsection{WPA2}

WPA2 representa la implementación final del estándar 802.11i, también conocida como Red de Seguridad Robusta (RSN). Esta versión introduce mejoras significativas en términos de seguridad criptográfica y gestión de claves, estableciendo un nuevo estándar para la seguridad WiFi.

WPA2 utiliza CCMP (Counter Mode-CBC MAC Protocol) con el algoritmo AES (Advanced Encryption Standard) para proporcionar encriptación robusta que es ampliamente reconocida y aprobada por gobiernos y organizaciones de seguridad. El protocolo mantiene la compatibilidad con 802.1X para autenticación empresarial y PSK para configuraciones domésticas, pero introduce jerarquías de claves más robustas que mejoran significativamente la gestión de claves criptográficas.

\subsubsection{WPA3}

WPA3 es el estándar más reciente que introduce mejoras revolucionarias en la seguridad WiFi, abordando vulnerabilidades específicas identificadas en versiones anteriores y preparando las redes para los desafíos de seguridad futuros.

Una de las características más importantes de WPA3 es la encriptación individualizada, donde cada sesión tiene su propia clave de encriptación única. Esto significa que incluso si un atacante compromete una sesión, no puede acceder a otras comunicaciones en la red. WPA3 también implementa protección mejorada contra ataques de diccionario, utilizando métodos de autenticación más robustos que previenen ataques de fuerza bruta contra contraseñas débiles.

El concepto de forward secrecy asegura que las claves anteriores no pueden ser utilizadas para descifrar tráfico futuro, proporcionando protección adicional contra ataques de interceptación. Para entornos empresariales críticos, WPA3 incluye una suite de seguridad de 192 bits que cumple con los estándares más estrictos de seguridad gubernamental y corporativa.

\subsection{Autenticación 802.1X}

IEEE 802.1X es un estándar fundamental para el control de acceso basado en puertos para redes LAN. En el contexto específico de una WLAN 802.11, los términos corresponden directamente a la estación inalámbrica y el punto de acceso (AP). El servidor de autenticación (AS) típicamente es un dispositivo separado ubicado en el lado cableado de la red, proporcionando una capa adicional de seguridad y control.

\subsubsection{Componentes de 802.1X}

El estándar 802.1X define tres componentes principales que trabajan en conjunto para proporcionar autenticación robusta. El suplicante es el cliente que solicita acceso a la red, típicamente representado por la estación inalámbrica del usuario. Este componente es responsable de proporcionar las credenciales de autenticación y manejar el proceso de intercambio de claves.

El autenticador es el dispositivo que controla el acceso físico a la red, generalmente el punto de acceso inalámbrico. Este componente actúa como intermediario entre el suplicante y el servidor de autenticación, facilitando el proceso de autenticación sin tener acceso directo a las credenciales del usuario. El servidor de autenticación es el componente que verifica las credenciales del suplicante utilizando bases de datos de usuarios, directorios LDAP o servicios de autenticación externos como RADIUS.

\subsubsection{Proceso de autenticación}

El proceso de autenticación 802.1X sigue una secuencia específica de pasos diseñada para garantizar la seguridad y la integridad del proceso. El suplicante inicia el proceso enviando sus credenciales al autenticador, típicamente en respuesta a una solicitud de autenticación. Estas credenciales pueden incluir certificados digitales, tokens de seguridad o credenciales de usuario tradicionales como nombre de usuario y contraseña.

El autenticador actúa como intermediario, reenviando las credenciales recibidas al servidor de autenticación sin modificarlas. El servidor de autenticación entonces verifica las credenciales contra su base de datos de usuarios autorizados, aplicando las políticas de seguridad configuradas. Si las credenciales son válidas y el usuario cumple con todas las políticas de seguridad, el servidor envía una respuesta de éxito al autenticador, incluyendo información sobre los permisos y restricciones del usuario.

Finalmente, el autenticador permite el acceso a la red para el suplicante autenticado, configurando los parámetros de red apropiados como VLAN, ancho de banda y políticas de acceso. Este proceso asegura que solo los usuarios autorizados puedan acceder a la red, proporcionando una base sólida para la seguridad de la red inalámbrica.

\subsection{Mejores prácticas de seguridad WiFi}

\subsubsection{Configuración de puntos de acceso}

La configuración adecuada de puntos de acceso es fundamental para establecer una base sólida de seguridad WiFi. Cambiar las credenciales por defecto es el primer paso crítico, ya que muchos dispositivos vienen con contraseñas predefinidas que son ampliamente conocidas por los atacantes. Es esencial usar contraseñas fuertes y únicas que combinen letras, números y caracteres especiales, evitando información personal o patrones predecibles.

Desactivar la administración remota es otra práctica importante que limita el acceso administrativo solo a conexiones locales o VPNs seguras. Esto previene que atacantes externos intenten acceder a la configuración del punto de acceso desde Internet. Mantener el firmware actualizado regularmente es crucial, ya que los fabricantes lanzan parches de seguridad que abordan vulnerabilidades específicas identificadas en versiones anteriores.

Configurar filtrado MAC proporciona una capa adicional de control mediante la creación de una lista blanca de dispositivos autorizados. Aunque el filtrado MAC puede ser evadido por atacantes sofisticados, actúa como una barrera efectiva contra acceso casual no autorizado. Finalmente, reducir la potencia de transmisión al nivel mínimo necesario para la cobertura requerida minimiza el alcance de la señal, reduciendo la superficie de ataque potencial.

\subsubsection{Segmentación de red}

La segmentación de red es una estrategia fundamental que separa diferentes tipos de tráfico y usuarios en redes lógicas independientes. Las VLANs separadas permiten crear segmentos distintos para invitados, empleados y dispositivos IoT, cada uno con diferentes niveles de acceso y políticas de seguridad. Esta separación previene que un compromiso en un segmento afecte a otros segmentos de la red.

Los firewalls internos proporcionan control granular del tráfico entre segmentos, permitiendo a los administradores definir políticas específicas sobre qué comunicaciones están permitidas entre diferentes áreas de la red. Las redes aisladas son especialmente importantes para dispositivos críticos o sensibles que requieren el más alto nivel de protección, como servidores de bases de datos o sistemas de control industrial.

\subsubsection{Monitoreo y detección}

El monitoreo y detección proactivos son componentes esenciales de una estrategia de seguridad WiFi efectiva. El análisis de tráfico permite identificar patrones anómalos que pueden indicar actividad maliciosa, como conexiones inusuales, transferencias de datos excesivas o intentos de acceso fuera del horario normal de trabajo. Este análisis puede detectar tanto ataques externos como actividad maliciosa de usuarios internos.

El monitoreo de dispositivos es crucial para identificar dispositivos no autorizados que intentan conectarse a la red. Esto incluye la detección de puntos de acceso maliciosos (rogue access points) y dispositivos que no cumplen con las políticas de seguridad de la organización. Las alertas de seguridad proporcionan notificaciones inmediatas cuando se detectan eventos sospechosos, permitiendo una respuesta rápida antes de que los incidentes escalen.

Los logs de auditoría mantienen un registro completo de todas las actividades de red, incluyendo conexiones, desconexiones, intentos de autenticación fallidos y cambios en la configuración. Estos logs son esenciales para la investigación forense después de un incidente de seguridad y para demostrar cumplimiento con regulaciones de seguridad.

\subsection{Vulnerabilidades específicas de WiFi}

\subsubsection{KRACK (Key Reinstallation Attack)}

KRACK representa una vulnerabilidad crítica en el protocolo WPA2 que permite a los atacantes interceptar y manipular el tráfico WiFi de manera efectiva. Esta vulnerabilidad explota el handshake de 4 vías de WPA2 para reinstalar una clave ya en uso, comprometiendo la seguridad de las comunicaciones inalámbricas.

El ataque funciona interceptando el proceso de establecimiento de claves entre el cliente y el punto de acceso, manipulando los mensajes del handshake para forzar la reutilización de claves criptográficas. Esto permite al atacante descifrar el tráfico encriptado y potencialmente inyectar contenido malicioso en las comunicaciones.

Las medidas de mitigación incluyen actualizar todos los dispositivos con parches de seguridad que abordan esta vulnerabilidad específica. Usar WPA3 cuando sea posible proporciona protección adicional, ya que este estándar más reciente no es vulnerable a KRACK. Implementar VPNs para tráfico crítico añade una capa de encriptación adicional que protege los datos incluso si la capa WiFi está comprometida. El monitoreo continuo de conexiones sospechosas ayuda a detectar intentos de explotación de esta vulnerabilidad.

\subsubsection{Evil Twin Attacks}

Los ataques Evil Twin representan una amenaza sofisticada donde los atacantes crean puntos de acceso falsos que imitan perfectamente redes WiFi legítimas para interceptar el tráfico de usuarios desprevenidos. Estos ataques son particularmente efectivos en entornos públicos como cafés, aeropuertos y hoteles, donde los usuarios buscan conectividad WiFi.

El atacante configura un punto de acceso con el mismo SSID y características de seguridad que la red legítima, pero con una señal ligeramente más fuerte para atraer a los usuarios. Una vez que los usuarios se conectan al punto de acceso falso, el atacante puede interceptar todo el tráfico, incluyendo contraseñas, correos electrónicos y datos bancarios.

Las medidas de mitigación incluyen verificar certificados SSL/TLS antes de enviar información sensible, ya que los sitios web legítimos mantienen certificados válidos incluso cuando se accede a través de puntos de acceso falsos. Usar VPNs para tráfico sensible proporciona encriptación adicional que protege los datos incluso en redes comprometidas. Configurar alertas para nuevas redes WiFi ayuda a detectar puntos de acceso sospechosos, y educar a los usuarios sobre estos riesgos es fundamental para prevenir estos ataques.

\subsubsection{Deauthentication Attacks}

Los ataques de desautenticación son ataques de denegación de servicio específicos para redes WiFi que envían frames de desautenticación falsos para desconectar dispositivos de la red. Estos ataques pueden ser utilizados tanto para interrumpir servicios como para facilitar otros tipos de ataques más sofisticados.

El atacante envía frames de desautenticación que parecen provenir del punto de acceso legítimo, forzando a los dispositivos a desconectarse de la red. Este tipo de ataque es especialmente problemático porque puede ser ejecutado con herramientas relativamente simples y no requiere credenciales de red.

Las medidas de mitigación incluyen implementar detección de anomalías que puede identificar patrones de desconexión inusuales. Usar redes con autenticación robusta como WPA3 proporciona mejor protección contra estos ataques. El monitoreo de patrones de desconexión ayuda a identificar cuando estos ataques están ocurriendo, y configurar reconexión automática segura minimiza la interrupción del servicio para los usuarios legítimos.

\subsection{Seguridad en dispositivos móviles}

\subsubsection{Amenazas específicas a dispositivos móviles}

Los dispositivos móviles enfrentan amenazas únicas que los distinguen de los sistemas tradicionales de escritorio. La falta de controles de seguridad física es un problema fundamental, ya que los dispositivos móviles están bajo control completo del usuario y pueden ser perdidos, robados o comprometidos físicamente sin que la organización tenga conocimiento inmediato. Esta característica hace que los dispositivos móviles sean especialmente vulnerables a ataques de acceso físico.

El uso de dispositivos móviles no confiables representa otro desafío significativo, especialmente en entornos BYOD (Bring Your Own Device). Los dispositivos personales pueden no emplear encriptación adecuada o pueden tener configuraciones de seguridad insuficientes, creando puntos de entrada vulnerables a la red corporativa. El uso de redes no confiables, como WiFi público en cafés o aeropuertos, expone los dispositivos móviles a ataques de interceptación y man-in-the-middle.

La facilidad de instalación de aplicaciones de terceros en dispositivos móviles introduce riesgos adicionales, ya que estas aplicaciones pueden contener malware o solicitar permisos excesivos que comprometen la seguridad del dispositivo. La sincronización automática con otros dispositivos puede propagar malware o datos sensibles a sistemas no autorizados. El acceso a contenido no confiable, como códigos QR maliciosos o enlaces en redes sociales, puede exponer los dispositivos a ataques de phishing y malware.

El uso de servicios de ubicación GPS representa una amenaza de privacidad significativa, ya que esta información puede ser explotada para realizar ataques dirigidos o para crear perfiles detallados de los movimientos de los usuarios. Los atacantes pueden utilizar esta información para planificar ataques físicos o para identificar patrones de comportamiento que faciliten otros tipos de ataques.

\subsubsection{Estrategia de seguridad para dispositivos móviles}

Una estrategia efectiva de seguridad para dispositivos móviles debe abordar tres categorías principales de protección que trabajan en conjunto para crear un entorno seguro. La seguridad del dispositivo se enfoca en proteger el hardware y software del dispositivo móvil mediante la implementación de medidas como encriptación de datos, autenticación biométrica y gestión de aplicaciones. Esta capa incluye la capacidad de borrar remotamente los datos del dispositivo en caso de pérdida o robo.

La seguridad del tráfico cliente/servidor protege las comunicaciones entre el dispositivo móvil y los servicios corporativos. Esto incluye el uso de VPNs para encriptar todo el tráfico, implementación de certificados digitales para autenticación mutua, y monitoreo continuo de las comunicaciones para detectar actividad anómala. Esta capa es especialmente importante cuando los dispositivos se conectan a redes no confiables.

La seguridad de barrera protege la infraestructura de red corporativa contra amenazas que puedan provenir de dispositivos móviles comprometidos. Esto incluye la implementación de controles de acceso a red (NAC) que verifican el estado de seguridad del dispositivo antes de permitir la conexión, segmentación de red para aislar dispositivos móviles de sistemas críticos, y monitoreo de tráfico para detectar actividad maliciosa proveniente de dispositivos móviles.

\subsection{Herramientas de seguridad WiFi}

\subsubsection{Herramientas de análisis}

Las herramientas de análisis de seguridad WiFi son esenciales para evaluar la seguridad de las redes inalámbricas y identificar vulnerabilidades potenciales. Aircrack-ng representa una suite completa para auditoría de seguridad WiFi que incluye herramientas para captura de paquetes, análisis de tráfico y pruebas de penetración. Esta suite es ampliamente utilizada por profesionales de seguridad para evaluar la robustez de las configuraciones WiFi.

Kismet es un detector de redes inalámbricas y sniffer pasivo que puede identificar redes ocultas y detectar actividad anómala en el espectro WiFi. Esta herramienta es particularmente útil para identificar puntos de acceso maliciosos y dispositivos no autorizados en la red. Wireshark, aunque no es específico para WiFi, proporciona capacidades avanzadas de análisis de protocolos que pueden ser aplicadas al tráfico inalámbrico para identificar patrones sospechosos y ataques en curso.

inSSIDer es un escáner de redes WiFi que proporciona información detallada sobre las redes disponibles, incluyendo información sobre el tipo de encriptación, intensidad de señal y canales utilizados. Esta información es valiosa para optimizar la configuración de redes WiFi y detectar interferencias. NetStumbler es una herramienta específica para Windows que detecta redes WiFi y proporciona información sobre la configuración de seguridad de cada red.

\subsubsection{Herramientas de monitoreo}

Las herramientas de monitoreo especializadas son cruciales para mantener la seguridad de las redes WiFi en tiempo real. Los sistemas WIDS/WIPS (Wireless Intrusion Detection/Prevention Systems) proporcionan monitoreo continuo del espectro WiFi, detectando automáticamente dispositivos no autorizados, ataques de desautenticación y otros comportamientos maliciosos. Estos sistemas pueden configurarse para responder automáticamente a amenazas detectadas, bloqueando dispositivos sospechosos o alertando a los administradores.

Los analizadores de espectro son herramientas especializadas que detectan interferencias y señales no autorizadas en el espectro WiFi. Estas herramientas son esenciales para identificar ataques de jamming y otros tipos de interferencia que pueden afectar el funcionamiento de la red. Los monitores de tráfico proporcionan análisis detallado de los patrones de tráfico WiFi, permitiendo identificar anomalías que pueden indicar actividad maliciosa.

Las herramientas de gestión de claves facilitan la administración de certificados digitales y claves criptográficas utilizadas en la autenticación WiFi. Estas herramientas son especialmente importantes en entornos empresariales donde se utilizan certificados para la autenticación de dispositivos y usuarios.

\subsection{Consideraciones de cumplimiento}

\subsubsection{Estándares de la industria}

Los estándares de la industria proporcionan marcos de referencia esenciales para implementar medidas de seguridad WiFi efectivas y demostrar cumplimiento con regulaciones específicas. PCI DSS (Payment Card Industry Data Security Standard) es particularmente importante para organizaciones que procesan tarjetas de crédito, estableciendo requisitos específicos para la protección de datos de pago que incluyen medidas de seguridad WiFi robustas.

HIPAA (Health Insurance Portability and Accountability Act) establece estándares para organizaciones de salud, requiriendo medidas específicas para proteger la información médica confidencial transmitida a través de redes inalámbricas. SOX (Sarbanes-Oxley Act) aplica a empresas públicas, estableciendo requisitos de control interno que incluyen la seguridad de las comunicaciones corporativas, incluyendo WiFi.

ISO 27001 representa un estándar internacional de seguridad de la información que proporciona un marco comprehensivo para la gestión de seguridad de la información, incluyendo aspectos específicos de seguridad de redes inalámbricas. Este estándar es ampliamente reconocido y puede ser utilizado por organizaciones de cualquier tamaño y sector.

\subsubsection{Políticas organizacionales}

Las políticas organizacionales establecen las reglas y procedimientos que rigen el uso seguro de las redes WiFi dentro de una organización. La política de uso aceptable define claramente qué constituye uso apropiado de las redes WiFi corporativas, incluyendo restricciones sobre el tipo de tráfico permitido y las consecuencias de violar estas políticas. Esta política debe ser comunicada regularmente a todos los empleados y actualizada según sea necesario.

La política de dispositivos móviles establece reglas específicas para el uso de dispositivos personales en redes corporativas, incluyendo requisitos de seguridad mínimos y procedimientos para el acceso a recursos corporativos. Esta política es especialmente importante en entornos BYOD donde los empleados utilizan sus propios dispositivos para acceder a recursos corporativos.

La política de acceso remoto define los controles y procedimientos para acceder a recursos corporativos desde fuera de la oficina, incluyendo el uso de VPNs y autenticación multifactor. La política de incidentes establece procedimientos claros para responder a incidentes de seguridad WiFi, incluyendo pasos de contención, investigación y recuperación.

\subsection{Capa 7: Aplicación - Seguridad a nivel de aplicación}

La capa de aplicación es donde los usuarios interactúan directamente con los servicios de red. Es también donde ocurren muchos de los ataques más sofisticados.

\textbf{Protocolos y servicios principales}:
\begin{itemize}
    \item \textbf{HTTP/HTTPS}: Navegación web
    \item \textbf{SMTP/POP3/IMAP}: Correo electrónico
    \item \textbf{FTP/SFTP}: Transferencia de archivos
    \item \textbf{DNS}: Resolución de nombres
    \item \textbf{SSH}: Acceso remoto seguro
    \item \textbf{Telnet}: Acceso remoto (inseguro)
\end{itemize}

\textbf{Vulnerabilidades típicas}:
\begin{itemize}
    \item \textbf{SQL Injection}: Inyección de código SQL malicioso
    \item \textbf{Cross-Site Scripting (XSS)}: Ejecución de scripts en navegadores
    \item \textbf{Cross-Site Request Forgery (CSRF)}: Suplantación de solicitudes
    \item \textbf{Buffer Overflow}: Desbordamiento de buffers de memoria
    \item \textbf{Directory Traversal}: Acceso a archivos fuera del directorio web
    \item \textbf{Command Injection}: Ejecución de comandos del sistema
    \item \textbf{Session Hijacking}: Secuestro de sesiones de usuario
\end{itemize}

\textbf{Medidas de mitigación}:
    \begin{itemize}
    \item \textbf{Web Application Firewalls (WAF)}: Filtrado de tráfico malicioso
    \item \textbf{Validación de entrada}: Verificación de datos de entrada
    \item \textbf{Sanitización de datos}: Limpieza de datos de entrada
    \item \textbf{Autenticación multifactor}: Verificación en múltiples niveles
    \item \textbf{Encriptación de datos}: Protección de información sensible
    \item \textbf{Monitoreo de logs}: Detección de actividades sospechosas
    \item \textbf{Parches regulares}: Actualizaciones de seguridad
\end{itemize}

\subsection{Capa 6: Presentación - Seguridad de datos}

La capa de presentación se encarga de la codificación, compresión y encriptación de datos. Es fundamental para la confidencialidad e integridad de la información.

\textbf{Protocolos principales}:
\begin{itemize}
    \item \textbf{SSL/TLS}: Encriptación de comunicaciones
    \item \textbf{MIME}: Tipos de contenido multimedia
    \item \textbf{JPEG, MPEG}: Formatos de imagen y video
    \item \textbf{XML}: Lenguaje de marcado extensible
    \item \textbf{ASCII, UTF-8}: Codificación de caracteres
\end{itemize}

\textbf{Vulnerabilidades típicas}:
\begin{itemize}
    \item \textbf{Format String Attacks}: Explotación de funciones de formato
    \item \textbf{Encoding Attacks}: Ataques a través de codificación maliciosa
    \item \textbf{Compression Bombs}: Archivos que se expanden infinitamente
    \item \textbf{Man-in-the-Middle (MITM)}: Interceptación de comunicaciones SSL/TLS
    \item \textbf{Downgrade Attacks}: Forzar uso de protocolos menos seguros
    \item \textbf{Mal manejo de entradas}: Procesamiento incorrecto de datos
\end{itemize}

\textbf{Medidas de mitigación}:
\begin{itemize}
    \item \textbf{Validación de entrada}: Verificación de datos antes del procesamiento
    \item \textbf{Revisión de sistemas criptográficos}: Auditoría de SSL/TLS
    \item \textbf{Validación de certificados}: Verificación de certificados digitales
    \item \textbf{Monitoreo de protocolos}: Vigilancia de comunicaciones seguras
    \item \textbf{Implementación de TLS 1.3}: Último estándar de seguridad
    \item \textbf{Análisis de tráfico encriptado}: Detección de anomalías
    \end{itemize}

\subsection{Capa 5: Sesión - Seguridad de sesiones}

La capa de sesión maneja el establecimiento, mantenimiento y terminación de sesiones de comunicación entre aplicaciones.

\textbf{Protocolos principales}:
        \begin{itemize}
    \item \textbf{NetBIOS}: Servicios de red en Windows
    \item \textbf{RPC}: Llamadas a procedimientos remotos
    \item \textbf{SQL*Net}: Comunicación con bases de datos Oracle
    \item \textbf{PPTP/L2TP}: Protocolos de túnel VPN
    \item \textbf{SOCKS}: Proxy de nivel de sesión
        \end{itemize}

\textbf{Vulnerabilidades típicas}:
\begin{itemize}
    \item \textbf{Session Fixation}: Forzar el uso de un ID de sesión específico
    \item \textbf{Session Replay}: Reutilización de sesiones capturadas
    \item \textbf{Session Hijacking}: Secuestro de sesiones activas
    \item \textbf{Man-in-the-Middle}: Interceptación de comunicaciones
    \item \textbf{Brute Force}: Ataques de fuerza bruta en autenticación
    \item \textbf{Interceptación de credenciales}: Robo de passwords e identificadores
\end{itemize}

\textbf{Medidas de mitigación}:
\begin{itemize}
    \item \textbf{SSL/TLS}: Encriptación de sesiones
    \item \textbf{Tokens de sesión únicos}: Identificadores únicos por sesión
    \item \textbf{Timeouts automáticos}: Expiración automática de sesiones
    \item \textbf{Regeneración de IDs}: Cambio periódico de identificadores
    \item \textbf{Encriptación de sesiones}: Protección de datos de sesión
    \item \textbf{Monitoreo de actividad}: Detección de comportamiento anómalo
    \item \textbf{Nivel de corte}: Bloqueo tras múltiples intentos fallidos
\end{itemize}

\subsection{Capa 4: Transporte - Seguridad de transporte}

La capa de transporte es responsable de la comunicación confiable entre aplicaciones, incluyendo el control de flujo y la recuperación de errores.

\textbf{Protocolos principales}:
\begin{itemize}
    \item \textbf{TCP}: Protocolo de control de transmisión (orientado a conexión)
    \item \textbf{UDP}: Protocolo de datagramas de usuario (sin conexión)
    \item \textbf{SCTP}: Protocolo de control de transmisión de stream
    \item \textbf{DCCP}: Protocolo de control de congestión de datagramas
\end{itemize}

\textbf{Vulnerabilidades típicas}:
\begin{itemize}
    \item \textbf{SYN Flooding}: Sobreflujo de solicitudes de conexión TCP
    \item \textbf{Port Scanning}: Escaneo de puertos para identificar servicios
    \item \textbf{Session Hijacking}: Secuestro de conexiones TCP
    \item \textbf{TCP Sequence Number Prediction}: Predicción de números de secuencia
    \item \textbf{UDP Flooding}: Sobreflujo de paquetes UDP
    \item \textbf{Amplification Attacks}: Ataques de amplificación usando UDP
    \item \textbf{Fingerprinting}: Identificación de sistemas operativos
\end{itemize}

\textbf{Medidas de mitigación}:
\begin{itemize}
    \item \textbf{Firewalls de estado}: Control de conexiones activas
    \item \textbf{IPS/IDS}: Sistemas de detección y prevención de intrusiones
    \item \textbf{Rate limiting}: Limitación de velocidad de conexiones
    \item \textbf{SYN cookies}: Protección contra ataques SYN flood
    \item \textbf{Monitoreo de conexiones}: Vigilancia de conexiones activas
    \item \textbf{Honeypots}: Sistemas trampa para atrapar atacantes
    \item \textbf{Parches de seguridad}: Actualizaciones de sistemas
\end{itemize}

\subsection{Capa 3: Red - Seguridad de red}

La capa de red se encarga del enrutamiento de paquetes entre redes diferentes, incluyendo el direccionamiento lógico.

\textbf{Protocolos principales}:
\begin{itemize}
    \item \textbf{IP}: Protocolo de Internet (IPv4, IPv6)
    \item \textbf{ICMP}: Protocolo de mensajes de control de Internet
    \item \textbf{IGMP}: Protocolo de gestión de grupos de Internet
    \item \textbf{OSPF, BGP, RIP}: Protocolos de enrutamiento
\end{itemize}

\textbf{Vulnerabilidades típicas}:
\begin{itemize}
    \item \textbf{IP Spoofing}: Falsificación de direcciones IP
    \item \textbf{Man-in-the-Middle}: Interceptación de comunicaciones
    \item \textbf{DDoS/DoS}: Ataques de denegación de servicio
    \item \textbf{Route Poisoning}: Envenenamiento de tablas de enrutamiento
    \item \textbf{ICMP Flooding}: Sobreflujo de mensajes ICMP
    \textbf{Smurf Attacks}: Ataques usando ICMP echo
    \item \textbf{Ping of Death}: Paquetes ICMP malformados
\end{itemize}

\textbf{Medidas de mitigación}:
\begin{itemize}
    \item \textbf{Firewalls de red}: Filtrado de paquetes
    \item \textbf{IDS/IPS}: Detección y prevención de intrusiones
    \item \textbf{Filtrado de ICMP/IGMP}: Limitación de mensajes de control
    \item \textbf{Servicios de mitigación DDoS}: Protección contra ataques distribuidos
    \item \textbf{Monitoreo de tráfico}: Vigilancia de patrones anómalos
    \item \textbf{Implementación de BCP38}: Mejores prácticas de filtrado
    \item \textbf{Rate limiting}: Limitación de velocidad de paquetes
\end{itemize}

\subsection{Capa 2: Enlace de Datos - Seguridad de enlace}

La capa de enlace de datos maneja la comunicación entre dispositivos en la misma red local, incluyendo el control de acceso al medio.

\textbf{Protocolos principales}:
\begin{itemize}
    \item \textbf{Ethernet}: Estándar para redes cableadas
    \item \textbf{WiFi (802.11)}: Estándar para redes inalámbricas
    \item \textbf{PPP}: Protocolo punto a punto
    \item \textbf{VLAN}: Redes locales virtuales
    \item \textbf{Frame Relay, ATM}: Protocolos WAN
\end{itemize}

\textbf{Vulnerabilidades típicas}:
\begin{itemize}
    \item \textbf{MAC Spoofing}: Falsificación de direcciones MAC
    \item \textbf{ARP Poisoning}: Envenenamiento de tabla ARP
    \item \textbf{VLAN Hopping}: Salto entre VLANs
    \item \textbf{MAC Flooding}: Sobreflujo de direcciones MAC
    \item \textbf{Rogue Access Points}: Puntos de acceso maliciosos
    \item \textbf{Evil Twin Attacks}: Puntos de acceso falsos
    \item \textbf{DHCP Spoofing}: Falsificación de servidor DHCP
\end{itemize}

\textbf{Medidas de mitigación}:
\begin{itemize}
    \item \textbf{Port Security}: Control de acceso por MAC
    \item \textbf{VLANs}: Segmentación de redes
    \item \textbf{802.1X}: Autenticación de dispositivos
    \item \textbf{ARP monitoring}: Vigilancia de tabla ARP
    \item \textbf{Switches inteligentes}: Dispositivos con funciones de seguridad
    \item \textbf{Monitoreo de dispositivos}: Control de dispositivos autorizados
    \item \textbf{Encriptación WiFi}: Protección de comunicaciones inalámbricas
\end{itemize}

\subsection{Capa 1: Física - Seguridad física}

La capa física se encarga de la transmisión de bits a través del medio físico, incluyendo cables y ondas de radio. Esta capa es la base de toda comunicación digital y presenta vulnerabilidades únicas que pueden afectar directamente el medio de transmisión y la infraestructura física.

Los estándares principales incluyen Ethernet (IEEE 802.3) para redes cableadas, WiFi (IEEE 802.11) para redes inalámbricas, fibra óptica para transmisión por luz, cable coaxial y par trenzado para transmisión por pares de cobre. Cada uno de estos medios presenta características específicas de seguridad que deben ser consideradas.

\textbf{Vulnerabilidades específicas de la capa física}:

\textbf{Jamming (especialmente en WiFi)}: Este ataque bloquea las señales inalámbricas, haciendo que los equipos no logren conectarse. El atacante transmite señales de radio en la misma frecuencia que la red WiFi, creando interferencia que impide la comunicación legítima. En entornos empresariales, esto puede paralizar completamente las operaciones que dependen de conectividad inalámbrica.

\textbf{Pinchado de cable (tapping)}: En cables de cobre o fibra, un atacante puede colocar un dispositivo que copia la señal del cable. En fibra óptica, esto se logra a veces con una leve curvatura controlada que "filtra" luz, permitiendo la interceptación sin interrumpir la comunicación. Este tipo de ataque es particularmente peligroso porque puede ser difícil de detectar.

\textbf{Conexión física no autorizada}: Un atacante puede acceder físicamente a un rack u oficina y conectar un equipo (mini-switch, laptop, Raspberry Pi) a un puerto libre. Este dispositivo puede actuar como punto de entrada para ataques posteriores o como herramienta de interceptación de tráfico.

\textbf{Sabotaje físico}: El corte de cables o desconexión de equipos puede interrumpir completamente los servicios de red. Este tipo de ataque es simple pero efectivo, especialmente en infraestructuras críticas.

\textbf{Robo físico}: La sustracción de equipos puede resultar en pérdida de datos y acceso no autorizado a la infraestructura de red. Los dispositivos robados pueden contener configuraciones sensibles o credenciales almacenadas.

\textbf{Interceptación de datos}: La captura de señales electromagnéticas puede permitir la interceptación de comunicaciones sin acceso físico directo al cable. Esto es especialmente relevante en entornos donde las señales pueden ser capturadas desde el exterior.

\textbf{Medidas de mitigación para la capa física}:

\textbf{Control de acceso físico}: La implementación de cerraduras, cámaras de seguridad y control de acceso en salas de servidores es fundamental para prevenir acceso no autorizado a la infraestructura física. Los centros de datos deben tener múltiples niveles de seguridad física.

\textbf{Uso de cableado blindado}: Los cables blindados electromagnéticamente pueden prevenir interferencias y escuchas no autorizadas. Este tipo de cableado es especialmente importante en entornos donde la seguridad es crítica.

\textbf{Monitoreo de infraestructura}: La implementación de sensores, alarmas y sistemas de detección de desconexiones puede alertar rápidamente sobre problemas físicos en la infraestructura de red.

\textbf{Separación de redes sensibles}: Evitar que redes críticas compartan infraestructura física puede reducir el impacto de ataques físicos. Las redes más sensibles deben tener infraestructura dedicada.

\textbf{Protección contra jamming}: El uso de frecuencias seguras y sistemas de detección de interferencias puede ayudar a identificar y mitigar ataques de jamming en redes inalámbricas.

\textbf{Jaulas de Faraday}: Para instalaciones críticas, las jaulas de Faraday pueden proporcionar protección contra interferencias electromagnéticas y ataques de interceptación.

\textbf{Análisis de espectro}: Los analizadores de espectro pueden detectar interferencias y señales no autorizadas en el entorno de red, permitiendo la identificación temprana de ataques.

\section{Casos de estudio: Ataques por capa}

\subsection{Caso 1: Ataque a nivel de aplicación - SQL Injection}

\textbf{Escenario}: Un atacante explota una vulnerabilidad SQL Injection en un sitio web de comercio electrónico.

\textbf{Proceso del ataque}:
\begin{enumerate}
    \item El atacante identifica un formulario vulnerable en el sitio web
    \item Inyecta código SQL malicioso en el campo de entrada
    \item El servidor ejecuta el código SQL sin validación
    \item El atacante obtiene acceso a la base de datos
    \item Extrae información confidencial de clientes
\end{enumerate}

\textbf{Medidas de mitigación}:
\begin{itemize}
    \item Validación y sanitización de entrada
    \item Uso de consultas preparadas
    \item Implementación de WAF
    \item Monitoreo de logs de base de datos
    \item Encriptación de datos sensibles
\end{itemize}

\subsection{Caso 2: Ataque a nivel de transporte - SYN Flood}

\textbf{Escenario}: Un atacante realiza un ataque SYN flood contra un servidor web.

\textbf{Proceso del ataque}:
\begin{enumerate}
    \item El atacante envía múltiples paquetes SYN
    \item El servidor responde con SYN-ACK
    \item El atacante no responde con ACK
    \item El servidor mantiene conexiones en estado SYN-RECEIVED
    \item Se agotan los recursos del servidor
    \item El servidor se vuelve inaccesible
\end{enumerate}

\textbf{Medidas de mitigación}:
\begin{itemize}
    \item Implementación de SYN cookies
    \item Rate limiting en firewalls
    \item Monitoreo de conexiones TCP
    \item Servicios de mitigación DDoS
    \item Configuración de timeouts apropiados
\end{itemize}

\subsection{Caso 3: Ataque a nivel de red - Man-in-the-Middle}

\textbf{Escenario}: Un atacante intercepta comunicaciones entre un cliente y un servidor.

\textbf{Proceso del ataque}:
\begin{enumerate}
    \item El atacante se posiciona entre el cliente y el servidor
    \item Intercepta el tráfico de red
    \item Modifica o registra las comunicaciones
    \item Puede robar credenciales o datos sensibles
    \item El cliente y servidor no detectan la interceptación
\end{enumerate}

\textbf{Medidas de mitigación}:
\begin{itemize}
    \item Uso de VPNs
    \item Implementación de HTTPS/TLS
    \item Validación de certificados digitales
    \item Monitoreo de tráfico de red
    \item Implementación de certificados pinning
\end{itemize}

\section{Herramientas de seguridad de redes}

\subsection{Herramientas de análisis de tráfico}

\begin{itemize}
    \item \textbf{Wireshark}: Analizador de paquetes de red
    \item \textbf{tcpdump}: Captura de paquetes desde línea de comandos
    \item \textbf{NetFlow}: Análisis de flujo de tráfico
    \item \textbf{Snort}: Sistema de detección de intrusiones
    \item \textbf{Suricata}: Motor de detección de amenazas
\end{itemize}

\subsection{Herramientas de monitoreo}

\begin{itemize}
    \item \textbf{Nagios}: Monitoreo de infraestructura
    \item \textbf{Zabbix}: Monitoreo de red y aplicaciones
    \item \textbf{PRTG}: Monitoreo de red empresarial
    \item \textbf{SolarWinds}: Suite de monitoreo de red
    \item \textbf{OpenNMS}: Monitoreo de red de código abierto
\end{itemize}

\subsection{Herramientas de análisis de vulnerabilidades}

\begin{itemize}
    \item \textbf{Nessus}: Escáner de vulnerabilidades
    \item \textbf{OpenVAS}: Escáner de vulnerabilidades de código abierto
    \item \textbf{Qualys}: Plataforma de gestión de vulnerabilidades
    \item \textbf{Rapid7}: Herramientas de seguridad de red
    \item \textbf{Nmap}: Escáner de puertos y servicios
\end{itemize}

\section{Mejores prácticas de seguridad de redes}

\subsection{Arquitectura de red segura}

\begin{itemize}
    \item \textbf{Segmentación de red}: Separación de sistemas críticos
    \item \textbf{Zonas de seguridad}: Creación de áreas con diferentes niveles de protección
    \item \textbf{DMZ}: Zona desmilitarizada para servicios públicos
    \item \textbf{VLANs}: Separación lógica de tráfico
    \item \textbf{Microsegmentación}: Protección granular de aplicaciones
\end{itemize}

\subsection{Control de acceso}

\begin{itemize}
    \item \textbf{Autenticación multifactor}: Verificación en múltiples niveles
    \item \textbf{Control de acceso basado en roles (RBAC)}: Permisos según función
    \item \textbf{Principio de menor privilegio}: Acceso mínimo necesario
    \item \textbf{802.1X}: Autenticación de dispositivos en red
    \item \textbf{NAC}: Control de acceso a red
\end{itemize}

\subsection{Monitoreo y respuesta}

\begin{itemize}
    \item \textbf{SIEM}: Gestión de eventos e información de seguridad
    \item \textbf{Análisis de comportamiento}: Detección de anomalías
    \item \textbf{Respuesta automática}: Acciones automáticas ante amenazas
    \item \textbf{Forense de red}: Análisis de incidentes
    \item \textbf{Threat hunting}: Búsqueda proactiva de amenazas
\end{itemize}

\section{Tendencias emergentes en seguridad de redes}

\subsection{Seguridad definida por software (SDS)}

\begin{itemize}
    \item \textbf{SDN}: Redes definidas por software
    \item \textbf{NFV}: Virtualización de funciones de red
    \item \textbf{SDP}: Perímetro definido por software
    \item \textbf{Zero Trust}: Modelo de confianza cero
    \item \textbf{SASE}: Secure Access Service Edge
\end{itemize}

\subsection{Automatización e inteligencia artificial}

\begin{itemize}
    \item \textbf{SOAR}: Orquestación, automatización y respuesta de seguridad
    \item \textbf{IA para detección}: Machine learning en análisis de tráfico
    \item \textbf{Automatización de respuesta}: Respuesta automática a incidentes
    \item \textbf{Análisis predictivo}: Predicción de amenazas futuras
    \item \textbf{Threat intelligence}: Inteligencia de amenazas
\end{itemize}

\subsection{Seguridad en redes 5G/6G}

\begin{itemize}
    \item \textbf{Network slicing}: Segmentación de redes virtuales
    \item \textbf{Edge computing}: Procesamiento en el borde de la red
    \item \textbf{IoT security}: Seguridad de dispositivos IoT
    \item \textbf{Quantum-resistant cryptography}: Criptografía post-cuántica
    \item \textbf{AI-powered security}: Seguridad basada en inteligencia artificial
\end{itemize}

\section{Conclusiones}

La seguridad de redes es un campo dinámico que requiere un enfoque holístico y en capas. El modelo OSI proporciona un marco conceptual valioso para implementar controles de seguridad apropiados en cada nivel de la comunicación de red.

Los profesionales de seguridad deben:

\begin{itemize} 
    \item Comprender las vulnerabilidades específicas de cada capa
    \item Implementar controles de seguridad apropiados
    \item Mantener un monitoreo continuo de la infraestructura
    \item Actualizar regularmente las medidas de seguridad
    \item Adaptarse a las nuevas amenazas y tecnologías
    \item Colaborar con equipos técnicos y de seguridad
\end{itemize}

La evolución hacia redes más complejas y la proliferación de dispositivos IoT presentan nuevos desafíos que requieren enfoques innovadores y la integración de tecnologías emergentes como la inteligencia artificial y la automatización.

\section{Casos de estudio adicionales}

\subsection{Caso 3: Análisis de incidente DNS}

\textbf{Escenario}: Usted es un analista de ciberseguridad que trabaja en una empresa que se especializa en brindar servicios de consultoría de TI. Varios clientes se comunicaron con su empresa para informar que no podían acceder al sitio web de la empresa www.yummyrecipesforme.com y vieron el error "puerto de destino inalcanzable" después de esperar a que se cargara la página.

\textbf{Análisis}: Tiene la tarea de analizar la situación y determinar qué protocolo de red se vio afectado durante este incidente. Para comenzar, visita el sitio web y también recibe el error "puerto de destino inalcanzable". A continuación, carga su herramienta de análisis de red, tcpdump, y carga la página web nuevamente. Esta vez, recibe muchos paquetes en su analizador de red. El analizador muestra que cuando envía paquetes UDP y recibe una respuesta ICMP devuelta a su host, los resultados contienen un mensaje de error: "puerto udp 53 inalcanzable".

\textbf{Diagnóstico}: El puerto UDP 53 corresponde al servicio DNS. El error indica que el servidor DNS no está respondiendo, lo que impide la resolución de nombres de dominio.

\subsection{Caso 4: Incidente de puerto HTTPS bloqueado}

\textbf{Escenario}: Los registros del analizador de protocolo de red indican que el puerto 443 no es accesible al intentar acceder al sitio web seguro de verificación de antecedentes de empleados. El puerto 443 se utiliza normalmente para el tráfico HTTPS.

\textbf{Análisis}: El incidente ocurrió temprano en la mañana cuando el equipo de recursos humanos (RH) informó que no podían acceder al portal web de verificación de antecedentes. El equipo de seguridad de la red respondió y comenzó a realizar pruebas con la herramienta analizadora de protocolo de red tcpdump. Los registros resultantes revelaron que el puerto 443, que se utiliza para el tráfico HTTPS, no es accesible.

\textbf{Investigación}: Los próximos pasos incluyen verificar la configuración del firewall para ver si el puerto 443 está bloqueado y contactar al administrador del sistema del servidor web para que compruebe el sistema en busca de signos de un ataque. El equipo de RH cree que es posible que un nuevo empleado quiera evitar que realicen la verificación de antecedentes. El equipo de seguridad de la red sospecha que esta persona podría haber lanzado un ataque para bloquear el sitio web de verificación de antecedentes.

\section{Técnicas de evasión y ataques avanzados}

\subsection{Fast Flux}

\textbf{Fast Flux} es una técnica de DNS utilizada por botnets para ocultar actividades de phishing, proxy web, entrega de malware y comunicación de malware detrás de hosts comprometidos que actúan como proxy. El propósito de utilizar la red Fast Flux es hacer que la comunicación entre el malware y su servidor de comando y control (C\&C) sea difícil de descubrir para los profesionales de la seguridad.

\textbf{Características principales}:
\begin{itemize}
    \item \textbf{Múltiples direcciones IP}: Asociadas con un nombre de dominio que cambia constantemente
    \item \textbf{Rotación rápida}: Cambio frecuente de direcciones IP para evadir detección
    \item \textbf{Hosts comprometidos}: Uso de máquinas infectadas como proxies
    \item \textbf{Resistencia a bloqueos}: Dificulta el bloqueo por IP específica
\end{itemize}

\textbf{Referencia}: Palo Alto Networks creó un escenario ficticio para explicar Fast Flux: \url{https://unit42.paloaltonetworks.com/fast-flux-101/}

\subsection{Ataques Punycode}

Los ataques Punycode utilizan caracteres Unicode para crear dominios que se ven legítimos pero redirigen a sitios maliciosos.

\textbf{Medidas de protección}:
\begin{itemize}
    \item \textbf{Verificación de certificados}: Hacer clic en el candado para inspeccionar el certificado HTTPS
    \item \textbf{Administrador de contraseñas}: Reduce el riesgo de pegar contraseñas en sitios poco fiables
    \item \textbf{Configuración del navegador}: Obligar a mostrar nombres Punycode (disponible en Firefox)
    \item \textbf{Solución de seguridad móvil}: Protección adicional en dispositivos móviles
    \item \textbf{Verificación de ofertas}: Ir al sitio original de la empresa para comprobar disponibilidad
\end{itemize}

\textbf{Señales de alerta}:
\begin{itemize}
    \item Presión para actuar rápidamente
    \item Ofertas "solo por tiempo limitado"
    \item Ventanas emergentes que dificultan salir
    \item Letras extrañas en la barra de direcciones
    \item Diseño del sitio web diferente al original
\end{itemize}

\subsection{URL Shorteners maliciosos}

Los atacantes suelen ocultar los dominios maliciosos en acortadores de URL. Según Cofense, los atacantes utilizan los siguientes servicios de acortamiento de URL para generar enlaces maliciosos:

\begin{itemize}
    \item bit.ly
    \item goo.gl
    \item ow.ly
    \item s.id
    \item smarturl.it
    \item pequeño.pl
    \item tinyurl.com
    \item x.co
\end{itemize}

\textbf{Protección}: Puede ver el sitio web real al que lo redirige el enlace agregando un "+" al final de la URL corta.

\subsection{Evasión de detección por hash}

Los profesionales de seguridad suelen utilizar los valores hash para obtener información sobre una muestra de malware específica, un archivo malicioso o sospechoso, y como una forma de identificar y hacer referencia de forma única al artefacto malicioso.

\textbf{Herramientas de búsqueda de hash}:
\begin{itemize}
    \item VirusTotal
    \item Metadefender Cloud - OPSWAT
    \item sny.run (caza malware)
\end{itemize}

\textbf{Limitaciones}: Como atacante, modificar un archivo aunque sea un solo bit es trivial, lo que produciría un valor hash diferente. Con tantas variaciones e instancias de malware o ransomware conocidos, la búsqueda de amenazas utilizando hashes de archivos como IOC (indicadores de compromiso) puede resultar difícil.

\section{Herramientas específicas de análisis de red}

\subsection{Nmap (Network Mapper)}

Nmap es un escáner de red que nos ayuda a descubrir máquinas en ejecución y cualquier programa que se ejecute en ellas que sea visible para el mundo exterior.

\textbf{Capacidades principales}:
\begin{itemize}
    \item \textbf{Descubrimiento de hosts}: Identificar dispositivos activos en la red
    \item \textbf{Escaneo de puertos}: Detectar servicios abiertos
    \item \textbf{Detección de sistema operativo}: Identificar el SO de los hosts
    \item \textbf{Scripts de detección}: Automatizar tareas de análisis
    \item \textbf{Generación de reportes}: Documentar hallazgos
\end{itemize}

\subsection{Netcat}

El comando netcat, también conocido como nc, es una utilidad de línea de comandos que permite a los usuarios leer y escribir datos a través de una conexión de red. Se puede usar para establecer conexiones con servidores y clientes, enviar y recibir datos y realizar una variedad de otras tareas relacionadas con la red.

\textbf{Usos principales}:
\begin{itemize}
    \item \textbf{Transferencia de archivos}: Enviar y recibir archivos entre sistemas
    \item \textbf{Testing de puertos}: Verificar si un puerto está abierto
    \item \textbf{Proxy simple}: Crear conexiones proxy básicas
    \item \textbf{Banner grabbing}: Obtener información de servicios
    \item \textbf{Backdoor}: Crear acceso remoto (uso legítimo en testing)
\end{itemize}

\subsection{Snort - Sistema de detección de intrusiones}

Snort es un sistema de prevención y detección de intrusos en la red (NIDS/NIPS) de código abierto y basado en reglas. Fue desarrollado y mantenido por Martin Roesch, colaboradores de código abierto y el equipo de Cisco Talos.

\textbf{Capacidades de Snort}:
\begin{itemize}
    \item \textbf{Análisis de tráfico en vivo}: Monitoreo en tiempo real
    \item \textbf{Detección de ataques y sondeos}: Identificación de amenazas
    \item \textbf{Registro de paquetes}: Captura de tráfico para análisis
    \item \textbf{Análisis de protocolo}: Inspección profunda de protocolos
    \item \textbf{Alertas en tiempo real}: Notificaciones inmediatas
    \item \textbf{Módulos y complementos}: Extensibilidad del sistema
    \item \textbf{Preprocesadores}: Análisis avanzado de tráfico
    \item \textbf{Soporte multiplataforma}: Linux y Windows
\end{itemize}

\textbf{Tres modos de uso principales}:
\begin{itemize}
    \item \textbf{Modo sniffer}: Lee paquetes IP y los solicita en la aplicación de la consola
    \item \textbf{Modo registrador de paquetes}: Registra todos los paquetes IP (entrantes y salientes) que visitan la red
    \item \textbf{Modos NIDS/NIPS}: Registra/elimina los paquetes que se consideran maliciosos de acuerdo con las reglas definidas por el usuario
\end{itemize}

\section{Controles de seguridad específicos}

\subsection{NAT (Network Address Translation)}

NAT es un método de asignación de dirección de IP que usa una alternativa de dirección de IP pública para esconder la IP interna de un sistema.

\textbf{Beneficios de seguridad}:
\begin{itemize}
    \item \textbf{Ocultación de IPs internas}: Las direcciones privadas no son visibles desde Internet
    \item \textbf{Reducción de superficie de ataque}: Menos direcciones expuestas
    \item \textbf{Control de acceso}: Filtrado de tráfico entrante
    \item \textbf{Conservación de direcciones IPv4}: Uso eficiente del espacio de direcciones
\end{itemize}

\subsection{Filtrado de paquetes}

El filtrado de paquetes es un proceso que sucede cada vez que un router recibe un paquete. El dispositivo filtra paquetes comparándolo con reglas configuradas por el administrador de red, que le dice al dispositivo qué permitir y qué no. Si no se especifica regla, entonces el firewall lo bloquea, e informa o no.

\textbf{Características}:
\begin{itemize}
    \item \textbf{Reglas configurables}: Definidas por el administrador
    \item \textbf{Decisión binaria}: Permitir o bloquear
    \item \textbf{Inspección de encabezados}: Análisis de información de paquetes
    \item \textbf{Logging opcional}: Registro de decisiones tomadas
\end{itemize}

\subsection{Control de acceso a la red (NAC)}

NAC controla la idoneidad de los dispositivos antes del acceso a la red. Está diseñado para verificar que las especificaciones y condiciones del dispositivo cumplan con el perfil predeterminado antes de conectarse a la red.

\textbf{Funciones principales}:
\begin{itemize}
    \item \textbf{Evaluación de dispositivos}: Verificar cumplimiento de políticas
    \item \textbf{Control de acceso}: Permitir o denegar conexión
    \item \textbf{Remediación automática}: Corregir problemas de seguridad
    \item \textbf{Segmentación}: Aislar dispositivos no conformes
\end{itemize}

\subsection{Gestión de identidad y acceso (IAM)}

IAM controla y administra las identidades de los activos y el acceso de los usuarios a los sistemas y recursos de datos a través de la red.

\textbf{Componentes principales}:
\begin{itemize}
    \item \textbf{Autenticación}: Verificación de identidad
    \item \textbf{Autorización}: Asignación de permisos
    \item \textbf{Administración de cuentas}: Gestión del ciclo de vida
    \item \textbf{Auditoría}: Registro de actividades
\end{itemize}

\subsection{Balanceo de carga}

El balanceo de carga controla el uso de recursos para distribuir (basado en métricas) tareas sobre un conjunto de recursos y mejorar el flujo general de procesamiento de datos.

\textbf{Beneficios de seguridad}:
\begin{itemize}
    \item \textbf{Distribución de carga}: Evita sobrecarga de servidores
    \item \textbf{Alta disponibilidad}: Redundancia de servicios
    \item \textbf{Protección contra DDoS}: Dispersión de ataques
    \item \textbf{Monitoreo de salud}: Detección de servicios comprometidos
\end{itemize}

\subsection{Segmentación de red}

La segmentación de red crea y controla rangos de red y segmentación para aislar los niveles de acceso de los usuarios, agrupar activos con funcionalidades comunes y mejorar la protección de dispositivos/datos sensibles/internos en una red más segura.

\textbf{Técnicas de segmentación}:
\begin{itemize}
    \item \textbf{VLANs}: Separación lógica de tráfico
    \item \textbf{Microsegmentación}: Protección granular
    \item \textbf{Zonas de seguridad}: Áreas con diferentes niveles de protección
    \item \textbf{Firewalls internos}: Control entre segmentos
\end{itemize}

\section{Análisis de tráfico detallado}

\subsection{Técnicas de análisis}

Hay dos técnicas principales utilizadas en el análisis de tráfico:

\subsubsection{Análisis de flujo}

Recopilación de datos/evidencia de los dispositivos de red. Este tipo de análisis tiene como objetivo proporcionar resultados estadísticos a través del resumen de datos sin aplicar una investigación profunda a nivel de paquete.

\textbf{Ventajas}:
\begin{itemize}
    \item Fácil de recopilar y analizar
    \item Bajo impacto en el rendimiento de red
    \item Escalabilidad para redes grandes
    \item Análisis estadístico eficiente
\end{itemize}

\textbf{Desafíos}:
\begin{itemize}
    \item No proporciona detalles completos del paquete
    \item Limitado para obtener la causa raíz de un caso
    \item Puede perder información importante
\end{itemize}

\subsubsection{Análisis de paquetes}

Recopilación de todos los datos de red disponibles. Aplicación de una investigación en profundidad a nivel de paquetes (a menudo llamada Inspección profunda de paquetes - DPI) para detectar y bloquear paquetes anómalos y maliciosos.

\textbf{Ventajas}:
\begin{itemize}
    \item Proporciona detalles completos del paquete
    \item Permite obtener la causa raíz de un caso
    \item Detección precisa de amenazas
    \item Análisis forense completo
\end{itemize}

\textbf{Desafíos}:
\begin{itemize}
    \item Requiere tiempo y habilidades para analizar
    \item Alto impacto en el rendimiento de red
    \item Requiere almacenamiento significativo
    \item Complejidad en el análisis
\end{itemize}

\subsection{Inspección profunda de paquetes (DPI)}

DPI es una técnica que examina el contenido de los paquetes de red, no solo los encabezados, para detectar y bloquear tráfico malicioso.

\textbf{Capacidades}:
\begin{itemize}
    \item \textbf{Análisis de contenido}: Inspección del payload de paquetes
    \item \textbf{Detección de malware}: Identificación de código malicioso
    \item \textbf{Filtrado de aplicaciones}: Control basado en contenido
    \item \textbf{Análisis de protocolos}: Verificación de cumplimiento
\end{itemize}

\section{Zero Trust Microsoft}

Como cumplimiento de directivas unificado, la directiva Confianza cero intercepta la solicitud, verifica explícitamente las señales de los seis elementos fundamentales, basándose en la configuración de directivas, y aplica acceso con privilegios mínimos.

\textbf{Elementos fundamentales}:
\begin{itemize}
    \item \textbf{Rol del usuario}: Identidad y permisos
    \item \textbf{Ubicación}: Contexto geográfico y de red
    \item \textbf{Cumplimiento normativo del dispositivo}: Estado de seguridad
    \item \textbf{Confidencialidad de los datos}: Clasificación de información
    \item \textbf{Confidencialidad de la aplicación}: Sensibilidad del servicio
    \item \textbf{Telemetría}: Información de estado y comportamiento
\end{itemize}

\textbf{Características principales}:
\begin{itemize}
    \item \textbf{Evaluación continua}: Verificación durante toda la sesión
    \item \textbf{Protección automatizada}: Respuesta en tiempo real a amenazas
    \item \textbf{Segmentación}: Filtros de tráfico antes del acceso
    \item \textbf{Encriptación}: Protección de datos en tránsito y reposo
    \item \textbf{Control de acceso adaptable}: Para aplicaciones SaaS y locales
\end{itemize}

\textbf{Implementación}:
\begin{itemize}
    \item \textbf{Etiquetado y clasificación}: De correos, documentos y datos
    \item \textbf{Control de tiempo de ejecución}: Para infraestructura
    \item \textbf{Servicios bajo demanda}: Just-in-time (JIT)
    \item \textbf{Controles de versiones}: En funcionamiento activo
    \item \textbf{Telemetría integrada}: Con análisis y evaluación
\end{itemize}

\section{Referencias y recursos adicionales}

\begin{itemize}
    \item IEEE 802.1X - Port-Based Network Access Control
    \item RFC 5246 - The Transport Layer Security (TLS) Protocol
    \item RFC 826 - Address Resolution Protocol (ARP)
    \item RFC 791 - Internet Protocol (IP)
    \item RFC 793 - Transmission Control Protocol (TCP)
\end{itemize}

\subsection{Organizaciones de seguridad}

\begin{itemize}
    \item \textbf{CERT/CC}: Centro de Coordinación de Incidentes Informáticos
    \item \textbf{SANS Institute}: Entrenamiento y recursos de seguridad
    \item \textbf{ISC²}: Certificaciones profesionales en seguridad
    \item \textbf{CompTIA}: Certificaciones en tecnologías de la información
    \item \textbf{OWASP}: Open Web Application Security Project
\end{itemize}

\subsection{Lecturas recomendadas}

\begin{itemize}
    \item "Network Security: Private Communication in a Public World" - Charlie Kaufman
    \item "Applied Cryptography" - Bruce Schneier
    \item "The Art of Deception" - Kevin Mitnick
    \item "Hacking: The Art of Exploitation" - Jon Erickson
    \item "Computer Networks" - Andrew S. Tanenbaum
\end{itemize}

























\section{Ataques}

Irrumpir en una red de destino generalmente incluye una serie de pasos. Según Lockheed Martin , the Cyber Kill Chain consta de siete pasos: 
\begin{itemize}
    \item \textbf{Recon:} Recon, abreviatura de reconocimiento, se refiere al paso en el que el atacante intenta aprender tanto como sea posible sobre el objetivo. La información como los tipos de servidores, el sistema operativo, las direcciones IP, los nombres de los usuarios y las direcciones de correo electrónico pueden ayudar al éxito del ataque. 
    \item \textbf{Armamento}: este paso se refiere a preparar un archivo con un componente malicioso, por ejemplo, para proporcionar acceso remoto al atacante. \item \textbf{Entrega:} Entrega significa entregar el archivo ``armado'' al objetivo a través de cualquier método factible, como correo electrónico o memoria flash USB. 
    \item \textbf{Explotación:} cuando el usuario abre el archivo malicioso, su sistema ejecuta el componente malicioso. 
    \item \textbf{Instalación}: el paso anterior debería instalar el malware en el sistema de destino. 
    \item \textbf{Comando y control (C2)}: la instalación exitosa del malware proporciona al atacante una capacidad de comando y control sobre el sistema de destino. 
    \item \textbf{Acciones sobre los objetivos}: después de obtener el control de un sistema de destino, el atacante ha logrado sus objetivos. Un objetivo de ejemplo es la filtración de datos (robar los datos del objetivo).
\end{itemize}

Se centra en el diseño del sistema, la operación y la gestión de la arquitectura/infraestructura para proporcionar accesibilidad, integridad, continuidad y confiabilidad a la red. 

Análisis de Tráfico
El análisis de tráfico (a menudo llamado análisis de tráfico de red. se centra en dos conceptos centrales: autenticación y autorización. 


\subsection{Casos}

\begin{tcolorbox}[colback=gray!5!white,colframe=orange!60!gray,title=caso 1]
\textbf{Antes del ataque}: un grupo de estudiantes universitarios creó una botnet con la intención de atacar varios servidores y redes de juegos. Una botnet es un conjunto de computadoras infectadas por malware que están bajo el control de un único actor de amenaza, conocido como ``bot-herder''. Cada computadora en la botnet se puede controlar de forma remota para enviar un paquete de datos a un sistema de destino. En un ataque de botnet, los ciberdelincuentes instruyen a todos los bots de la botnet a enviar paquetes de datos al sistema objetivo al mismo tiempo, lo que resulta en un ataque DDoS.
%
El grupo de estudiantes universitarios publicó el código de la botnet en línea para que fuera accesible a miles de usuarios de Internet y las autoridades no pudieran rastrear la botnet hasta los estudiantes. Al hacerlo, hicieron posible que otros actores maliciosos aprendieran el código de la botnet y lo controlaran de forma remota. 

\textbf{El día del ataque}
A las 7:00 am del día del ataque, la botnet envió decenas de millones de solicitudes de DNS al proveedor de servicios. Esto abrumó el sistema y el servicio DNS se cerró. Esto significaba que no se podía acceder a todos los sitios web que utilizaban el proveedor de servicios. Cuando los usuarios intentaron acceder a varios sitios web, no fueron dirigidos al sitio web que escribieron en su navegador. Se produjeron interrupciones en cada servicio en toda América del Norte y Europa. \\ \textbf{Referencia}: \url{https://www.welivesecurity.com/la-es/2016/10/26/ataques-ddos-a-la-iot-octubre/} 
\end{tcolorbox}

hint: \textit{servidores DNS traducen los nombres de dominio de los sitios web a la dirección IP del sistema que contiene la información del sitio web. Por ejemplo, si un usuario escribiera la URL de un sitio web, un servidor DNS la traduciría en una dirección IP numérica que dirige el tráfico de la red a la ubicación del servidor del sitio web.}


%\textbf{Ejercicio: buscar la IP de uai.cl - escribirla en el navegador - chequear que suceda lo descrito.}





\textbf{Pregunta}\\
Los ataques DDoS pueden ser muy perjudiciales para una organización. Como analista de seguridad, es importante reconocer la gravedad de un ataque de este tipo para conocer las oportunidades de proteger la red de ellos. ¿Qué acciones concretas  pueden tomar para ayudar para que si se envían nuevos ataques, la empresa esté preparada y mitigue el impacto?

\end{itemize}


