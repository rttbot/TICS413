\chapterimage{Pictures/map.png}
\chapter{Introducción}
\vspace{95px}
\begin{flushright}
    \textit{\textit{"x"}. x.}
\end{flushright}
%capitulo I y III libro 
%Y por qué son excusas baratas
%Consecuencias de un ataque
%Preliminares
%Principios de seguridad
%El señor Hacker



\section{¿Qué es la Seguridad}


La {seguridad computacional}, la seguridad de la información y la ciberseguridad son términos relacionados, pero tienen enfoques y alcances ligeramente diferentes en el ámbito de la protección de la tecnología y la información. 

\textbf{La seguridad computacional  (Computer Security)} se centra en la protección de sistemas informáticos, incluidos hardware, software y redes, contra amenazas físicas y lógicas que podrían afectar su disponibilidad, integridad y confidencialidad. Los aspectos clave de la seguridad computacional incluyen la \textbf{prevención de accesos no autorizados}, el \textbf{control de acceso a los recursos}, la \textbf{protección contra malware} y la \textbf{implementación de medidas para garantizar la continuidad del servicio}.
Busca proteger los activos tecnológicos y recursos informáticos de una organización o individuo.

La \textbf{Seguridad de la Información} (Information Security) se enfoca en la \textbf{protección de la información}, ya sea almacenada digitalmente o en formato físico. Esta disciplina busca salvaguardar la confidencialidad, integridad y disponibilidad de los datos. Incluye la \textbf{aplicación de controles, políticas y procedimientos para proteger la información sensible y evitar que caiga en manos no autorizadas o se vea comprometida de alguna manera}. Abarca aspectos más amplios que solo los sistemas informáticos y se aplica a todos los aspectos de la información en una organización, como los datos almacenados en bases de datos, archivos, documentos impresos, comunicaciones electrónicas, entre otros.

La \textbf{{Ciberseguridad}} (Cybersecurity) es una rama específica de la seguridad de la información que se enfoca en proteger los sistemas informáticos y las redes contra ataques cibernéticos. Estos ataques pueden provenir de diversas fuentes, como hackers, ciberdelincuentes, gobiernos hostiles o incluso errores humanos.
%
Los profesionales de la ciberseguridad trabajan para identificar y mitigar vulnerabilidades en sistemas y redes, implementar medidas preventivas y reactivas para protegerse contra amenazas cibernéticas y también se ocupan de la detección y respuesta a incidentes de seguridad.

Acorde al NIST SP 800, los siguientes conceptos deben ser comprendido por todos los que trabajen en ambientes que potencialmente pueden recibir ataques:
\begin{itemize}
    \item \textbf{Evento} es cualquier ocurrencia observable en una red o sistema. 

    \item \textbf{Incumplimiento} (NIST SP 800-53 Rev. 5):  La pérdida de control, el compromiso, la divulgación no autorizada, la adquisición no autorizada o cualquier evento similar en el que: una persona que no sea un usuario autorizado acceda o acceda potencialmente a información de identificación personal; o un usuario autorizado accede a información de identificación personal para un propósito distinto al autorizado. 
    
    \item \textbf{incidente}: Un evento que real o potencialmente pone en peligro la confidencialidad, integridad o disponibilidad de un sistema de información o la información que el sistema procesa, almacena o transmite
   
    \item \textbf{Intrusión}: Un evento de seguridad, o una combinación de eventos, que constituye un incidente de seguridad deliberado en el que un intruso obtiene o intenta obtener acceso a un sistema o recurso del sistema sin autorización.
   
    \item \textbf{Amenaza}: Cualquier circunstancia o evento con el potencial de afectar negativamente las operaciones de la organización (incluyendo la misión, las funciones, la imagen o la reputación), los activos de la organización, las personas, otras organizaciones o la nación a través de un sistema de información a través del acceso no autorizado, la destrucción, la divulgación, la modificación de la información y/ o denegación de servicio
    \item \textbf{Vulnerabilidad} Debilidad en un sistema de información, procedimientos de seguridad del sistema, controles internos o implementación que podría ser aprovechada por una fuente de amenazas
    \item \textbf{Día cero}: Una vulnerabilidad del sistema previamente desconocida con el potencial de explotación sin riesgo de detección o prevención porque, en general, no se ajusta a patrones, firmas o métodos reconocidos. 
    \item \textbf{Exploit} Un ataque particular. Se llama así porque estos ataques aprovechan las vulnerabilidades del sistema.
\end{itemize}

\subsection{Confidencialidad}
La confidencialidad se refiere a la protección de la información contra accesos no autorizados. Esto implica el uso de técnicas como el cifrado, que asegura que solo las partes autorizadas puedan acceder a la información. La pérdida de confidencialidad puede resultar en la divulgación de datos sensibles, lo que puede tener consecuencias legales y financieras significativas.

\subsection{Integridad}
La integridad asegura que los datos no sean alterados de manera no autorizada. Esto es vital para mantener la confianza en la información que se maneja. Las técnicas de control de versiones y los hashes son herramientas comunes para verificar la integridad de los datos. Un ataque que compromete la integridad puede llevar a decisiones erróneas basadas en información manipulada.

\subsection{Disponibilidad}
La disponibilidad garantiza que la información esté accesible cuando se necesite. Esto es especialmente crítico en entornos empresariales donde la inactividad puede resultar en pérdidas financieras. Los ataques de denegación de servicio (DoS) son un ejemplo de cómo se puede comprometer la disponibilidad, afectando la capacidad de una organización para operar.



\section{Modelos de Amenazas}
Al diseñar una defensa, es fundamental conocer el objetivo. Esto incluye definir los activos en riesgo, las vulnerabilidades existentes y las capacidades del atacante. Solo así se puede determinar cómo la defensa puede prevenir el éxito del ataque.

\subsection{Ejemplo de Modelado de Amenazas: HTTPS}
En el caso de las comunicaciones HTTPS, se negocia una clave en comunicación abierta, conocida solo por el usuario y el servidor. Todo el contenido se cifra con esta clave, lo que protege la información sensible. El proceso de handshake en HTTPS implica varios pasos:

\begin{enumerate}
    \item \textbf{Conexión Inicial}: El cliente envía un mensaje "ClientHello" al servidor, indicando las versiones de protocolo y los métodos de cifrado que soporta.
    \item \textbf{Respuesta del Servidor}: El servidor responde con un mensaje "ServerHello", eligiendo la versión de protocolo y el método de cifrado que se utilizará.
    \item \textbf{Intercambio de Certificados}: El servidor envía su certificado digital al cliente, que contiene la clave pública del servidor. El cliente verifica la autenticidad del certificado.
    \item \textbf{Generación de Clave de Sesión}: El cliente genera una clave de sesión, la cifra con la clave pública del servidor y la envía al servidor.
    \item \textbf{Establecimiento de Conexión Segura}: Ambos, cliente y servidor, utilizan la clave de sesión para cifrar la comunicación.
\end{enumerate}

Este proceso asegura que incluso si un atacante intercepta la comunicación, no podrá descifrar los datos sin la clave de sesión.
\section{Gestión de Riesgos}
La gestión de riesgos es un proceso esencial en la seguridad informática que implica identificar, evaluar y tratar los riesgos asociados con las amenazas a los activos de información. Este proceso ayuda a las organizaciones a priorizar sus esfuerzos de seguridad y a asignar recursos de manera efectiva.

\subsection{Análisis de Riesgos}
El análisis de riesgos comienza con la identificación de activos clave y las amenazas a las que están expuestos. Se busca determinar el nivel de riesgo que cada amenaza representa para la organización. Esto se puede expresar mediante la fórmula:

\[
\text{Riesgo} = (\text{Probabilidad de que ocurra la amenaza}) \times (\text{Costo para la organización})
\]

Este enfoque permite a la dirección evaluar los riesgos y decidir cómo tratarlos. La evaluación de riesgos también implica clasificar las amenazas en función de su probabilidad de ocurrencia y el impacto que tendrían en la organización.

\subsection{Determinación de la Probabilidad}
La probabilidad de que una amenaza ocurra se clasifica generalmente en las siguientes categorías:

\begin{table}[h]
    \centering
    \caption{Clasificación de la Probabilidad de Ocurrencia}
    \begin{tabularx}{\textwidth}{@{}XXX@{}}
        \toprule
        \textbf{Rating} & \textbf{Descripción} & \textbf{Definición Expandida} \\ \midrule
        1 & Rara & Puede ocurrir solo en circunstancias excepcionales y puede considerarse "desafortunado" o muy poco probable. \\
        2 & Poco Probable & Podría ocurrir en algún momento, pero no se espera dado los controles actuales. \\
        3 & Posible & Podría ocurrir en algún momento, pero también podría no ocurrir. \\
        4 & Probable & Probablemente ocurrirá en algunas circunstancias. \\
        5 & Casi Cierto & Se espera que ocurra en la mayoría de las circunstancias. \\ \bottomrule
    \end{tabularx}
\end{table}

\subsection{Determinación de Consecuencias}
Es fundamental especificar las consecuencias de que una amenaza se materialice. Esto no solo se refiere al impacto en el activo afectado, sino también a las repercusiones para la organización en su conjunto. Las consecuencias se pueden clasificar como:

\begin{table}[h]
    \centering
    \caption{Clasificación de Consecuencias}
 \begin{tabularx}{\textwidth}{@{}XXX@{}}
        \toprule
        \textbf{Rating} & \textbf{Consecuencia} & \textbf{Definición Expandida} \\ \midrule
        1 & Insignificante & Resultado de una brecha de seguridad menor. El impacto es breve y requiere poca o ninguna intervención. \\
        2 & Menor & Resultado de una brecha de seguridad que puede ser manejada sin intervención de la alta dirección. \\
        3 & Moderado & Brecha de seguridad que requiere intervención significativa y puede tener consecuencias graves. \\
        4 & Mayor & Brecha de seguridad que requiere intervención significativa y puede resultar en daños severos y prolongados. \\
        5 & Catastrófico & Brecha de seguridad que resulta en daños severos y prolongados, afectando gravemente a la organización. \\ \bottomrule
     
    \end{tabularx}
  
\end{table}

 


\section{Superficie de Ataque}
La superficie de ataque se refiere a las vulnerabilidades accesibles y explotables en un sistema. Incluye el software, la red y los humanos. Por ejemplo, un empleado con acceso a información sensible puede ser vulnerable a un ataque de ingeniería social. La gestión de la superficie de ataque es crucial para proteger los activos de la empresa, ya que cada punto de acceso representa una posible vulnerabilidad que los atacantes pueden explotar.

\subsection{Ejemplos de Superficie de Ataque}
Los ejemplos de superficie de ataque incluyen puertos abiertos en servidores web, servicios disponibles dentro de un firewall y código que procesa datos entrantes. Cada uno de estos elementos puede ser un punto de entrada para un atacante. Por lo tanto, es esencial realizar auditorías de seguridad periódicas para identificar y mitigar estas vulnerabilidades.

\section{Estrategia de Seguridad}
La estrategia de seguridad debe incluir especificaciones claras sobre los objetivos, la implementación de mecanismos de prevención y detección, y la evaluación continua de la efectividad de las medidas adoptadas. Esto implica la capacitación de empleados, el uso de sistemas de detección de intrusos (IDS) y la realización de pruebas de seguridad periódicas.

\subsection{Prevención y Detección}
La prevención es clave, y se deben implementar firewalls y segmentación de red para proteger los sistemas críticos. La detección de intrusos es igualmente importante, y el uso de un IDS permite monitorear la actividad sospechosa en la red, generando alertas automáticas cuando se detectan intentos de acceso no autorizado.

\subsection{Pruebas de Seguridad}
Las pruebas de seguridad periódicas, como el pentesting, son esenciales para identificar vulnerabilidades antes de que puedan ser explotadas. Estas pruebas simulan ataques reales para evaluar la efectividad de las medidas de seguridad implementadas.


\section{ Análisis de Riesgos}

\begin{itemize}
\item Identificación de Activos: Enumera al menos cinco activos clave de una organización ficticia.
\item Identificación de Amenazas: Para cada activo, identifica al menos dos amenazas potenciales.
\item Evaluación de Riesgos: Utiliza la fórmula de riesgo para calcular el nivel de riesgo para cada amenaza, considerando una probabilidad y un costo estimado.
\item Propuestas de Mitigación: Propón al menos dos medidas de mitigación para cada amenaza identificada.
\end{itemize}

A continuación, se presenta un ejercicio práctico para aplicar los conceptos: 
\textbf{Activo}: Base de datos de clientes.\\
\textbf{Amenazas}: 
- Inyección SQL.
- Acceso no autorizado.\\
\textbf{Evaluación de Riesgos}:
- Inyección SQL: Probabilidad = 4 (Probable), Costo = \$50,000.\\
\[
\text{Riesgo} = 4 \times 50,000 = 200,000
\]

\textbf{Propuestas de Mitigación}:
- Implementar validación de entrada.
- Usar un firewall de aplicaciones web.


\subsection{Prioridad de Riesgos}
Una vez que se han identificado y evaluado los riesgos, es crucial priorizarlos para determinar cuáles requieren atención inmediata. La priorización se basa en la combinación de la probabilidad de ocurrencia y el impacto que tendría cada riesgo en la organización. Esto permite a la dirección enfocar sus recursos en los riesgos más críticos.

La priorización de riesgos se puede realizar mediante un registro de riesgos, que clasifica cada riesgo en función de su probabilidad y consecuencias. Este registro no solo ayuda a visualizar los riesgos, sino que también proporciona una base para la toma de decisiones sobre las medidas de mitigación a implementar.




\section{Superficie de Ataque}
La superficie de ataque se refiere a las vulnerabilidades accesibles y explotables en un sistema. Incluye el software, la red y los humanos. Por ejemplo, un empleado con acceso a información sensible puede ser vulnerable a un ataque de ingeniería social. La gestión de la superficie de ataque es crucial para proteger los activos de la empresa, ya que cada punto de acceso representa una posible vulnerabilidad que los atacantes pueden explotar.

\subsection{Ejemplos de Superficie de Ataque}
Los ejemplos de superficie de ataque incluyen puertos abiertos en servidores web, servicios disponibles dentro de un firewall y código que procesa datos entrantes. Cada uno de estos elementos puede ser un punto de entrada para un atacante. Por lo tanto, es esencial realizar auditorías de seguridad periódicas para identificar y mitigar estas vulnerabilidades.

\section{Estrategia de Seguridad}
La estrategia de seguridad debe incluir especificaciones claras sobre los objetivos, la implementación de mecanismos de prevención y detección, y la evaluación continua de la efectividad de las medidas adoptadas. Esto implica la capacitación de empleados, el uso de sistemas de detección de intrusos (IDS) y la realización de pruebas de seguridad periódicas.

\subsection{Prevención y Detección}
La prevención es clave, y se deben implementar firewalls y segmentación de red para proteger los sistemas críticos. La detección de intrusos es igualmente importante, y el uso de un IDS permite monitorear la actividad sospechosa en la red, generando alertas automáticas cuando se detectan intentos de acceso no autorizado.

\subsection{Pruebas de Seguridad}
Las pruebas de seguridad periódicas, como el pentesting, son esenciales para identificar vulnerabilidades antes de que puedan ser explotadas. Estas pruebas simulan ataques reales para evaluar la efectividad de las medidas de seguridad implementadas.


\section{Amenazas y Ataques}

En esta sección abordaremos las principales amenazas y ataques que enfrentan los sistemas informáticos modernos. Comenzaremos con el software malicioso (malware) y luego analizaremos los ataques de denegación de servicio (DoS).
\section{Software Malicioso (Malware)}

\subsection{Definición y clasificación del malware}

El malware se puede definir como software que se ejecuta en un sistema sin el conocimiento o consentimiento del propietario del sistema. Esta definición abarca una amplia gama de programas maliciosos que pueden tener diferentes objetivos y métodos de operación.

\textbf{Características comunes del malware}:
\begin{itemize}
    \item \textbf{Ejecución no autorizada}: Se ejecuta sin el conocimiento del usuario
    \item \textbf{Objetivos maliciosos}: Diseñado para causar daño o beneficio ilícito
    \item \textbf{Ocultación}: Intenta evadir la detección por sistemas de seguridad
    \item \textbf{Propagación}: Se replica y se extiende a otros sistemas
    \item \textbf{Persistencia}: Mantiene su presencia en el sistema infectado
\end{itemize}

\subsection{Tipos principales de malware}

\subsubsection{Advanced Persistent Threats (APTs)}

Las APTs han surgido como una de las amenazas más sofisticadas en los últimos años. No son un nuevo tipo de malware, sino la aplicación bien financiada y persistente de una amplia variedad de tecnologías de intrusión y malware a objetivos seleccionados, generalmente empresariales o políticos.

\textbf{Características de las APTs}:
\begin{itemize}
    \item \textbf{Advanced}: Uso de una amplia variedad de tecnologías de intrusión y malware, incluyendo el desarrollo de malware personalizado si es necesario
    \item \textbf{Persistent}: Aplicación determinada de los ataques durante un período extendido contra el objetivo elegido para maximizar las posibilidades de éxito
    \item \textbf{Threats}: Amenazas a los objetivos seleccionados como resultado de atacantes organizados, capaces y bien financiados
\end{itemize}

\textbf{Técnicas utilizadas por las APTs}:
\begin{itemize}
    \item \textbf{Social engineering}: Manipulación psicológica de usuarios
    \item \textbf{Spear-phishing}: Correos electrónicos dirigidos específicamente
    \item \textbf{Drive-by-downloads}: Descargas automáticas desde sitios web comprometidos
    \item \textbf{Malware sofisticado}: Con múltiples mecanismos de propagación y payloads
\end{itemize}

\textbf{Ejemplos históricos}: Aurora, RSA, APT1, Stuxnet

\subsubsection{Virus}

Un virus es una pieza de software que puede "infectar" otros programas, o cualquier tipo de contenido ejecutable, modificándolos. La modificación incluye inyectar el código original con una rutina para hacer copias del código del virus, que luego puede continuar infectando otro contenido.

\textbf{Características de los virus}:
\begin{itemize}
    \item \textbf{Infección}: Se adjunta a archivos ejecutables legítimos
    \item \textbf{Replicación}: Se copia a sí mismo en otros archivos
    \item \textbf{Propagación}: Se extiende de computadora a computadora
    \item \textbf{Ejecución secreta}: Se ejecuta cuando el programa huésped se ejecuta
    \item \textbf{Especificidad}: Específico para sistema operativo y hardware
\end{itemize}

\textbf{Historia}: Los virus informáticos aparecieron por primera vez a principios de la década de 1980, y el término se atribuye a Fred Cohen. El virus Brain, visto por primera vez en 1986, fue uno de los primeros en apuntar a sistemas MSDOS.

\subsubsection{Adware}

El adware es software que muestra anuncios no deseados al usuario. Aunque no siempre es malicioso por naturaleza, puede ser molesto y puede recopilar información del usuario sin su consentimiento.

\textbf{Características del adware}:
\begin{itemize}
    \item \textbf{Anuncios intrusivos}: Muestra publicidad no solicitada
    \item \textbf{Recopilación de datos}: Puede rastrear hábitos de navegación
    \item \textbf{Reducción del rendimiento}: Puede ralentizar el sistema
    \item \textbf{Cambios en el navegador}: Puede modificar configuraciones
\end{itemize}

\subsubsection{Attack Kit}

Los attack kits son herramientas que permiten a los atacantes crear malware personalizado sin necesidad de conocimientos técnicos avanzados. Estos kits proporcionan interfaces gráficas y funcionalidades predefinidas para crear malware.

\textbf{Características de los attack kits}:
\begin{itemize}
    \item \textbf{Facilidad de uso}: Interfaces gráficas intuitivas
    \item \textbf{Personalización}: Opciones para personalizar el malware
    \item \textbf{Actualizaciones}: Reciben actualizaciones regulares
    \item \textbf{Distribución}: Se venden en mercados clandestinos
\end{itemize}

\subsubsection{Auto-rooter}

Un auto-rooter es una herramienta que automatiza el proceso de obtención de acceso root (administrador) en sistemas vulnerables. Estas herramientas escanean redes en busca de sistemas vulnerables y explotan automáticamente las vulnerabilidades encontradas.

\textbf{Características de los auto-rooters}:
\begin{itemize}
    \item \textbf{Escaneo automático}: Busca sistemas vulnerables
    \item \textbf{Explotación automática}: Explota vulnerabilidades sin intervención manual
    \item \textbf{Acceso privilegiado}: Obtiene acceso de administrador
    \item \textbf{Propagación}: Puede infectar múltiples sistemas
\end{itemize}

\subsubsection{Backdoor}

Un backdoor es un método secreto para eludir la autenticación normal o el cifrado en un sistema. Los backdoors pueden ser instalados por malware o pueden ser creados intencionalmente por desarrolladores para acceso de mantenimiento.

\textbf{Características de los backdoors}:
\begin{itemize}
    \item \textbf{Acceso oculto}: Proporciona acceso no autorizado al sistema
    \item \textbf{Evasión de seguridad}: Evita mecanismos de autenticación normales
    \item \textbf{Control remoto}: Permite control del sistema desde el exterior
    \item \textbf{Persistencia}: Mantiene el acceso incluso después de reinicios
\end{itemize}

\subsubsection{Downloaders}

Los downloaders son programas que descargan e instalan otros programas maliciosos en el sistema. Actúan como el primer paso en una cadena de infección, descargando malware más sofisticado.

\textbf{Características de los downloaders}:
\begin{itemize}
    \item \textbf{Descarga automática}: Descarga malware sin intervención del usuario
    \item \textbf{Instalación silenciosa}: Instala malware sin notificación
    \item \textbf{Actualizaciones}: Puede descargar versiones actualizadas del malware
    \item \textbf{Evasión}: Intenta evadir la detección por antivirus
\end{itemize}

\subsection{Métodos de propagación del malware}

\subsubsection{Propagación por red}

El malware puede propagarse a través de redes utilizando varios métodos:

\begin{itemize}
    \item \textbf{Vulnerabilidades de red}: Explota fallos en protocolos o servicios
    \item \textbf{Compartir archivos}: Se propaga a través de recursos compartidos
    \item \textbf{Correo electrónico}: Se adjunta a mensajes de correo
    \item \textbf{Sitios web maliciosos}: Se descarga desde sitios comprometidos
    \item \textbf{Dispositivos móviles}: Se propaga a través de conexiones Bluetooth o WiFi
\end{itemize}

\subsubsection{Propagación por medios físicos}

El malware también puede propagarse a través de medios físicos:

\begin{itemize}
    \item \textbf{Dispositivos USB}: Se ejecuta automáticamente al conectar
    \item \textbf{CDs/DVDs}: Se incluye en medios ópticos
    \item \textbf{Dispositivos móviles}: Se propaga a través de conexiones físicas
    \item \textbf{Tarjetas de memoria}: Se almacena en medios extraíbles
\end{itemize}

\subsection{Detección y prevención del malware}

\subsubsection{Métodos de detección}

\begin{itemize}
    \item \textbf{Detección basada en firmas}: Identifica malware conocido
    \item \textbf{Detección basada en comportamiento}: Identifica actividad sospechosa
    \item \textbf{Detección basada en heurísticas}: Usa reglas para identificar malware desconocido
    \item \textbf{Análisis de sandbox}: Ejecuta archivos en un entorno aislado
    \item \textbf{Análisis de tráfico de red}: Monitorea comunicaciones maliciosas
\end{itemize}

\subsubsection{Medidas de prevención}

\begin{itemize}
    \item \textbf{Software antivirus actualizado}: Mantener protección al día
    \item \textbf{Parches de seguridad}: Actualizar sistemas regularmente
    \item \textbf{Educación del usuario}: Capacitar sobre amenazas
    \item \textbf{Políticas de seguridad}: Establecer reglas claras
    \item \textbf{Backups regulares}: Mantener copias de seguridad
    \item \textbf{Segmentación de red}: Aislar sistemas críticos
\end{itemize}

\subsection{Impacto del malware en la seguridad}

El malware puede tener un impacto significativo en la seguridad de la información:

\begin{itemize}
    \item \textbf{Pérdida de confidencialidad}: Robo de información sensible
    \item \textbf{Compromiso de integridad}: Alteración de datos
    \item \textbf{Pérdida de disponibilidad}: Interrupción de servicios
    \item \textbf{Daño a la reputación}: Pérdida de confianza de clientes
    \item \textbf{Costos financieros}: Gastos de remediación y multas
    \item \textbf{Responsabilidad legal}: Consecuencias legales por violaciones
\end{itemize}

\subsection{Tendencias actuales en malware}

\subsubsection{Malware como servicio (MaaS)}

El malware como servicio permite a los atacantes alquilar herramientas maliciosas sin necesidad de desarrollarlas, democratizando el acceso a capacidades de ataque sofisticadas.

\subsubsection{Malware dirigido a dispositivos móviles}

Con la proliferación de dispositivos móviles, el malware móvil ha aumentado significativamente, aprovechando las vulnerabilidades específicas de las plataformas móviles.

\subsubsection{Malware para IoT}

El malware dirigido a dispositivos IoT representa una amenaza creciente, ya que estos dispositivos suelen tener menos medidas de seguridad y pueden ser utilizados para crear botnets masivas.

\subsubsection{Ransomware}

El ransomware ha emergido como una de las amenazas más lucrativas, cifrando archivos y exigiendo pagos para su recuperación.

\section{Ataques de Denial of Service (DoS)}

Los ataques de Denial of Service (DoS) representan una de las amenazas más comunes y efectivas contra la disponibilidad de servicios informáticos. Según la Guía de Manejo de Incidentes de Seguridad Informática del NIST, un ataque DoS se define como:

\textit{"Una acción que previene o deteriora el uso autorizado de redes, sistemas o aplicaciones al agotar recursos como unidades de procesamiento central (CPU), memoria, ancho de banda y espacio en disco."}

\subsection{Definición y características de los ataques DoS}

Un ataque DoS es una forma de ataque contra la disponibilidad de algún servicio. En el contexto de la seguridad informática y de comunicaciones, el enfoque se centra generalmente en servicios de red que son atacados a través de su conexión de red.

\textbf{Características principales}:
\begin{itemize}
    \item \textbf{Objetivo}: Comprometer la disponibilidad de servicios
    \item \textbf{Método}: Agotar recursos críticos del sistema
    \item \textbf{Impacto}: Prevenir o deteriorar el uso autorizado
    \item \textbf{Recursos atacados}: CPU, memoria, ancho de banda, espacio en disco
\end{itemize}

\subsection{Categorías de recursos atacados}

Los ataques DoS pueden dirigirse a diferentes tipos de recursos:

\subsubsection{Ancho de banda de red}

Se relaciona con la capacidad de los enlaces de red que conectan un servidor a Internet. Para la mayoría de las organizaciones, esto es su conexión a su Proveedor de Servicios de Internet (ISP).

\textbf{Características}:
\begin{itemize}
    \item \textbf{Objetivo}: Sobreloadar la capacidad de conexión
    \item \textbf{Método}: Generar tráfico masivo
    \item \textbf{Impacto}: Bloqueo de comunicaciones legítimas
    \item \textbf{Ejemplo}: Flooding attacks con ping o paquetes UDP
\end{itemize}

\subsubsection{Recursos del sistema}

Apunta a sobreloadar o hacer crash el software de manejo de red del sistema objetivo.

\textbf{Características}:
\begin{itemize}
    \item \textbf{Objetivo}: Agotar recursos del sistema operativo
    \item \textbf{Método}: Explotar vulnerabilidades en el código de red
    \item \textbf{Impacto}: Caída del sistema o servicios
    \item \textbf{Ejemplo}: SYN flooding, buffer overflow attacks
\end{itemize}

\subsubsection{Recursos de aplicación}

Se enfoca en agotar recursos específicos de aplicaciones como servidores web, bases de datos o servicios de correo.

\textbf{Características}:
\begin{itemize}
    \item \textbf{Objetivo}: Sobreloadar aplicaciones específicas
    \item \textbf{Método}: Generar solicitudes masivas a la aplicación
    \item \textbf{Impacto}: Indisponibilidad del servicio de aplicación
    \item \textbf{Ejemplo}: HTTP flood, SQL injection masivo
\end{itemize}

\subsection{Ataques DoS clásicos}

\subsubsection{Flooding con ping}

El ataque DoS clásico más simple es un ataque de flooding contra una organización. El objetivo es sobreloadar la capacidad de la conexión de red a la organización objetivo.

\textbf{Características del ataque}:
\begin{itemize}
    \item \textbf{Método}: Envío masivo de paquetes ICMP echo request (ping)
    \item \textbf{Objetivo}: Agotar el ancho de banda de la conexión objetivo
    \item \textbf{Identificación}: La fuente del ataque es claramente identificada a menos que se use una dirección falsificada
    \item \textbf{Impacto}: El rendimiento de la red se ve notablemente afectado
\end{itemize}

\textbf{Proceso del ataque}:
\begin{enumerate}
    \item El atacante identifica un objetivo con conexión de menor capacidad
    \item Genera tráfico masivo desde un sistema con mayor capacidad
    \item Los paquetes se descartan cuando se alcanza la capacidad límite
    \item El tráfico legítimo tiene pocas posibilidades de sobrevivir
\end{enumerate}

\subsubsection{SYN Spoofing}

El SYN spoofing es otro ataque DoS clásico común que ataca la capacidad de un servidor de red para responder a solicitudes de conexión TCP.

\textbf{Características del ataque}:
\begin{itemize}
    \item \textbf{Método}: Desbordar las tablas utilizadas para manejar conexiones TCP
    \item \textbf{Objetivo}: Recursos del sistema, específicamente el código de manejo de red del sistema operativo
    \item \textbf{Impacto}: Los usuarios legítimos son denegados acceso al servidor
    \item \textbf{Proceso}: Envío de múltiples paquetes SYN sin completar el handshake TCP
\end{itemize}

\textbf{Proceso del handshake TCP normal}:
\begin{enumerate}
    \item Cliente envía SYN (seq = x)
    \item Servidor responde con SYN-ACK (seq = y, ack = x+1)
    \item Cliente responde con ACK (ack = y+1)
    \item Conexión establecida
\end{enumerate}

\textbf{Proceso del ataque SYN spoofing}:
\begin{enumerate}
    \item Atacante envía múltiples SYN con direcciones fuente falsificadas
    \item Servidor responde con SYN-ACK a direcciones inexistentes
    \item Servidor mantiene conexiones en estado SYN\_RECEIVED
    \item Se agotan los recursos del servidor
    \item Conexiones legítimas son rechazadas
\end{enumerate}

\subsection{Ataques DoS distribuidos (DDoS)}

Los ataques DDoS representan una evolución significativa de los ataques DoS tradicionales, utilizando múltiples sistemas coordinados para generar tráfico malicioso.

\subsubsection{Características de los ataques DDoS}

\textbf{Ventajas para el atacante}:
\begin{itemize}
    \item \textbf{Mayor volumen}: Múltiples fuentes generan más tráfico
    \item \textbf{Dificultad de detección}: Más difícil identificar la fuente real
    \item \textbf{Resistencia}: Continuación del ataque aunque algunas fuentes sean bloqueadas
    \item \textbf{Escalabilidad}: Fácil ampliación del ataque
\end{itemize}

\textbf{Evolución de la potencia}:
\begin{itemize}
    \item \textbf{2002}: 400 Mbps (modesto)
    \item \textbf{2010}: 100 Gbps
    \item \textbf{2013}: 300 Gbps (ataque Spamhaus)
    \item \textbf{2015}: 600 Gbps (ataque BBC)
    \item \textbf{2016}: 1.2 TBps (ataque Dyn con dispositivos IoT)
\end{itemize}

\subsubsection{Tipos de ataques DDoS}

\textbf{Flooding attacks}:
\begin{itemize}
    \item \textbf{UDP flood}: Envío masivo de paquetes UDP
    \item \textbf{ICMP flood}: Envío masivo de paquetes ICMP
    \item \textbf{HTTP flood}: Solicitudes HTTP masivas
    \item \textbf{SYN flood}: Paquetes SYN masivos
\end{itemize}

\textbf{Amplification attacks}:
\begin{itemize}
    \item \textbf{DNS amplification}: Uso de servidores DNS para amplificar tráfico
    \item \textbf{NTP amplification}: Uso de servidores NTP para amplificar tráfico
    \item \textbf{SNMP amplification}: Uso de dispositivos SNMP para amplificar tráfico
\end{itemize}

\subsection{Ataques DoS con dispositivos IoT}

El ataque DDoS de octubre de 2016 contra Dyn representa una nueva tendencia ominosa en la amenaza. Este ataque duró muchas horas e involucró múltiples oleadas de ataques desde más de 100,000 endpoints maliciosos.

\textbf{Características del ataque Dyn}:
\begin{itemize}
    \item \textbf{Duración}: Muchas horas
    \item \textbf{Fuentes}: Más de 100,000 endpoints maliciosos
    \item \textbf{Dispositivos}: IoT (webcams, monitores de bebés)
    \item \textbf{Volumen}: Pico de hasta 1.2 TBps
    \item \textbf{Impacto}: Interrupción de servicios DNS críticos
\end{itemize}

\textbf{Vulnerabilidades de dispositivos IoT}:
\begin{itemize}
    \item \textbf{Contraseñas por defecto}: Muchos dispositivos mantienen credenciales de fábrica
    \item \textbf{Falta de actualizaciones}: Dispositivos no reciben parches de seguridad
    \item \textbf{Configuración insegura}: Configuraciones por defecto vulnerables
    \item \textbf{Recursos limitados}: Capacidad limitada para implementar seguridad robusta
\end{itemize}

\subsection{Motivaciones de los ataques DoS}

\subsubsection{Extorsión financiera}

Los atacantes pueden exigir pagos para detener el ataque, especialmente contra organizaciones que dependen críticamente de sus servicios en línea.

\subsubsection{Hacktivismo}

Ataques motivados por causas políticas o sociales, como el ataque contra Visa y MasterCard en 2010 por cortar lazos con WikiLeaks.

\subsubsection{Ataques patrocinados por estados}

Gobiernos pueden usar ataques DoS contra oponentes políticos o económicos.

\subsubsection{Diversión}

Los criminales pueden usar ataques DoS contra sistemas bancarios como diversión del ataque real contra sus sistemas de pagos o redes ATM.

\subsection{Medidas de defensa contra ataques DoS}

\subsubsection{Medidas preventivas}

\begin{itemize}
    \item \textbf{Filtrado de paquetes}: Implementar filtros en routers para bloquear tráfico malicioso
    \item \textbf{Rate limiting}: Limitar la velocidad de conexiones por IP
    \item \textbf{Validación de direcciones fuente}: Verificar que las direcciones IP fuente sean válidas
    \item \textbf{Implementación de RFC 2827}: Filtrado de direcciones fuente en bordes de red
    \item \textbf{Configuración segura de dispositivos IoT}: Cambiar contraseñas por defecto y actualizar firmware
\end{itemize}

\subsubsection{Medidas de detección}

\begin{itemize}
    \item \textbf{Monitoreo de tráfico}: Análisis continuo del tráfico de red
    \item \textbf{Detección de anomalías}: Identificación de patrones de tráfico inusuales
    \item \textbf{Análisis de backscatter}: Monitoreo de tráfico de respuesta a direcciones no utilizadas
    \item \textbf{Sistemas IDS/IPS}: Detección y prevención de intrusiones
\end{itemize}

\subsubsection{Medidas de respuesta}

\begin{itemize}
    \item \textbf{Servicios de mitigación DDoS}: Contratar servicios especializados
    \item \textbf{Coordinación con ISPs}: Trabajar con proveedores de internet para bloquear tráfico malicioso
    \item \textbf{Análisis forense}: Investigar la naturaleza y fuente del ataque
    \item \textbf{Plan de respuesta a incidentes}: Procedimientos predefinidos para responder a ataques
\end{itemize}

\subsection{Impacto y consecuencias}

Los ataques DoS pueden tener consecuencias significativas:

\begin{itemize}
    \item \textbf{Pérdida de ingresos}: Interrupción de servicios comerciales
    \item \textbf{Daño a la reputación}: Pérdida de confianza de clientes
    \item \textbf{Costos de remediación}: Gastos en servicios de mitigación y análisis
    \item \textbf{Responsabilidad legal}: Consecuencias legales por interrupción de servicios críticos
    \item \textbf{Impacto en usuarios finales}: Interrupción de servicios esenciales
\end{itemize}

\subsection{Tendencias actuales}

\textbf{Crecimiento en número e intensidad}:
\begin{itemize}
    \item Los ataques DDoS están creciendo en número e intensidad
    \item La mayoría dura 30 minutos o menos
    \item Impulsados por el uso de botnets-for-hire
    \item Ataques más sofisticados y difíciles de mitigar
\end{itemize}

\textbf{Nuevas amenazas}:
\begin{itemize}
    \item \textbf{Dispositivos IoT}: Nueva superficie de ataque masiva
    \item \textbf{5G/6G}: Mayor ancho de banda disponible para ataques
    \item \textbf{IA/ML}: Uso de inteligencia artificial para optimizar ataques
    \item \textbf{Quantum computing}: Futuras amenazas a la criptografía
\end{itemize}
La seguridad perfecta es imposible, pero un enfoque sistemático guiado por modelos puede ayudar a mitigar los riesgos. La tríada CIA, el modelado de la superficie de ataque y el análisis de amenazas son herramientas esenciales para cualquier profesional de la seguridad. La gestión proactiva de la seguridad es clave para proteger los activos en un entorno digital en constante evolución.
    
