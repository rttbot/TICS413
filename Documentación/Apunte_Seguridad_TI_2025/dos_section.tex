El malware se puede definir como software que se ejecuta en un sistema sin el conocimiento o consentimiento del propietario del sistema. Esta definición abarca una amplia gama de programas maliciosos que pueden tener diferentes objetivos y métodos de operación.

\textbf{Características comunes del malware}:
\begin{itemize}
    \item \textbf{Ejecución no autorizada}: Se ejecuta sin el conocimiento del usuario
    \item \textbf{Objetivos maliciosos}: Diseñado para causar daño o beneficio ilícito
    \item \textbf{Ocultación}: Intenta evadir la detección por sistemas de seguridad
    \item \textbf{Propagación}: Se replica y se extiende a otros sistemas
    \item \textbf{Persistencia}: Mantiene su presencia en el sistema infectado
\end{itemize}

\subsection{Tipos principales de malware}

\subsubsection{Advanced Persistent Threats (APTs)}

Las APTs han surgido como una de las amenazas más sofisticadas en los últimos años. No son un nuevo tipo de malware, sino la aplicación bien financiada y persistente de una amplia variedad de tecnologías de intrusión y malware a objetivos seleccionados, generalmente empresariales o políticos.

\textbf{Características de las APTs}:
\begin{itemize}
    \item \textbf{Advanced}: Uso de una amplia variedad de tecnologías de intrusión y malware, incluyendo el desarrollo de malware personalizado si es necesario
    \item \textbf{Persistent}: Aplicación determinada de los ataques durante un período extendido contra el objetivo elegido para maximizar las posibilidades de éxito
    \item \textbf{Threats}: Amenazas a los objetivos seleccionados como resultado de atacantes organizados, capaces y bien financiados
\end{itemize}

\textbf{Técnicas utilizadas por las APTs}:
\begin{itemize}
    \item \textbf{Social engineering}: Manipulación psicológica de usuarios
    \item \textbf{Spear-phishing}: Correos electrónicos dirigidos específicamente
    \item \textbf{Drive-by-downloads}: Descargas automáticas desde sitios web comprometidos
    \item \textbf{Malware sofisticado}: Con múltiples mecanismos de propagación y payloads
\end{itemize}

\textbf{Ejemplos históricos}: Aurora, RSA, APT1, Stuxnet

\subsubsection{Virus}

Un virus es una pieza de software que puede "infectar" otros programas, o cualquier tipo de contenido ejecutable, modificándolos. La modificación incluye inyectar el código original con una rutina para hacer copias del código del virus, que luego puede continuar infectando otro contenido.

\textbf{Características de los virus}:
\begin{itemize}
    \item \textbf{Infección}: Se adjunta a archivos ejecutables legítimos
    \item \textbf{Replicación}: Se copia a sí mismo en otros archivos
    \item \textbf{Propagación}: Se extiende de computadora a computadora
    \item \textbf{Ejecución secreta}: Se ejecuta cuando el programa huésped se ejecuta
    \item \textbf{Especificidad}: Específico para sistema operativo y hardware
\end{itemize}

\textbf{Historia}: Los virus informáticos aparecieron por primera vez a principios de la década de 1980, y el término se atribuye a Fred Cohen. El virus Brain, visto por primera vez en 1986, fue uno de los primeros en apuntar a sistemas MSDOS.

\subsubsection{Adware}

El adware es software que muestra anuncios no deseados al usuario. Aunque no siempre es malicioso por naturaleza, puede ser molesto y puede recopilar información del usuario sin su consentimiento.

\textbf{Características del adware}:
\begin{itemize}
    \item \textbf{Anuncios intrusivos}: Muestra publicidad no solicitada
    \item \textbf{Recopilación de datos}: Puede rastrear hábitos de navegación
    \item \textbf{Reducción del rendimiento}: Puede ralentizar el sistema
    \item \textbf{Cambios en el navegador}: Puede modificar configuraciones
\end{itemize}

\subsubsection{Attack Kit}

Los attack kits son herramientas que permiten a los atacantes crear malware personalizado sin necesidad de conocimientos técnicos avanzados. Estos kits proporcionan interfaces gráficas y funcionalidades predefinidas para crear malware.

\textbf{Características de los attack kits}:
\begin{itemize}
    \item \textbf{Facilidad de uso}: Interfaces gráficas intuitivas
    \item \textbf{Personalización}: Opciones para personalizar el malware
    \item \textbf{Actualizaciones}: Reciben actualizaciones regulares
    \item \textbf{Distribución}: Se venden en mercados clandestinos
\end{itemize}

\subsubsection{Auto-rooter}

Un auto-rooter es una herramienta que automatiza el proceso de obtención de acceso root (administrador) en sistemas vulnerables. Estas herramientas escanean redes en busca de sistemas vulnerables y explotan automáticamente las vulnerabilidades encontradas.

\textbf{Características de los auto-rooters}:
\begin{itemize}
    \item \textbf{Escaneo automático}: Busca sistemas vulnerables
    \item \textbf{Explotación automática}: Explota vulnerabilidades sin intervención manual
    \item \textbf{Acceso privilegiado}: Obtiene acceso de administrador
    \item \textbf{Propagación}: Puede infectar múltiples sistemas
\end{itemize}

\subsubsection{Backdoor}

Un backdoor es un método secreto para eludir la autenticación normal o el cifrado en un sistema. Los backdoors pueden ser instalados por malware o pueden ser creados intencionalmente por desarrolladores para acceso de mantenimiento.

\textbf{Características de los backdoors}:
\begin{itemize}
    \item \textbf{Acceso oculto}: Proporciona acceso no autorizado al sistema
    \item \textbf{Evasión de seguridad}: Evita mecanismos de autenticación normales
    \item \textbf{Control remoto}: Permite control del sistema desde el exterior
    \item \textbf{Persistencia}: Mantiene el acceso incluso después de reinicios
\end{itemize}

\subsubsection{Downloaders}

Los downloaders son programas que descargan e instalan otros programas maliciosos en el sistema. Actúan como el primer paso en una cadena de infección, descargando malware más sofisticado.

\textbf{Características de los downloaders}:
\begin{itemize}
    \item \textbf{Descarga automática}: Descarga malware sin intervención del usuario
    \item \textbf{Instalación silenciosa}: Instala malware sin notificación
    \item \textbf{Actualizaciones}: Puede descargar versiones actualizadas del malware
    \item \textbf{Evasión}: Intenta evadir la detección por antivirus
\end{itemize}

\subsection{Métodos de propagación del malware}

\subsubsection{Propagación por red}

El malware puede propagarse a través de redes utilizando varios métodos:

\begin{itemize}
    \item \textbf{Vulnerabilidades de red}: Explota fallos en protocolos o servicios
    \item \textbf{Compartir archivos}: Se propaga a través de recursos compartidos
    \item \textbf{Correo electrónico}: Se adjunta a mensajes de correo
    \item \textbf{Sitios web maliciosos}: Se descarga desde sitios comprometidos
    \item \textbf{Dispositivos móviles}: Se propaga a través de conexiones Bluetooth o WiFi
\end{itemize}

\subsubsection{Propagación por medios físicos}

El malware también puede propagarse a través de medios físicos:

\begin{itemize}
    \item \textbf{Dispositivos USB}: Se ejecuta automáticamente al conectar
    \item \textbf{CDs/DVDs}: Se incluye en medios ópticos
    \item \textbf{Dispositivos móviles}: Se propaga a través de conexiones físicas
    \item \textbf{Tarjetas de memoria}: Se almacena en medios extraíbles
\end{itemize}

\subsection{Detección y prevención del malware}

\subsubsection{Métodos de detección}

\begin{itemize}
    \item \textbf{Detección basada en firmas}: Identifica malware conocido
    \item \textbf{Detección basada en comportamiento}: Identifica actividad sospechosa
    \item \textbf{Detección basada en heurísticas}: Usa reglas para identificar malware desconocido
    \item \textbf{Análisis de sandbox}: Ejecuta archivos en un entorno aislado
    \item \textbf{Análisis de tráfico de red}: Monitorea comunicaciones maliciosas
\end{itemize}

\subsubsection{Medidas de prevención}

\begin{itemize}
    \item \textbf{Software antivirus actualizado}: Mantener protección al día
    \item \textbf{Parches de seguridad}: Actualizar sistemas regularmente
    \item \textbf{Educación del usuario}: Capacitar sobre amenazas
    \item \textbf{Políticas de seguridad}: Establecer reglas claras
    \item \textbf{Backups regulares}: Mantener copias de seguridad
    \item \textbf{Segmentación de red}: Aislar sistemas críticos
\end{itemize}

\subsection{Impacto del malware en la seguridad}

El malware puede tener un impacto significativo en la seguridad de la información:

\begin{itemize}
    \item \textbf{Pérdida de confidencialidad}: Robo de información sensible
    \item \textbf{Compromiso de integridad}: Alteración de datos
    \item \textbf{Pérdida de disponibilidad}: Interrupción de servicios
    \item \textbf{Daño a la reputación}: Pérdida de confianza de clientes
    \item \textbf{Costos financieros}: Gastos de remediación y multas
    \item \textbf{Responsabilidad legal}: Consecuencias legales por violaciones
\end{itemize}

\subsection{Tendencias actuales en malware}

\subsubsection{Malware como servicio (MaaS)}

El malware como servicio permite a los atacantes alquilar herramientas maliciosas sin necesidad de desarrollarlas, democratizando el acceso a capacidades de ataque sofisticadas.

\subsubsection{Malware dirigido a dispositivos móviles}

Con la proliferación de dispositivos móviles, el malware móvil ha aumentado significativamente, aprovechando las vulnerabilidades específicas de las plataformas móviles.

\subsubsection{Malware para IoT}

El malware dirigido a dispositivos IoT representa una amenaza creciente, ya que estos dispositivos suelen tener menos medidas de seguridad y pueden ser utilizados para crear botnets masivas.

\subsubsection{Ransomware}

El ransomware ha emergido como una de las amenazas más lucrativas, cifrando archivos y exigiendo pagos para su recuperación.

\section{Ataques de Denial of Service (DoS)}

Los ataques de Denial of Service (DoS) representan una de las amenazas más comunes y efectivas contra la disponibilidad de servicios informáticos. Según la Guía de Manejo de Incidentes de Seguridad Informática del NIST, un ataque DoS se define como:

\textit{"Una acción que previene o deteriora el uso autorizado de redes, sistemas o aplicaciones al agotar recursos como unidades de procesamiento central (CPU), memoria, ancho de banda y espacio en disco."}

\subsection{Definición y características de los ataques DoS}

Un ataque DoS es una forma de ataque contra la disponibilidad de algún servicio. En el contexto de la seguridad informática y de comunicaciones, el enfoque se centra generalmente en servicios de red que son atacados a través de su conexión de red.

\textbf{Características principales}:
\begin{itemize}
    \item \textbf{Objetivo}: Comprometer la disponibilidad de servicios
    \item \textbf{Método}: Agotar recursos críticos del sistema
    \item \textbf{Impacto}: Prevenir o deteriorar el uso autorizado
    \item \textbf{Recursos atacados}: CPU, memoria, ancho de banda, espacio en disco
\end{itemize}

\subsection{Categorías de recursos atacados}

Los ataques DoS pueden dirigirse a diferentes tipos de recursos:

\subsubsection{Ancho de banda de red}

Se relaciona con la capacidad de los enlaces de red que conectan un servidor a Internet. Para la mayoría de las organizaciones, esto es su conexión a su Proveedor de Servicios de Internet (ISP).

\textbf{Características}:
\begin{itemize}
    \item \textbf{Objetivo}: Sobreloadar la capacidad de conexión
    \item \textbf{Método}: Generar tráfico masivo
    \item \textbf{Impacto}: Bloqueo de comunicaciones legítimas
    \item \textbf{Ejemplo}: Flooding attacks con ping o paquetes UDP
\end{itemize}

\subsubsection{Recursos del sistema}

Apunta a sobreloadar o hacer crash el software de manejo de red del sistema objetivo.

\textbf{Características}:
\begin{itemize}
    \item \textbf{Objetivo}: Agotar recursos del sistema operativo
    \item \textbf{Método}: Explotar vulnerabilidades en el código de red
    \item \textbf{Impacto}: Caída del sistema o servicios
    \item \textbf{Ejemplo}: SYN flooding, buffer overflow attacks
\end{itemize}

\subsubsection{Recursos de aplicación}

Se enfoca en agotar recursos específicos de aplicaciones como servidores web, bases de datos o servicios de correo.

\textbf{Características}:
\begin{itemize}
    \item \textbf{Objetivo}: Sobreloadar aplicaciones específicas
    \item \textbf{Método}: Generar solicitudes masivas a la aplicación
    \item \textbf{Impacto}: Indisponibilidad del servicio de aplicación
    \item \textbf{Ejemplo}: HTTP flood, SQL injection masivo
\end{itemize}

\subsection{Ataques DoS clásicos}

\subsubsection{Flooding con ping}
