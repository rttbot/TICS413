\chapterimage{ampolletas.png}
\chapter{Carta de Riesgos}

\section{Introducción a la Actividad}

La carta de riesgos es una herramienta fundamental para identificar, evaluar y gestionar los riesgos asociados con la integración de SecureShop en las sucursales de Yumbo. Esta actividad se enmarca dentro de las normativas \textbf{ISO/IEC 27001:2022}, la \textbf{Ley N° 21.663 de Ciberseguridad} y la \textbf{Ley N° 19.628 sobre Protección de la Vida Privada}.

\subsection{Objetivo de la Actividad}
El objetivo es que los estudiantes trabajen de manera colaborativa para identificar riesgos, proponer controles y evaluar el impacto de no actuar sobre los riesgos no mitigados. Utilizarán el apunte y los mazos de cartas \textbf{Riskio} para guiar su análisis.

\subsection{División de Grupos y Asignación de Tareas}

\begin{itemize}
  \item \textbf{Grupo 1: Identificación de Riesgos}
    \begin{itemize}
      \item Mazos de cartas: Riesgos de Seguridad y Vulnerabilidades
      \item Tiempo asignado: 30 minutos
      \item Tareas: Describir los riesgos identificados, asignar probabilidad e impacto.
    \end{itemize}
  \item \textbf{Grupo 2: Implementación de Controles}
    \begin{itemize}
      \item Mazos de cartas: Controles de Seguridad y Normativas
      \item Tiempo asignado: 30 minutos
      \item Tareas: Proponer controles, justificar su implementación y relacionarlos con normativas.
    \end{itemize}
  \item \textbf{Grupo 3: Evaluación de Impacto}
    \begin{itemize}
      \item Mazos de cartas: Impacto y Consecuencias
      \item Tiempo asignado: 30 minutos
      \item Tareas: Evaluar el impacto potencial y justificar la aceptación del riesgo.
    \end{itemize}
\end{itemize}

\subsection{Documentación y Presentación}

Cada grupo debe documentar sus hallazgos en un documento compartido, asegurándose de completar su sección de manera detallada y clara. Al final de la actividad,subirán el documento compartido al foro.

\subsection{Motivación y Recursos}

\begin{itemize}
  \item Se motiva a los estudiantes a leer el apunte para obtener ideas y utilizar los mazos de cartas \textbf{Riskio} para guiar su análisis.
  \item Ejemplo de carta Riskio: "Información - No hay seguridad física en la sala de reuniones del cliente."
\end{itemize}

\section{Plantilla para la Carta de Riesgos}

\begin{itemize}
  \item \textbf{Identificación del Riesgo}
    \begin{itemize}
      \item Descripción detallada del riesgo:
      \item Identificador único:
    \end{itemize}
  \item \textbf{Evaluación del Riesgo}
    \begin{itemize}
      \item Probabilidad:
      \item Impacto:
      \item Justificación de la evaluación:
    \end{itemize}
  \item \textbf{Controles Existentes}
    \begin{itemize}
      \item Descripción de los controles actuales:
      \item Efectividad de los controles:
    \end{itemize}
  \item \textbf{Evaluación del Riesgo Residual}
    \begin{itemize}
      \item Riesgo después de aplicar controles:
      \item Justificación del riesgo residual:
    \end{itemize}
  \item \textbf{Recomendaciones y Mejores Prácticas}
    \begin{itemize}
      \item Propuestas para mitigar el riesgo:
      \item Relación con normativas:
    \end{itemize}
  \item \textbf{Plan de Acción}
    \begin{itemize}
      \item Pasos específicos para implementar recomendaciones:
      \item Responsables y plazos:
    \end{itemize}
  \item \textbf{Monitoreo y Revisión}
    \begin{itemize}
      \item Estrategias para monitorear el riesgo:
      \item Frecuencia de revisión:
    \end{itemize}
  \item \textbf{Documentación y Comunicación}
    \begin{itemize}
      \item Cómo se documentará y comunicará el riesgo:
      \item Audiencia objetivo:
    \end{itemize}
  \item \textbf{Aprobación}
    \begin{itemize}
      \item Firma y fecha de aprobación:
      \item Autoridad responsable:
    \end{itemize}
\end{itemize}

\section{Ejemplo de Carta de Riesgos}

\textbf{Contexto:} Proyecto de investigación de una vacuna para el COVID-19 en un recinto universitario, donde se realizan secuenciamientos de DNA y se maneja información sensible de pacientes.

\begin{itemize}
  \item \textbf{Identificación del Riesgo}
    \begin{itemize}
      \item Descripción detallada del riesgo: Acceso no autorizado a datos sensibles de pacientes.
      \item Identificador único: R002
    \end{itemize}
  \item \textbf{Evaluación del Riesgo}
    \begin{itemize}
      \item Probabilidad: Alta
      \item Impacto: Muy Alto
      \item Justificación de la evaluación: La información incluye datos personales y resultados de secuenciamiento de DNA.
    \end{itemize}
  \item \textbf{Controles Existentes}
    \begin{itemize}
      \item Descripción de los controles actuales: Acceso restringido a laboratorios y sistemas de gestión de datos.
      \item Efectividad de los controles: Alta
    \end{itemize}
  \item \textbf{Evaluación del Riesgo Residual}
    \begin{itemize}
      \item Riesgo después de aplicar controles: Medio
      \item Justificación del riesgo residual: Posibilidad de brechas de seguridad internas.
    \end{itemize}
  \item \textbf{Recomendaciones y Mejores Prácticas}
    \begin{itemize}
      \item Propuestas para mitigar el riesgo: Implementar cifrado de datos y auditorías de seguridad regulares.
      \item Relación con normativas: Cumplimiento con la Ley N° 19.628 sobre Protección de la Vida Privada
    \end{itemize}
  \item \textbf{Plan de Acción}
    \begin{itemize}
      \item Pasos específicos para implementar recomendaciones: Desplegar soluciones de cifrado y establecer un calendario de auditorías.
      \item Responsables y plazos: Equipo de Seguridad de TI, 6 meses.
    \end{itemize}
  \item \textbf{Monitoreo y Revisión}
    \begin{itemize}
      \item Estrategias para monitorear el riesgo: Revisiones mensuales de seguridad.
      \item Frecuencia de revisión: Mensual
    \end{itemize}
  \item \textbf{Documentación y Comunicación}
    \begin{itemize}
      \item Cómo se documentará y comunicará el riesgo: Informes de seguridad trimestrales.
      \item Audiencia objetivo: Comité de Ética y Seguridad
    \end{itemize}
  \item \textbf{Aprobación}
    \begin{itemize}
      \item Firma y fecha de aprobación: Director de Investigación, 01/05/2023
      \item Autoridad responsable: Director de Investigación
    \end{itemize}
\end{itemize}
