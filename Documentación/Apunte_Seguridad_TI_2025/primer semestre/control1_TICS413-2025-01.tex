\documentclass[a4paper]{article}
\usepackage[a4paper, left=2.5cm, right=2.5cm, top=2.5cm, bottom=2.5cm]{geometry}

\usepackage{tikz}

\newcommand{\dottedlines}[1]{%
  \foreach \i in {1,...,#1} {%
    \noindent
    \begin{tikzpicture}
      \draw[dotted] (0,0) -- (15,0);
    \end{tikzpicture}\\[0.4cm]
  }%
}

\usepackage{pgffor} 

\usepackage[utf8]{inputenc}
\usepackage{graphicx}
\usepackage{color}
\usepackage{longtable}  % Para manejar tablas largas si es necesario
\usepackage{titlesec}   % Para personalizar los títulos
\usepackage{tikz}
\usetikzlibrary{shapes.geometric, arrows}

\titleformat{\section}[block]{\normalfont\Large\bfseries}{\thesection}{1em}{}

\begin{document}
\begin{center}
    \vspace*{-2cm}
    \includegraphics[width=1\textwidth]{Pictures/UAI_35.png} \
    %\textbf{FACULTAD DE INGENIERÍA Y CIENCIAS} \\
    %\textbf{UNIVERSIDAD ADOLFO IBÁÑEZ} \\
    \vspace{0.5cm} % Espacio vertical para separar la imagen
    Control\# 1 \\
    TICS413 Seguridad TI \\
    Profesores: Romina Torres \\
    Fecha: 14/04/2025
\end{center}

\vspace{0.6cm}

\begin{flushleft}
    \makebox[16cm]{\textbf{Nombre}:\ \hrulefill} \\
    \medskip
    \makebox[16cm]{\textbf{Rut}:\ \hrulefill}
\end{flushleft}
Esta prueba contiene X preguntas de X puntos cada una,  totalizando X puntos. Recuerde responder con letra clara y legible.


%\vspace{1cm} 
\textbf{Código de honor (extracto):} Constituyen infracciones gravísimas al deber de honestidad hacer pasar como propia, exclusivamente o con otro, una obra ajena o parte de ella para obtener una evaluación favorable. Emplear o facilitar a otro estudiante información de un modo prohibido por las reglas o instrucciones aplicables a esa actividad o que sea incompatible con el sentido de esa actividad.

Por favor \textbf{NO comenzar}, hasta que se lo indiquen.


\section*{Selección múltiple - 30 puntos (60 minutos)}

\begin{enumerate}
    \item ¿Qué tipo de ataque utiliza mensajes fraudulentos para engañar a las víctimas y obtener información confidencial? 
    \begin{itemize}
        \item a. Phishing
        \item b. Malware
        %\item c. Cross-Site Scripting (XSS)
        \item d. Denial of Service (DoS)
        \item e. Man-in-the-middle
    \end{itemize}
    %\textcolor{white}{Respuesta: a. Phishing. Justificación: El phishing es un ataque que utiliza mensajes fraudulentos para engañar a las víctimas y obtener información confidencial. Las otras opciones no se centran en el engaño a través de mensajes.}
    %\vspace{1cm} % Espacio para que los estudiantes justifiquen su respuesta
    
    \item ¿Cuál es una buena práctica para proteger un servidor web contra ataques comunes? 
    \begin{itemize}
        \item a. Utilizar un Web Application Firewall (WAF)
        \item b. Permitir el tráfico entrante únicamente por el puerto 80
        \item c. Configurar únicamente acceso SSH
       % \item d. Verificar los campos antes de enviarlos a la API
        \item e. Evitar almacenar contraseñas en texto plano
    \end{itemize}
    %\textcolor{white}{Respuesta: a. Utilizar un Web Application Firewall (WAF) y d. Verificar los campos antes de enviarlos a la API. Justificación: Un WAF ayuda a proteger contra ataques web como inyección SQL y XSS, y verificar los campos es crucial para evitar inyecciones.}
    %\vspace{1cm}
    
    \item ¿Qué describe mejor el propósito de un ataque de fuerza bruta? 
    \begin{itemize}
        \item a. Obtener acceso a un sistema probando todas las combinaciones posibles de contraseñas
        \item b. Interceptar datos entre el cliente y el servidor
        \item c. Saturar un servidor para hacerlo inaccesible
        \item d. Modificar datos en una base de datos
    \end{itemize}
    %\textcolor{white}{Respuesta: a. Obtener acceso a un sistema probando todas las combinaciones posibles de contraseñas. Justificación: Un ataque de fuerza bruta intenta todas las combinaciones posibles para acceder a un sistema. Las otras opciones no describen este método.}
    %\vspace{1cm}
    
    \item ¿Qué tipo de malware bloquea el acceso a los datos del usuario y exige un rescate para liberarlos? 
    \begin{itemize}
        \item a. Ransomware
        \item b. Spyware
        \item c. Worm
        \item d. Keylogger
    \end{itemize}
    %\textcolor{white}{Respuesta: a. Ransomware. Justificación: El ransomware cifra los datos del usuario y exige un rescate para liberarlos. Los otros tipos de malware no tienen este comportamiento.}
    %\vspace{1cm}
    
    % \item ¿Cuál de las siguientes vulnerabilidades aprovecha la especulación en la ejecución de instrucciones del procesador para acceder a datos sensibles? Seleccione todas las correctas.
    % \begin{itemize}
    %     \item a. Spectre
    %     \item b. SQL Injection
    %     \item c. Phishing
    %     \item d. Meltdown
    % \end{itemize}
    % %\textcolor{white}{Respuesta: a. Spectre y d. Meltdown. Justificación: Ambas vulnerabilidades explotan la especulación en la ejecución de instrucciones del procesador. Las otras opciones no están relacionadas con este tipo de vulnerabilidad.}
    % %\vspace{1cm}
    
    % \item ¿Qué ataque implica modificar las respuestas del DNS para redirigir a las víctimas a sitios fraudulentos? Seleccione todas las correctas.
    % \begin{itemize}
    %     \item a. Spoofing
    %     \item b. Ataque de envenenamiento de caché DNS
    %     \item c. Cross-Site Scripting (XSS)
    %     \item d. Denial of Service (DoS)
    % \end{itemize}
    % %\textcolor{white}{Respuesta: b. Ataque de envenenamiento de caché DNS. Justificación: Este ataque modifica las respuestas del DNS para redirigir a las víctimas. Las otras opciones no se centran en el DNS.}
    % %\vspace{1cm}
    
    \item ¿Qué tipo de ataque podría mitigarse mejor con la implementación de un límite de tasa (Rate Limiting) o reintentos? 
    \begin{itemize}
        \item a. Fuerza bruta
        \item b. Ransomware
        %\item c. Inyección SQL
        %\item d. Cross-Site Scripting
    \end{itemize}
    %\textcolor{white}{Respuesta: a. Fuerza bruta. Justificación: El límite de tasa puede reducir la efectividad de un ataque de fuerza bruta al limitar el número de intentos en un período de tiempo.}
    %\vspace{1cm}
    
    \item ¿Qué herramientas son comunes para detectar intrusiones en una red? Seleccione todas las correctas.
    \begin{itemize}
        \item a. IDS (Sistema de detección de intrusiones)
        \item b. VPN
        \item c. Firewall
        %\item d. SSH
    \end{itemize}
    %\textcolor{white}{Respuesta: a. IDS (Sistema de detección de intrusiones). Justificación: Un IDS está diseñado específicamente para detectar intrusiones en una red. Las otras opciones no tienen esta función principal.}
    %\vspace{1cm}
    
    \item Un atacante ha accedido a una red y está interceptando comunicaciones entre dos dispositivos para robar credenciales. ¿Qué técnica está utilizando? %Seleccione todas las correctas.
    \begin{itemize}
        \item a. Man-in-the-Middle (MitM)
        \item b. Phishing
        \item c. DoS (Denial of Service)
        %\item d. SQL Injection
    \end{itemize}
    %\textcolor{white}{Respuesta: a. Man-in-the-Middle (MitM). Justificación: Un ataque MitM intercepta y potencialmente altera la comunicación entre dos partes. Las otras opciones no describen este tipo de ataque.}
    %\vspace{1cm}
    
    \item Un administrador de red configura un servidor web. ¿Qué combinación de configuraciones debe implementar para garantizar que solo se permita tráfico HTTPS hacia el servidor? Seleccione todas las correctas.
    \begin{itemize}
        \item I. Permitir solo el puerto 443 en el firewall.
        \item II. Configurar un certificado SSL/TLS válido.
        \item III. Deshabilitar todos los puertos excepto el 22.
        \item IV. Implementar autenticación de dos factores (2FA) para los usuarios del sitio.
    \end{itemize}
    \begin{itemize}
        \item A. Solo I y II
        \item B. Solo II y III
        \item C. Solo I, II y IV
        \item D. Solo III y IV
        \item E. Otra combinación (ninguna de las anteriores)
    \end{itemize}
    %\textcolor{white}{Respuesta: A. Solo I y II. Justificación: Permitir solo el puerto 443 y configurar un certificado SSL/TLS válido asegura que solo el tráfico HTTPS sea permitido.}
    %\vspace{1cm}

    \item ¿Qué tipo de ataque implica engañar a una persona para que revele información confidencial?
    \begin{itemize}
        \item a. Phishing
        \item b. DoS (Denial of Service)
        \item c. DDoS
        \item d. Man-in-the-Middle
        %\item e. SQL Injection
    \end{itemize}
    %\textcolor{white}{Respuesta: a. Phishing. Justificación: El phishing engaña a las personas para que revelen información confidencial. Las otras opciones no se centran en el engaño directo a personas.}
    %\vspace{1cm}

    \item ¿Cuál es el propósito principal de un ataque de ransomware?
    \begin{itemize}
        \item a. Interrumpir el servicio
        \item b. Secuestrar datos a cambio de un rescate
        \item c. Obtener acceso no autorizado
        \item d. Destruir los datos del sistema
    \end{itemize}
    %\textcolor{white}{Respuesta: b. Secuestrar datos a cambio de un rescate. Justificación: El ransomware cifra los datos y exige un rescate para liberarlos. Las otras opciones no describen el propósito principal del ransomware.}
    %\vspace{1cm}

    \item ¿Qué describe mejor un ataque de "Denial of Service" (DoS)?
    \begin{itemize}
        \item a. Saturar un servidor con tráfico para interrumpir su funcionamiento
        \item b. Interceptar comunicaciones entre dos partes
        \item c. Explorar vulnerabilidades de software
        \item d. Robar información de cuentas bancarias
    \end{itemize}
    %\textcolor{white}{Respuesta: a. Saturar un servidor con tráfico para interrumpir su funcionamiento. Justificación: Un ataque DoS busca agotar los recursos de un sistema para interrumpir su funcionamiento.}
    %\vspace{1cm}

    % \item ¿Cuál de los siguientes ataques explota vulnerabilidades en el navegador para ejecutar código malicioso en el lado del cliente?
    % \begin{itemize}
    %     \item a. Cross-Site Scripting (XSS)
    %     \item b. Ransomware
    %     \item c. SQL Injection
    %     \item d. Man-in-the-Middle
    % \end{itemize}
    % %\textcolor{white}{Respuesta: a. Cross-Site Scripting (XSS). Justificación: XSS explota vulnerabilidades en el navegador para ejecutar código malicioso en el lado del cliente.}
    % %\vspace{1cm}

    \item ¿Qué ataque utiliza la técnica de fuerza bruta para descifrar contraseñas mediante la prueba de múltiples combinaciones?
    \begin{itemize}
        \item a. Phishing
        %\item b. SQL Injection
        \item c. Ataque de Fuerza Bruta
        \item d. Man-in-the-Middle
    \end{itemize}
    %\textcolor{white}{Respuesta: c. Ataque de Fuerza Bruta. Justificación: Un ataque de fuerza bruta prueba múltiples combinaciones de contraseñas para descifrarlas.}
    %\vspace{1cm}
    %
    % \item ¿Qué significa un ataque de "Zero Day"?
    % \begin{itemize}
    %     \item a. Un ataque descubierto el mismo día que ocurre
    %     \item b. Un ataque que explota una vulnerabilidad antes de que se publique un parche
    %     \item c. Un ataque que tiene lugar el primer día del mes
    %     \item d. Un ataque que se dirige a software de código abierto
    % \end{itemize}
    %\textcolor{white}{Respuesta: b. Un ataque que explota una vulnerabilidad antes de que se publique un parche. Justificación: Un ataque de "Zero Day" explota vulnerabilidades desconocidas antes de que se publiquen parches.}
    %\vspace{1cm}
    %
    % \item ¿Cuál es el objetivo principal de un ataque de ingeniería social?
    % \begin{itemize}
    %     \item a. Manipular a las personas para que revelen información confidencial
    %     \item b. Obtener acceso físico a un sistema
    %     \item c. Infectar con malware
    %     \item d. Desencriptar datos cifrados
    % \end{itemize}
    % %\textcolor{white}{Respuesta: a. Manipular a las personas para que revelen información confidencial. Justificación: La ingeniería social se centra en manipular a las personas para obtener información.}
    % %\vspace{1cm}

    \item ¿Qué tipo de ataque consiste en infectar múltiples dispositivos para crear una red controlada por el atacante, conocida como botnet?
    \begin{itemize}
        \item a. Phishing
        \item b. Ataque de Botnet
        %\item c. Cross-Site Scripting
        \item d. Man-in-the-Middle
    \end{itemize}
    %\textcolor{white}{Respuesta: b. Ataque de Botnet. Justificación: Un ataque de botnet infecta múltiples dispositivos para crear una red controlada por el atacante.}
    %\vspace{1cm}

    % \item ¿Cuál de los siguientes es un ataque que afecta la disponibilidad de un sistema en la tríada CIA?
    % \begin{itemize}
    %     \item a. Ataque de denegación de servicio (DoS)
    %     %\item b. Inyección SQL
    %     %\item c. Cross-Site Scripting (XSS)
    %     \item d. Phishing
    % \end{itemize}
    % %\textcolor{white}{Respuesta: a. Ataque de denegación de servicio (DoS). Justificación: Un ataque DoS afecta la disponibilidad de un sistema al agotar sus recursos.}
    % %\vspace{1cm}

    \item ¿Cuál de los siguientes es un ataque que afecta la integridad de un sistema en la tríada CIA?
    \begin{itemize}
        \item a. Phishing para robar credenciales
        \item b. Ataque de denegación de servicio (DoS)
        \item c. Sniffing de contraseñas en tránsito
        \item d. Ninguna de las anteriores %Alteración de datos mediante un ataque de inyección SQL
    \end{itemize}
    %\textcolor{white}{Respuesta: a. Alteración de datos mediante un ataque de inyección SQL. Justificación: La inyección SQL puede alterar datos, afectando la integridad del sistema. - cambia a ninguna de las anteriores}
    %\vspace{1cm}

    \item ¿Cuáles de las siguientes son buenas prácticas para proteger la integridad y confidencialidad de los datos en tránsito?
    \begin{itemize}
        \item I. Utilizar protocolos de cifrado como TLS (Transport Layer Security) para proteger la comunicación.
        \item II. Enviar datos sensibles en texto plano a través de la red para mejorar el rendimiento.
        \item III. Implementar firmas digitales para verificar la autenticidad e integridad de los datos transmitidos.
        \item IV. Desactivar el cifrado en redes privadas para facilitar el monitoreo del tráfico.
        %\item V. Configurar VPNs (Redes Privadas Virtuales) para crear un túnel seguro entre redes externas e internas.
    \end{itemize}
    \begin{itemize}
        \item a. Solo I y III
        \item b. Solo I, III 
        \item c. Solo II, IV 
        \item d. Solo I, II y IV
        \item e. Solo III 
    \end{itemize}
    %\textcolor{white}{Respuesta: b. Solo I, III . Justificación: Utilizar TLS, implementar firmas digitales y configurar VPNs son buenas prácticas para proteger datos en tránsito.}
    %\vspace{1cm}
    %
    % \item Análisis de Autenticación y Autorización: 
    % Situación:
    % \begin{itemize}
    %     \item Todos los empleados utilizan contraseñas simples para sus cuentas.
    %     \item No hay autenticación de dos factores (2FA) habilitada en las cuentas críticas.
    % \end{itemize}
    % ¿Cuál es el impacto de una posible fuga de contraseñas?
    % \begin{itemize}
    %     \item a. No hay impacto significativo en la seguridad.
    %     \item b. Riesgo de acceso no autorizado a cuentas críticas.
    %    % \item c. Mejora la seguridad al cambiar las contraseñas regularmente.
    %     \item d. Las contraseñas no son relevantes para la seguridad.
    % \end{itemize}
    % %\textcolor{white}{Respuesta: b. Riesgo de acceso no autorizado a cuentas críticas. Justificación: Las contraseñas simples aumentan el riesgo de acceso no autorizado.}
    % %\vspace{1cm}
    %
    % ¿Qué políticas de autenticación más seguras se podrían implementar?
    % \begin{itemize}
    %     \item a. Implementar autenticación de dos factores (2FA).
    %     \item b. Usar contraseñas simples para facilitar el acceso.
    %     \item c. No cambiar las políticas actuales.
    %     \item d. Desactivar la autenticación para simplificar el acceso.
    % \end{itemize}
    % %\textcolor{white}{Respuesta: a. Implementar autenticación de dos factores (2FA). Justificación: La implementación de 2FA mejora la seguridad al requerir un segundo factor de autenticación.}
    % %\vspace{1cm}
    %
    % \item Explica el concepto de ciberseguridad. 
    % ¿Por qué es importante para individuos y organizaciones proteger su información digital?
    % \begin{itemize}
    %     \item a. No es importante proteger la información digital.
    %     \item b. Protege contra el acceso no autorizado y pérdida de datos.
    %     \item c. Solo las grandes empresas necesitan preocuparse por la ciberseguridad.
    %     \item d. La ciberseguridad es solo una moda pasajera.
    % \end{itemize}
    %\textcolor{white}{Respuesta: b. Protege contra el acceso no autorizado y pérdida de datos. Justificación: La ciberseguridad es crucial para proteger la información digital de accesos no autorizados y pérdidas.}
    %\vspace{1cm}
    %
    % \item Describe qué es un ataque de phishing. 
    % Proporciona un ejemplo práctico y explica cómo alguien puede identificar y evitar caer en este tipo de ataque.
    % \begin{itemize}
    %     \item a. Un ataque que utiliza software malicioso para dañar sistemas.
    %     \item b. Un intento de obtener información confidencial haciéndose pasar por una entidad confiable.
    %     \item c. Un método para mejorar la seguridad de las contraseñas.
    %     \item d. No es un tipo de ataque relevante.
    % \end{itemize}
    %\textcolor{white}{Respuesta: b. Un intento de obtener información confidencial haciéndose pasar por una entidad confiable. Justificación: El phishing engaña a las personas para obtener información confidencial.}
    %\vspace{1cm}
    
    \item ¿Cuáles son las buenas prácticas más importantes para crear y almacenar contraseñas seguras? 
    Justifica por qué cada práctica es crucial para la protección de datos.
    \begin{itemize}
        \item a. Usar contraseñas cortas y fáciles de recordar.
        \item b. Cambiar contraseñas regularmente y usar combinaciones complejas.
        \item c. Compartir contraseñas con compañeros de trabajo para mayor seguridad.
        \item d. No es necesario preocuparse por la seguridad de las contraseñas.
    \end{itemize}
    %\textcolor{white}{Respuesta: b. Cambiar contraseñas regularmente y usar combinaciones complejas. Justificación: Contraseñas complejas y cambios regulares mejoran la seguridad de las cuentas.}
    %\vspace{1cm}
    
    \item ¿Qué es el malware y cuáles son los tipos más comunes? 
    Describe las características de al menos dos tipos de malware y cómo pueden afectar a un sistema.
    \begin{itemize}
        \item a. Software diseñado para proteger sistemas.
        \item b. Software malicioso que incluye virus y troyanos.
        \item c. Herramientas de seguridad para mejorar el rendimiento del sistema.
        \item d. No tiene impacto en los sistemas modernos.
    \end{itemize}
    %\textcolor{white}{Respuesta: b. Software malicioso que incluye virus y troyanos. Justificación: Virus y troyanos son tipos comunes de malware que pueden dañar sistemas.}
    %\vspace{1cm}
    
    \item Explica el papel de un firewall en la protección de una red. 
    ¿Cómo funciona y qué diferencias existen entre un firewall de hardware y uno de software?
    \begin{itemize}
        \item a. No tiene un papel importante en la seguridad de la red.
        \item b. Filtra el tráfico de red y protege contra accesos no autorizados.
        \item c. Solo los firewalls de hardware son efectivos.
        \item d. No hay diferencias entre firewalls de hardware y software.
    \end{itemize}
    %\textcolor{white}{Respuesta: b. Filtra el tráfico de red y protege contra accesos no autorizados. Justificación: Los firewalls filtran el tráfico y protegen la red de accesos no autorizados.}
    %\vspace{1cm}
    
    \item ¿Cuál es la función principal de una VPN (Red Privada Virtual) en un entorno de red?
    \begin{itemize}
        \item a) Cifrar todos los archivos almacenados en el dispositivo.
        \item b) Conectar una red local a otra red sin cifrado.
        \item c) Conectar una red con cifrado.
        \item d) Proporcionar acceso solo a la red local sin salir a internet.
        \item e) Mejorar la velocidad de conexión de internet.
        \item f) Evitar ataques de fuerza bruta en aplicaciones web.
    \end{itemize}
    %\textcolor{white}{Respuesta: c) Conectar una red con cifrado. Justificación: Una VPN conecta redes de forma segura mediante cifrado.}
    %\vspace{1cm}
    
    \item En relación a los ataques de Denegación de Servicio (DoS) y Denegación de Servicio Distribuida (DDoS):
    \begin{itemize}
        \item I. Un ataque DoS busca abrumar/saturar a un sistema con una única fuente de tráfico excesivo. - Verdadero
        \item II. Un ataque DDoS utiliza múltiples fuentes para generar un tráfico masivo coordinado. - D extra viene de "Distributed"
        \item III. Los ataques DoS y DDoS pueden ser mitigados únicamente con firewalls básicos. - Falso "únicamente", necesitamos más que solo un firewall básico
        \item IV. Hemos visto que los últimos ataques DDoS se componen de millones de paquetes por segundo, saturando el tráfico en su totalidad.
    \end{itemize}
    \begin{itemize}
        \item a) Solo I y II
        \item b) Solo I, II y IV
        \item c) Solo II y III
        \item d) I, II, III y IV
        \item e) Ninguna de las anteriores
    \end{itemize}
    %\textcolor{white}{Respuesta: b) Solo I, II y IV. Justificación: Las afirmaciones I, II y IV son correctas sobre DoS y DDoS.}
    %\vspace{1cm}
    
    \item Durante un análisis de tráfico de red con Wireshark, se observan numerosos paquetes SYN provenientes de una sola dirección IP sin las respuestas correspondientes (SYN/ACK). ¿Qué tipo de ataque es más probable que esté ocurriendo?
    \begin{itemize}
        \item a) Ataque Man-in-the-Middle
        \item b) Ataque de IP Spoofing
        \item c) Ataque de Phishing - Video, Audio, ya no solo texto.
        \item d) Ataque SYN Flood (DoS)
        \item e) Ninguna de las anteriores
    \end{itemize}
    %\textcolor{white}{Respuesta: d) Ataque SYN Flood (DoS). Justificación: Un SYN Flood envía múltiples solicitudes SYN sin completar el handshake, saturando la tabla de conexiones.}
    %\vspace{1cm}
    % \item Sobre el enfoque de seguridad Zero-Trust:
    % \begin{itemize}
    %     \item I. El principio de ``verificación explícita'' implica siempre autenticar y autorizar antes de conceder acceso.
    %     \item II. ``Acceso con privilegios mínimos'' significa que los usuarios tienen el acceso necesario para cumplir solo con sus tareas.
    %     \item III. Zero-Trust asume siempre que la red interna es segura. - Es lo opuesto! Desconfías de todos y de todo
    %     \item IV. La suposición de ``violation'' implica planificar como si el entorno ya hubiera sido comprometido.
    % \end{itemize}
    % \begin{itemize}
    %     \item a) Solo I y II
    %     \item b) I, II y IV
    %     \item c) Solo II y III
    %     \item d) I, II, III y IV
    %     \item e) Ninguna de las anteriores
    % \end{itemize}
    %\textcolor{white}{Respuesta: b) I, II y IV. Justificación: Zero-Trust se basa en verificación explícita, acceso mínimo y planificación para compromisos.}
    %\vspace{1cm}
    
    \item ¿Cuál de los siguientes controles de seguridad se clasifica como un control técnico?
    \begin{itemize}
        \item a) Políticas de acceso y uso aceptable
        \item b) Controles físicos como cerraduras y sistemas de vigilancia
        \item c) Autenticación de dos factores (2FA)
        \item d) Entrenamiento de concienciación en seguridad
        \item e) Evaluación de riesgos
    \end{itemize}
    %\textcolor{white}{Respuesta: c) Autenticación de dos factores (2FA). Justificación: 2FA es un control técnico que mejora la seguridad de acceso.}
    %\vspace{1cm}
    
    \item En el contexto de un ataque de "Man-in-the-Middle" (MITM):
    \begin{itemize}
        \item I. El atacante intercepta y potencialmente altera la comunicación entre dos partes sin que estas lo sepan.
        \item II. La autenticación mutua puede ayudar a prevenir los ataques MITM. - Validar comunicación + cifrado
        \item III. MITM es menos efectivo en conexiones HTTPS debido a la encriptación. - No evita que te estén escuchando - pero lo que ves es ilegible
        \item IV. Un MITM puede ser fácilmente detectado sin herramientas especializadas. - Necesitas herramientas como WireShark.
    \end{itemize}
    \begin{itemize}
        \item a) Solo I y II
        \item b) I, II y III
        \item c) Solo II y IV
        \item d) I, II, III y IV
        \item e) Ninguna de las anteriores
    \end{itemize}
    %\textcolor{white}{Respuesta: a) Solo I y II. Justificación: MITM intercepta comunicaciones y la autenticación mutua puede prevenirlo.}
    %\vspace{1cm}
    
    % \item ¿Cuál es la diferencia principal entre un firewall y un Web Application Firewall (WAF)?
    % \begin{itemize}
    %     \item a) Un firewall solo filtra tráfico HTTP y HTTPS, mientras que un WAF puede filtrar cualquier tráfico de red. - WAF ve solo HTTP y HTTPS (es la web) - Firewall incluye otros protocolos como SSH, Telnet…
    %     \item b) Un firewall protege las redes contra ataques de malware, mientras que un WAF protege aplicaciones web contra ataques como SQL injection y cross-site scripting (XSS). - Por lo general se requieren los dos, ya que resuelven casos de uso distintos.
    %     \item c) Un WAF es más fácil de configurar que un firewall tradicional. - No necesariamente. Igual de difíciles / fáciles.
    %     \item d) Un firewall es solo un software, mientras que un WAF es solo hardware. - Firewall y WAF pueden ser contratados como Software (as a Service). - FIrewall y WAF por detrás igualmente existen en físico - hardware.
    %     \item e) Ambos son equivalentes y se utilizan indistintamente en cualquier red. - No son lo mismo!
    % \end{itemize}
    %\textcolor{white}{Respuesta: b) Un firewall protege las redes contra ataques de malware, mientras que un WAF protege aplicaciones web contra ataques como SQL injection y cross-site scripting (XSS). Justificación: Los firewalls y WAFs tienen diferentes enfoques de protección.}
    %\vspace{1cm}
    
    % \item En la fase de 'Mantener el Acceso' durante un ataque cibernético, el atacante puede:
    % \begin{itemize}
    %     \item I. Instalar backdoors para asegurar el acceso continuo.
    %     \item II. Utilizar rootkits para ocultar su presencia en el sistema.
    %     \item III. Explotar vulnerabilidades para ganar acceso inicial al sistema. - Estamos en la fase donde ya tengo acceso
    %     \item IV. Borrar logs para evitar la detección. - No afecta el hecho de tener acceso
    % \end{itemize}
    % \begin{itemize}
    %     \item a) Solo I y II
    %     \item b) I, II y IV
    %     \item c) Solo III y IV
    %     \item d) I, II, III y IV
    %     \item e) Ninguna de las anteriores
    % \end{itemize}
    % %\textcolor{white}{Respuesta: a) Solo I y II. Justificación: Durante la fase de mantener el acceso, los atacantes instalan backdoors y rootkits.}
    % %\vspace{1cm}
    
    \item Un atacante envía correos electrónicos fraudulentos que parecen provenir de un banco legítimo, solicitando a los destinatarios que ingresen sus credenciales de inicio de sesión en un sitio web falso que parece auténtico.
    ¿Qué vector de ataque describe mejor este escenario?
    \begin{itemize}
        \item a) Phishing
        \item b) Ataque de diccionario
        \item c) Ransomware
        \item d) Malware
        \item e) Keylogging
    \end{itemize}
    %\textcolor{white}{Respuesta: a) Phishing. Justificación: El phishing utiliza correos fraudulentos para engañar a las personas y obtener sus credenciales.}
    %\vspace{1cm}
    
    \item Un atacante utiliza un programa malicioso que registra las pulsaciones de teclas de un usuario y envía esta información a un servidor remoto sin que el usuario lo sepa.
    ¿Qué vector de ataque se está utilizando aquí?
    \begin{itemize}
        \item a) Ataque Man-in-the-Middle (MITM)
        \item b) Keylogging - No se puede confiar ni en el teclado (puede venir instalado el malware)
        \item c) Spoofing
        %\item d) SQL Injection
        %\item e) Cross-Site Scripting (XSS)
    \end{itemize}
    %\textcolor{white}{Respuesta: b) Keylogging. Justificación: El keylogging registra las pulsaciones de teclas y envía la información a un servidor remoto.}
    %\vspace{1cm}
    
    \item Una empresa descubre que un atacante ha explotado una vulnerabilidad en su software de servidor web, permitiendo al atacante ejecutar comandos arbitrarios en el servidor, obteniendo acceso no autorizado a la base de datos.
    ¿Qué vector de ataque describe mejor este escenario?
    \begin{itemize}
        \item a) Phishing
        \item b) Ejecución remota de código (RCE)
        \item c) Fuerza bruta
        \item d) Ataque de repetición (replay attack)
        \item e) Exfiltración de datos
    \end{itemize}
    %\textcolor{white}{Respuesta: b) Ejecución remota de código (RCE). Justificación: La ejecución remota de código permite al atacante ejecutar comandos arbitrarios en el servidor.}
    %\vspace{1cm}
    
    \item ¿Qué es la seguridad computacional?
    \begin{itemize}
        \item a) Protección de sistemas informáticos contra amenazas físicas y lógicas.
        \item b) Protección de la información almacenada digitalmente.
        \item c) Protección de redes contra ataques cibernéticos.
        \item d) Ninguna de las anteriores.
    \end{itemize}
    %\textcolor{white}{Respuesta: a) Protección de sistemas informáticos contra amenazas físicas y lógicas. Justificación: La seguridad computacional se centra en proteger sistemas informáticos de amenazas físicas y lógicas.}
    %\vspace{1cm}
    
    \item ¿Cuál es el enfoque principal de la ciberseguridad?
    \begin{itemize}
        \item a) Proteger la información en formato físico.
        \item b) Proteger sistemas informáticos y redes contra ataques cibernéticos.
        \item c) Proteger la confidencialidad de los datos.
        \item d) Proteger la integridad de los datos.
    \end{itemize}
    %\textcolor{white}{Respuesta: b) Proteger sistemas informáticos y redes contra ataques cibernéticos. Justificación: La ciberseguridad se enfoca en proteger sistemas y redes de ataques cibernéticos.}
    %\vspace{1cm}
    
    \item ¿Qué es un evento según el NIST SP 800?
    \begin{itemize}
        \item a) Un ataque cibernético exitoso.
        \item b) Cualquier ocurrencia observable en una red o sistema.
        \item c) Una vulnerabilidad en un sistema de información.
        \item d) Un acceso no autorizado a un sistema.
    \end{itemize}
    %\textcolor{white}{Respuesta: b) Cualquier ocurrencia observable en una red o sistema. Justificación: Según el NIST SP 800, un evento es cualquier ocurrencia observable en una red o sistema.}
    %\vspace{1cm}
    
    \item ¿Qué es un exploit?
    \begin{itemize}
        \item a) Un tipo de malware que se propaga automáticamente.
        \item b) Un ataque que aprovecha vulnerabilidades del sistema.
        \item c) Un programa que protege contra malware.
        \item d) Un método de cifrado de datos.
    \end{itemize}
    %\textcolor{white}{Respuesta: b) Un ataque que aprovecha vulnerabilidades del sistema. Justificación: Un exploit es un ataque que aprovecha vulnerabilidades del sistema para comprometerlo.}
    %\vspace{1cm}
    
    \item ¿Qué garantiza la disponibilidad en seguridad informática?
    \begin{itemize}
        \item a) Que la información esté protegida contra accesos no autorizados.
        \item b) Que los datos no sean alterados de manera no autorizada.
        \item c) Que la información esté accesible cuando se necesite.
        \item d) Que los datos estén cifrados.
    \end{itemize}
    %\textcolor{white}{Respuesta: c) Que la información esté accesible cuando se necesite. Justificación: La disponibilidad garantiza que la información esté accesible cuando se necesite.}
    %\vspace{1cm}
    
    \item ¿Qué es la autenticación multifactor (MFA)?
    \begin{itemize}
        \item a) Uso de contraseñas y PINs para acceder a sistemas.
        \item b) Combinación de contraseñas con tokens o biometría.
        \item c) Uso de tokens físicos para acceder a áreas críticas.
        \item d) Uso de reconocimiento facial para autenticar usuarios.
    \end{itemize}
    %\textcolor{white}{Respuesta: b) Combinación de contraseñas con tokens o biometría. Justificación: La autenticación multifactor combina múltiples métodos de autenticación para aumentar la seguridad.}
    %\vspace{1cm}
    
    \item ¿Qué es un ataque de fuerza bruta?
    \begin{itemize}
        \item a) Un ataque que utiliza software malicioso para dañar sistemas.
        \item b) Un intento de obtener información confidencial haciéndose pasar por una entidad confiable.
        \item c) Un método para mejorar la seguridad de las contraseñas.
        \item d) Un ataque que prueba múltiples combinaciones de contraseñas.
    \end{itemize}
    %\textcolor{white}{Respuesta: d) Un ataque que prueba múltiples combinaciones de contraseñas. Justificación: Un ataque de fuerza bruta prueba múltiples combinaciones de contraseñas para acceder a un sistema.}
    %\vspace{1cm}

    % \item ¿Qué es un hash en seguridad informática?
    % \begin{itemize}
    %     \item a) Un método de cifrado de datos.
    %     \item b) Una función que convierte una entrada en una salida de longitud fija.
    %     \item c) Un tipo de malware que se propaga automáticamente.
    %     \item d) Un programa que protege contra malware.
    % \end{itemize}
    % %\textcolor{white}{Respuesta: b) Una función que convierte una entrada en una salida de longitud fija. Justificación: Un hash es una función que convierte una entrada en una salida de longitud fija, utilizada para verificar la integridad de los datos.}
    % %\vspace{1cm}

    % \item ¿Qué es un ataque de phishing?
    % \begin{itemize}
    %     \item a) Un ataque que utiliza software malicioso para dañar sistemas.
    %     \item b) Un intento de obtener información confidencial haciéndose pasar por una entidad confiable.
    %     \item c) Un método para mejorar la seguridad de las contraseñas.
    %     \item d) Un ataque que prueba múltiples combinaciones de contraseñas.
    % \end{itemize}
    %\textcolor{white}{Respuesta: b) Un intento de obtener información confidencial haciéndose pasar por una entidad confiable. Justificación: El phishing engaña a las personas para obtener información confidencial.}
    %\vspace{1cm}
    
    \item ¿Qué es un ataque de denegación de servicio (DoS)?
    \begin{itemize}
        \item a) Un ataque que utiliza software malicioso para dañar sistemas.
        \item b) Un intento de obtener información confidencial haciéndose pasar por una entidad confiable.
        \item c) Un ataque que busca agotar los recursos de un sistema.
        \item d) Un ataque que prueba múltiples combinaciones de contraseñas.
    \end{itemize}
    %\textcolor{white}{Respuesta: c) Un ataque que busca agotar los recursos de un sistema. Justificación: Un ataque DoS busca agotar los recursos de un sistema para interrumpir su funcionamiento.}
    %\vspace{1cm}

    % \textbf{Preguntas de Verdadero/Falso}

    % \item La seguridad de la información se enfoca en la protección de la información almacenada digitalmente. (Falso)
    % %\textcolor{white}{Justificación: La seguridad de la información abarca la protección de la información en todos los formatos, no solo digitalmente.}
    % %\vspace{1cm}

    % \item Un incidente es un evento que pone en peligro la confidencialidad, integridad o disponibilidad de un sistema de información. (Verdadero)
    % %\textcolor{white}{Justificación: Un incidente afecta la confidencialidad, integridad o disponibilidad de un sistema de información.}
    % %\vspace{1cm}

    % \item La confidencialidad se refiere a la protección de la información contra accesos no autorizados. (Verdadero)
    % %\textcolor{white}{Justificación: La confidencialidad asegura que solo las partes autorizadas puedan acceder a la información.}
    % %\vspace{1cm}

    % \item La integridad asegura que los datos no sean alterados de manera no autorizada. (Verdadero)
    % %\textcolor{white}{Justificación: La integridad garantiza que los datos se mantengan sin cambios no autorizados.}
    % %\vspace{1cm}

    % \item La disponibilidad garantiza que la información esté accesible cuando se necesite. (Verdadero)
    % %\textcolor{white}{Justificación: La disponibilidad asegura que la información esté disponible para su uso cuando se requiera.}
    % %\vspace{1cm}

    % \item La autenticación es un proceso que verifica la identidad de un usuario o sistema. (Verdadero)
    % %\textcolor{white}{Justificación: La autenticación verifica la identidad de usuarios o sistemas para asegurar el acceso autorizado.}
    % %\vspace{1cm}

    % \item Las contraseñas son el método de autenticación más utilizado, pero presentan varios riesgos de seguridad. (Verdadero)
    % %\textcolor{white}{Justificación: Las contraseñas son comunes pero pueden ser vulnerables a ataques como el phishing y la fuerza bruta.}
    % %\vspace{1cm}

    % \item Un hash es una función que convierte una entrada en una salida de longitud fija. (Verdadero)
    % %\textcolor{white}{Justificación: Un hash convierte una entrada en una salida de longitud fija, útil para verificar la integridad de los datos.}
    % %\vspace{1cm}

    % \item El phishing es una técnica en la que los atacantes engañan a los usuarios para que revelen sus credenciales. (Verdadero)
    % %\textcolor{white}{Justificación: El phishing engaña a los usuarios para obtener sus credenciales mediante correos o sitios falsos.}
    % %\vspace{1cm}

    % \item Un ataque de denegación de servicio (DoS) busca agotar los recursos de un sistema. (Verdadero)
    % %\textcolor{white}{Justificación: Un ataque DoS agota los recursos de un sistema para interrumpir su funcionamiento.}
    % %\vspace{1cm}

    % \textbf{Preguntas Adicionales}

    % \item ¿Qué es un ataque de inyección SQL?
    % \begin{itemize}
    %     \item a) Un ataque que utiliza software malicioso para dañar sistemas.
    %     \item b) Un intento de obtener información confidencial haciéndose pasar por una entidad confiable.
    %     \item c) Un ataque que inserta código SQL malicioso en un campo de entrada.
    %     \item d) Un ataque que prueba múltiples combinaciones de contraseñas.
    % \end{itemize}
    % %\textcolor{white}{Respuesta: c) Un ataque que inserta código SQL malicioso en un campo de entrada. Justificación: La inyección SQL inserta código malicioso en campos de entrada para manipular bases de datos.}
    % %\vspace{1cm}
    
    % \item ¿Qué es un ataque de botnet?
    % \begin{itemize}
    %     \item a) Un ataque que utiliza software malicioso para dañar sistemas.
    %     \item b) Un intento de obtener información confidencial haciéndose pasar por una entidad confiable.
    %     \item c) Un ataque que utiliza una red de computadoras infectadas para realizar ataques coordinados.
    %     \item d) Un ataque que prueba múltiples combinaciones de contraseñas.
    % \end{itemize}
    % %\textcolor{white}{Respuesta: c) Un ataque que utiliza una red de computadoras infectadas para realizar ataques coordinados. Justificación: Un botnet utiliza múltiples dispositivos infectados para realizar ataques coordinados.}
    % %\vspace{1cm}
    
    % \item ¿Qué es un ataque de ransomware?
    % \begin{itemize}
    %     \item a) Un ataque que utiliza software malicioso para dañar sistemas.
    %     \item b) Un intento de obtener información confidencial haciéndose pasar por una entidad confiable.
    %     \item c) Un ataque que cifra los datos de un sistema y exige un rescate para liberarlos.
    %     \item d) Un ataque que prueba múltiples combinaciones de contraseñas.
    % \end{itemize}
    % %\textcolor{white}{Respuesta: c) Un ataque que cifra los datos de un sistema y exige un rescate para liberarlos. Justificación: El ransomware cifra datos y exige un rescate para liberarlos.}
    % %\vspace{1cm}
    
    % \item ¿Qué es un ataque de ingeniería social?
    % \begin{itemize}
    %     \item a) Un ataque que utiliza software malicioso para dañar sistemas.
    %     \item b) Un intento de manipular a las personas para que revelen información confidencial.
    %     \item c) Un ataque que cifra los datos de un sistema y exige un rescate para liberarlos.
    %     \item d) Un ataque que prueba múltiples combinaciones de contraseñas.
    % \end{itemize}
    % %\textcolor{white}{Respuesta: b) Un intento de manipular a las personas para que revelen información confidencial. Justificación: La ingeniería social manipula a las personas para obtener información confidencial.}
    % %\vspace{1cm}
    
    \item ¿Qué es un ataque de spoofing?
    \begin{itemize}
        \item a) Un ataque que utiliza software malicioso para dañar sistemas.
        \item b) Un intento de obtener información confidencial haciéndose pasar por una entidad confiable.
        \item c) Un ataque que falsifica la identidad de un dispositivo o usuario.
        \item d) Un ataque que prueba múltiples combinaciones de contraseñas.
    \end{itemize}
    %\textcolor{white}{Respuesta: c) Un ataque que falsifica la identidad de un dispositivo o usuario. Justificación: El spoofing falsifica identidades para engañar a sistemas o usuarios.}
    %\vspace{1cm}
    % \item ¿Qué es un ataque de día cero?
    % \begin{itemize}
    %     \item a) Un ataque que utiliza software malicioso para dañar sistemas.
    %     \item b) Un intento de obtener información confidencial haciéndose pasar por una entidad confiable.
    %     \item c) Un ataque que explota una vulnerabilidad del sistema previamente desconocida.
    %     \item d) Un ataque que prueba múltiples combinaciones de contraseñas.
    % \end{itemize}
    % %\textcolor{white}{Respuesta: c) Un ataque que explota una vulnerabilidad del sistema previamente desconocida. Justificación: Un ataque de día cero explota vulnerabilidades desconocidas antes de que se publiquen parches.}
    % %\vspace{1cm}
    
    % \item ¿Qué es un ataque de denegación de servicio distribuida (DDoS)?
    % \begin{itemize}
    %     \item a) Un ataque que utiliza software malicioso para dañar sistemas.
    %     \item b) Un intento de obtener información confidencial haciéndose pasar por una entidad confiable.
    %     \item c) Un ataque que utiliza múltiples sistemas para agotar los recursos de un sistema objetivo.
    %     \item d) Un ataque que prueba múltiples combinaciones de contraseñas.
    % \end{itemize}
    % %\textcolor{white}{Respuesta: c) Un ataque que utiliza múltiples sistemas para agotar los recursos de un sistema objetivo. Justificación: Un DDoS utiliza múltiples sistemas para agotar los recursos de un sistema objetivo.}
    % %\vspace{1cm}
    
    % \item ¿Qué es un ataque de cross-site scripting (XSS)?
    % \begin{itemize}
    %     \item a) Un ataque que utiliza software malicioso para dañar sistemas.
    %     \item b) Un intento de obtener información confidencial haciéndose pasar por una entidad confiable.
    %     \item c) Un ataque que inserta scripts maliciosos en páginas web para ejecutar en el navegador del usuario.
    %     \item d) Un ataque que prueba múltiples combinaciones de contraseñas.
    % \end{itemize}
    % %\textcolor{white}{Respuesta: c) Un ataque que inserta scripts maliciosos en páginas web para ejecutar en el navegador del usuario. Justificación: XSS inserta scripts maliciosos en páginas web para ejecutar en el navegador del usuario.}
    % %\vspace{1cm}

    \item ¿Qué es un ataque de man-in-the-middle (MitM)?
    \begin{itemize}
        \item a) Un ataque que utiliza software malicioso para dañar sistemas.
        \item b) Un intento de obtener información confidencial haciéndose pasar por una entidad confiable.
        \item c) Un ataque que intercepta y potencialmente altera la comunicación entre dos partes.
        \item d) Un ataque que prueba múltiples combinaciones de contraseñas.
    \end{itemize}
    %\textcolor{white}{Respuesta: c) Un ataque que intercepta y potencialmente altera la comunicación entre dos partes. Justificación: Un ataque MitM intercepta y potencialmente altera la comunicación entre dos partes.}
    %\vspace{1cm}
\end{enumerate}


  

\end{document} 